\documentclass[]{article}
\usepackage[spanish.mexico]{babel}
\usepackage[T1]{fontenc}
\usepackage[utf8]{inputenc}
\usepackage{lmodern}
\usepackage[a4paper]{geometry}

%\usepackage{natbib}
\usepackage{cite}


%Grafico de barras
\usepackage{pgfplots}

%Graficos e imagenes
\usepackage{graphicx}

%URL
\usepackage{hyperref}


\title{México RLUG (LEGO® Users Group/Grupo de Usuarios de LEGO®)}
%\author{Pablo Vivar Colina}
\date{Noviembre 2018}

%%\usepackage[top=2cm,bottom=2cm,left=1cm,right=1cm]{geometry}


\begin{titlepage}
     \begin{center}
	\includegraphics[width=0.09\textwidth]{UNAM}\Large Universidad Nacional Autónoma de México
        	\includegraphics[width=0.09\textwidth]{FI}\\[1cm]
        \Large Facultad de Ingeniería\\[1cm]
       % \Large División de Ciencias Básicas\\[1cm]
         \Large Laboratorio de Dispositivos y Circuitos Electrónicos (6654)\\[1cm]
         %la clave antes era:4314
         \footnotesize Profesor: Zapata Rosales Arturo Ing.\\[1cm]
        \footnotesize Semestre 2018-1\\[1cm]
        %\Large Práctica No. 1\\[1cm]
    
        %\Large Práctica No. 2\\[1cm]
        
        %\Large Práctica No. 3\\[1cm]
       
        %\Large Práctica No. 4\\[1cm]
         
               
         %\Large Práctica No. 5\\[1cm]
         
         
         %\Large Práctica No. 6\\[1cm]
         
         %\Large Práctica No. 7\\[1cm]
         
             %\Large Práctica No. 8\\[1cm]
       

        \Large Práctica No. 9\\[1cm]
        
           %####AQUI VAMOS#### ya ahora sii
           
        %\Large Práctica No. 11\\[1cm]
        %\Large Práctica No. 12\\[1cm]
        %\Large Práctica No. 13\\[1cm]
        
        %\Large Amplificador Operacional como Integrador\\[1cm]
        %\Large{Filtros}\\[1cm]
         %\Large{Medición de  corrientes en un circuito}\\[1cm]
         %practica 4
         %Large{Amplificador operacional como seguidor de voltaje en entrada inversora}\\[1cm]
         %practica5
         %\Large{Amplificador operacional como integrador}
         
         %Practica 7
%Comportamiento de un diodo Zener

\Large Diodo Zener
        
         %Texto a la derecha
          \begin{flushright}
\footnotesize  Grupo 13\\[0.5cm]
\footnotesize Brigada: 7\\[0.5cm]

\footnotesize Vivar Colina Pablo\\[0.5cm]
 \end{flushright}
    %Texto a la izquierda
          \begin{flushleft}
        \footnotesize Ciudad Universitaria Abril de 2018.\\
          \end{flushleft}
         
          
        %\vfill
        %\today
   \end{center}
\end{titlepage}
 %agregar portada

\begin{document}

\maketitle



*\begin{figure}[h!]
	\centering
	\includegraphics[width=0.9\textwidth]{BannerA}
	%\caption{Banner}
	%\label{fig:lámparas}
\end{figure}




\section{¿Quiénes Somos?}

México LEGO Users Group, Un grupo de Adultos coleccionistas de LEGO sin fines de lucro, conformado por fans de la marca, dedicados a la construcción de dioramas con diversos temas utilizando bloques y figuras de LEGO.

\section{Misión}

Nuestra misión es crear un espacio donde los fans, y AFOLs (Adultos Fan de LEGO) puedan compartir sus creaciones, sus colecciones, y acercar el hobby a más personas.\\

Así como exposiciones y actividades que nos permitan crecer como comunidad, actividades que fomenten valores, mediante programas de creación por conceptos (Concursos de Fotografía, de armado de escenas, etc.) como el cuidado del medio ambiente, valores empáticos, y responsabilidad social.
Nuestro público va desde los mas pequeños, hasta los mas grandes, por lo que desarrollamos diferentes actividades dentro de nuestro grupo para estos públicos.\\

Nuestras actividades incluyen la realización de muestras de colecciones, actividades recreativas, concursos y rifas, que puedan potenciar nuestra relación con la comunidad.

\section{Propuesta de Proyecto}

Uno de nuestros objetivos es fomentar el “Fair Play” o “Juego Justo”, por lo que generamos actividades, que potencien este juego, a través de diversas actividades dirigidas a los niños, mediante la implementación de la enseñanza de valores, e impulsando su creatividad con actividades recreativas.\\

Nuestra propuesta es generar una muestra de diversas colecciones: se mostrarán colecciones de Licencias de LEGO, City (la vida en la ciudad)Space (relacionado al espacio y a la Ciencia Ficción) así como Licencias de diversos temas, como Star Wars, Super Héroes Marvel y DC, y el Señor de los Anillos, así como también creaciones propias, que abarcan diversos temas como Historia, Ciencia Ficción, y vida diaria.\\


Dentro de las actividades que nuestro grupo realiza están:


\subsection{Muestra de colecciones de LEGO}

En esta muestra se lleva a cabo la exhibición de diferentes dioramas por parte de los miembros del grupo.

%\subsection{Actividad Fair Play y Valores}

%Se disponde de material para colorear, relacionado a ciertos valores, como compartir, el reciclado, el cuidado del medio ambiente, y el buen uso del espacio público.\\

%Las hojas que coloreen, pueden llevarsela a casa los niños, alternativamente se puede armar un periódico mural donde los niños coloquen sus dibujos, y al final de la muestra pueden retirarlos.\\


\subsection{Proyección de Películas}

Contamos con una pequeña colección de Películas de LEGO, estas pueden ser proyectadas durante la muestra.


\subsection{Zona de armado}

La zona de armado consta en dos partes, la zona Duplo, y la Zona System, dependiendo del lugar asignado, se utiliza una u otra.\\

Junto con la zona de armado se realiza la actividad de Armado Rápido, que consta en una serie de tarjetas con imágenes, y que los participantes armen lo más rápido con los bloques, su versión de la imagen.\\


\section{Facebook}

\begin{itemize}
\item	 https://www.facebook.com/mexicorlug/
\item	 https://www.facebook.com/groups/MexicoRLUG/ 
\end{itemize}


%\url{ https://www.facebook.com/mexicorlug/}
%\url{https://www.facebook.com/groups/MexicoRLUG/}
%\bibliographystyle{plain}
%\bibliography{Referencias.bib}
%\addbibresource{Referencias.bib}
%\begin{thebibliography}{widestlabel}
	%\bibitem{EnergiaWiki}\textsc{Wikipedia}\textsc{Energia},\textit{},WikimediaGroup.
		
%\end{thebibliography}

\end{document}
