\documentclass[]{article}
\usepackage[spanish.mexico]{babel}
\usepackage[T1]{fontenc}
\usepackage[utf8]{inputenc}
%\usepackageB.\\{lmodern}
\usepackage[a4paper]{geometry}

%\usepackage{natbib}
\usepackage{cite}


%Grafico de barras
\usepackage{pgfplots}

%Graficos e imagenes
\usepackage{graphicx}


\title{Plantas y generación de energía}
\author{Pablo Vivar Colina}
%\date{Mayo 2018}



\begin{document}
	
%	%\usepackage[top=2cm,bottom=2cm,left=1cm,right=1cm]{geometry}


\begin{titlepage}
     \begin{center}
	\includegraphics[width=0.09\textwidth]{UNAM}\Large Universidad Nacional Autónoma de México
        	\includegraphics[width=0.09\textwidth]{FI}\\[1cm]
        \Large Facultad de Ingeniería\\[1cm]
       % \Large División de Ciencias Básicas\\[1cm]
         \Large Laboratorio de Dispositivos y Circuitos Electrónicos (6654)\\[1cm]
         %la clave antes era:4314
         \footnotesize Profesor: Zapata Rosales Arturo Ing.\\[1cm]
        \footnotesize Semestre 2018-1\\[1cm]
        %\Large Práctica No. 1\\[1cm]
    
        %\Large Práctica No. 2\\[1cm]
        
        %\Large Práctica No. 3\\[1cm]
       
        %\Large Práctica No. 4\\[1cm]
         
               
         %\Large Práctica No. 5\\[1cm]
         
         
         %\Large Práctica No. 6\\[1cm]
         
         %\Large Práctica No. 7\\[1cm]
         
             %\Large Práctica No. 8\\[1cm]
       

        \Large Práctica No. 9\\[1cm]
        
           %####AQUI VAMOS#### ya ahora sii
           
        %\Large Práctica No. 11\\[1cm]
        %\Large Práctica No. 12\\[1cm]
        %\Large Práctica No. 13\\[1cm]
        
        %\Large Amplificador Operacional como Integrador\\[1cm]
        %\Large{Filtros}\\[1cm]
         %\Large{Medición de  corrientes en un circuito}\\[1cm]
         %practica 4
         %Large{Amplificador operacional como seguidor de voltaje en entrada inversora}\\[1cm]
         %practica5
         %\Large{Amplificador operacional como integrador}
         
         %Practica 7
%Comportamiento de un diodo Zener

\Large Diodo Zener
        
         %Texto a la derecha
          \begin{flushright}
\footnotesize  Grupo 13\\[0.5cm]
\footnotesize Brigada: 7\\[0.5cm]

\footnotesize Vivar Colina Pablo\\[0.5cm]
 \end{flushright}
    %Texto a la izquierda
          \begin{flushleft}
        \footnotesize Ciudad Universitaria Abril de 2018.\\
          \end{flushleft}
         
          
        %\vfill
        %\today
   \end{center}
\end{titlepage}
 %agregar portada

\maketitle

%\tableofcontents  % Write out the Table of Contents

%\listoffigures  % Write out the List of Figures



\section{Campo magnético Giratorio}

Es un campo magnético que rota a una velocidad uniforme (idealmente) y generado a partir de una corriente alterna trifásica.\\

Fué descubierto por Galileo Ferraris en 1885 y es el fenómeno sobre el cual se fundamentan las máquinas de corriente alterna.\\

\begin{figure}[h!]
	\centering
	
	\begin{tikzpicture}
	\begin{axis}[
	axis lines = left,
	xlabel = {t[s]},
	ylabel = {V[V]},
	]
	
	\addplot
	[thick=0.1cm,
	domain=0:1, 
	samples=300, 
	color=green,
	]
	{sin(2*3.14159*x*60)};
	
	
	\addplot
	[thick=0.1cm,
	domain=0:1, 
	samples=300, 
	color=red,
	]
	{sin((2*3.14159*x*60)+120)};
	
	
	
	\addplot
	[thick=0.1cm,
	domain=0:1, 
	samples=300, 
	color=blue,
	]
	{sin((2*3.14159*x*60)+240)};
	
	%Se añade nota :D
	\addlegendentry{Va}
	\addlegendentry{Vb}
	\addlegendentry{Vc}
	
	\end{axis}
	\end{tikzpicture}
	\caption{Sistema Trifásico}
	\label{sitemaTrifasico}
\end{figure}


Para el T1 es en 0.25 segundos, T2 es en 0.4 y T3 en 0.55$  $ segundos aproximadamente.\\

\section{Primer cálculo T1}

\begin{figure}[h!]
	\centering
	
	\begin{tikzpicture}
	\begin{axis}[
	axis lines = left,
	xlabel = {r},
	ylabel = {J},
	]
	
	\addplot
	[thick=0.001cm,
	domain=0:1, 
	samples=3, 
	color=black,
	]
	{x};
	
	

	%Se añade nota :D
	%\addlegendentry{Va}
	%\addlegendentry{Vb}
	%\addlegendentry{Vc}
	
	\end{axis}
	
	%Flecha afuera Ba
	\draw[thick=0.5cm,
	color=red,
	]
	(0,0)--(0,6);
	
	
	%Flecha afuera Bb con 60 grados
	\draw[thick=0.5cm,
	color=red,
	]
	(0,0)--(6,3);
	
	
	%Flecha afuera S con 210 grados
	\draw[thick=0.5cm,
	color=red,
	]
	(0,0)--(-6,-3);
	
		%Flecha afuera Bt con 150 grados
	\draw[thick=0.5cm,
	color=red,
	]
	(0,0)--(-6,3);
	
	
	\end{tikzpicture}
	\caption{}
	\label{}
\end{figure}

\begin{equation}
    B_a=0+J
\end{equation}


\begin{equation}
B_b=\frac{\sqrt{3}}{4}+\frac{1}{4}J
\end{equation}


\begin{equation}
B_c=-\frac{\sqrt{3}}{4}+\frac{1}{4}J
\end{equation}


\begin{equation}
B_T=B_a+B_b+B_c=1.5J
\end{equation}


\section{Segundo cálculo T2}


\begin{figure}[h!]
	\centering
	
	\begin{tikzpicture}
	\begin{axis}[
	axis lines = left,
	xlabel = {r},
	ylabel = {J},
	]
	
	\addplot
	[thick=0.001cm,
	domain=0:1, 
	samples=3, 
	color=black,
	]
	{x};
	
	
	
	%Se añade nota :D
	%\addlegendentry{Va}
	%\addlegendentry{Vb}
	%\addlegendentry{Vc}
	
	\end{axis}
	
	%Flecha afuera Ba magnitud 0.5
	\draw[thick=0.5cm,
	color=red,
	]
	(0,0)--(0,3);
	
	
	%Flecha afuera Bc con 120 grados magnitud 1
	\draw[thick=0.5cm,
	color=red,
	]
	(0,0)--(-3,6);
	
	
	%Flecha afuera Bb con 210 grados
	\draw[thick=0.5cm,
	color=red,
	]
	(0,0)--(-6,-3);
	
	
	
	\end{tikzpicture}
	\caption{}
	\label{}
\end{figure}

\begin{equation}
B_a=0+0.5J
\end{equation}


\begin{equation}
B_c=\frac{\sqrt{3}}{2}+\frac{1}{2}J
\end{equation}


\begin{equation}
B_b=\frac{\sqrt{3}}{4}-\frac{1}{4}J
\end{equation}


\begin{equation}
B_T=-3\frac{\sqrt{3}}{4}+\frac{3}{4}J
\end{equation}

\begin{equation}
B_T=\sqrt{(\frac{\sqrt{3}}{4})^2+(\frac{3}{4})^2}
\end{equation}


\begin{equation}
B_T=\frac{3}{2}\angle 60^0
\end{equation}

\section{Calculo para T3}



\begin{equation}
B_a=\frac{1}{4}J
\end{equation}


%Angulo de 210 magnitud de 1
\begin{equation}
   B_b=-\frac{\sqrt{3}}{2}-\frac{1}{2}J
\end{equation}

%magnitud de 0.5 con angulo de 120
\begin{equation}
B_c=-\frac{\sqrt{3}}{4}+\frac{1}{4}J
\end{equation}


\section{Apendice}

La máquina síncrona conserva la velocidad del rotor a pesar de tener una carga, el motor de inducción no conserva ésta propiedad.\\

\section{Referencias}

%\bibliographystyle{plain}
%\bibliography{Prac1}



\begin{thebibliography}{9}
	%\bibitem{latexcompanion} 
	%Michel Goossens, Frank Mittelbach, and Alexander Samarin. 
	%\textit{The \LaTeX\ Companion}. 
	%Addison-Wesley, Reading, Massachusetts, 1993.
	
	%\bibitem{einstein} 
	%Albert Einstein. 
	%\textit{Zur Elektrodynamik bewegter K{\"o}rper}. (German) 
	%[\textit{On the electrodynamics of moving bodies}]. 
	%Annalen der Physik, 322(10):891–921, 1905.
	
	
	%\bibitem{ebullicion} 
	 %Punto ebullición,
	%\\\texttt{http://www.cie.unam.mx/~ojs/pub/Liquid3/node8.html}
	
	
	%\bibitem{coccion} 
   %UnComo:Tiempo de cocción de los frijoles,
	%\\\texttt{https://comida.uncomo.com/articulo/tiempo-de-coccion-de-los-frijoles-34777.html}
\end{thebibliography}






\end{document}
