\documentclass[]{article}
\usepackage[spanish.mexico]{babel}
\usepackage[T1]{fontenc}
\usepackage[utf8]{inputenc}
\usepackage{lmodern}
\usepackage[a4paper]{geometry}

%\usepackage{natbib}
\usepackage{cite}


%Grafico de barras
\usepackage{pgfplots}


\title{Mantenimiento de repositorios con Git en Debian o Ubuntu}
\author{Pablo Vivar Colina}
\date{Julio 2018}

\begin{document}

\maketitle

\section{Introducción}

El software libre es genial, sobretodo sus sistemas operativos, los sistemas GNU/Linux vienen con muchas variaciones, pero las más utilizadas en el mercado son las basadas en Ubuntu y Debian.\\

\section{Debian y Ubuntu}

\subsection{Debian}

Debian es un sistema operativo (S.O.) libre, para su computadora. El sistema operativo es el conjunto de programas básicos y utilidades que hacen que funcione su computadora.\\

Debian ofrece más que un S.O. puro; viene con 51000 paquetes, programas precompilados distribuidos en un formato que hace más fácil la instalación en su computadora.\\

\subsection{Ubuntu}

Ubuntu viene con todo lo que necesitas para dirigir tu organización, escuela, hogar o empresa. Todas las aplicaciones esenciales, como una suite de oficina, navegadores, correo electrónico y aplicaciones multimedia vienen preinstaladas y miles de juegos y aplicaciones más están disponibles en el Centro de Software de Ubuntu.\\

Ubuntu siempre ha sido libre de descargar, usar y compartir. Creemos en el poder del software de código abierto; Ubuntu no podría existir sin su comunidad mundial de desarrolladores voluntarios.\\

Con un firewall incorporado y un software de protección antivirus, Ubuntu es uno de los sistemas operativos más seguros del mercado. Y las versiones de soporte a largo plazo le ofrecen cinco años de parches y actualizaciones de seguridad.\\

La informática es para todos, independientemente de su nacionalidad, sexo o discapacidad. Ubuntu está totalmente traducido a más de 50 idiomas e incluye tecnologías de asistencia esenciales.\cite{UbuntuIntro}Traducción realizada con el traductor DeepL \cite{DeepL}\\

\subsection{Superusuario}

El superusuario en sistemas libres es una "herramienta" que nos permite realizar ciertas acciones con más permisos en la computadora que los que podriamos realizar con los permisos por defecto\\

\begin{enumerate}
	\item sudo <comando> $Ubuntu$
	
	\item su (acceso terminal root) $Debian$
\end{enumerate}

\subsection{Navegación Directorios}

En Debian y Ubuntu con los entornos de escritorios por defecto tenemos acceso a navegación de directorios en entornos gráficos, por ejemplo en Ubunto con gnome tenemos el navegador $nautilus$ y en Debian con XFCE tenemos el navegador $thunar$, éste último podemos accesar a él a través de invocarlo con su nombre desde la terminal.\\

Para navegar entre carpetas en la terminal podemos utilizar los siguientes comandos.\\

\begin{enumerate}
	\item ls (muestra el contenido en la carpeta en la cual nos encontramos)
	\item cd <carpeta> (accesa a carpeta)
	\item cd .. (sale de carpeta)
\end{enumerate}

Es importante mencionar que con la tecla <TAB> podemos autocompletar nombres de directorios, programas, accesos, etc. y que con el comando cd podemos accesar a un directorio (lejano) usando diagonales </> por ejemplo:

\begin{equation}
  cd usuario/Documentos/Repositorio
  \label{cdAcceso}
\end{equation}

Podemos notar que en \ref{cdAcceso} se utiliza una dirección directa, con esa única línea de comando ya nos podemos localizar en el directorio deseado.\\

\section{Repositorios en Git}

\subsection{Introducción}

 Git es un sistema de control de versiones distribuido, gratuito y de código abierto, diseñado para gestionar todo tipo de proyectos, desde pequeños hasta muy grandes, con rapidez y eficacia.
 
 Git es fácil de aprender y tiene una huella diminuta con un rendimiento increíblemente rápido. Supera a las herramientas SCM como Subversion, CVS, Perforce y ClearCase con características como sucursales locales baratas, áreas de preparación convenientes y múltiples flujos de trabajo.\cite{GitIntro} Traducción realizada con el traductor DeepL.\cite{DeepL}\\ 
 
 \subsection{Trabajando con Git}
 
 
 \subsubsection{Directorio de Git}
 
 El directorio.git es donde Git almacena los metadatos y la base de datos de objetos para el repositorio.
 
  \subsubsection{Directorio de trabajo}
 
 Una copia de una versión del proyecto git, tomada de la base de datos comprimida en el directorio.git 
 
  \subsubsection{Área de puesta a disposición/Índice}
  	
 Archivo que almacena información sobre qué se va a confirmar en el repositorio git
  
  \subsubsection{Configurar herramientas}
 
 git config --global user.name "[nombre]"\\
 Establece el nombre que desea adjuntar a sus transacciones de confirmación.\\[0.5cm]
 
 git config --global user.email "[dirección de correo electrónico]"\\
 Establece el correo electrónico que desea adjuntar a sus transacciones de confirmación.\\[0.5cm]
 
 \subsubsection{Realizar modificaciones}
 
git status\\
 Enumera todos los archivos nuevos o modificados que se van a confirmar.\\[0.5cm]
 
 git diff\\
 Muestra las diferencias de archivos que aún no se han preparado.\\[0.5cm]
 
 git add [archivo]
 Instantáneas del archivo en preparación para el versionado (la bandera $--all$ añade todos los cambios nuevos).\\[0.5cm]
 
 git diff $--etapas$
 Muestra las diferencias de archivo entre la puesta a disposición y la última versión del archivo.\\[0.5cm]
 
 git reset[archivo]\\
 Desescribe el archivo, pero conserva su contenido.\\[0.5cm]
 
 git commit -m "[mensaje descriptivo]"\\
 Registra las instantáneas del archivo permanentemente en el repositorio.\\[0.5cm]
 
 
 \subsubsection{Crear Repositorios}
 
 git init[nombre del proyecto]\\
 Crea un nuevo repositorio local con el nombre especificado.\\[0.5cm]

 git clone[url]\\
 Descarga un proyecto y toda su historia de versiones.\\[0.5cm]
 
 
 \subsubsection{Sincronizar cambios}
 
 git push\\
Sube todos los compromisos de las sucursales locales al repositorio.\\[0.5cm]

 git pull\\
 Descarga el historial de marcadores e incorpora cambios.\\[0.5cm]
 
 
  \cite{GitIntro}Traducción realizada con el traductor DeepL \cite{DeepL}.\\

\section{GitLab}

Para el mantenimiento de repositorios se estará usando la plataforma GitLab, ya que brinda muchas herremientas adicionales, como levantar issues, comunicación entre usuarios, etc.\\

\section{Conclusión}

En la creación y mantenimiento de repositorios de los comandos anteriormente mencionado se hará uso más frecuente de los siguientes y uno adicional para la creación de repositorios \\

\begin{enumerate}
    \item git pull
	\item git status
	\item git add [archivo] (con bandera --all)
	\item git commit -m "[mensaje descriptivo]"
	\item git push
\end{enumerate}





\bibliographystyle{plain}
\bibliography{Referencias.bib}


\end{document}
