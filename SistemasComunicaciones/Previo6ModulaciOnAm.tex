\documentclass{article}
\usepackage[utf8]{inputenc}
\usepackage[spanish.mexico]{babel}
\usepackage[american voltages, american currents,siunitx]{circuitikz}

%Plotting

\usepackage{pgfplots}
\pgfplotsset{width=10cm,compat=1.9} 
 \usepgfplotslibrary{external}
\tikzexternalize 

%Diagrama de bloques

\usetikzlibrary{shapes.geometric, arrows}

\tikzstyle{startstop} = [rectangle, rounded corners, minimum width=3cm, minimum height=1cm,text centered, draw=black, fill=red!30]
\tikzstyle{io} = [trapezium, trapezium left angle=70, trapezium right angle=110, minimum width=3cm, minimum height=1cm, text centered, draw=black, fill=blue!30]
\tikzstyle{process} = [rectangle, minimum width=3cm, minimum height=1cm, text centered, draw=black, fill=orange!30]
\tikzstyle{decision} = [diamond, minimum width=3cm, minimum height=1cm, text centered, draw=black, fill=green!30]


\tikzstyle{arrow} = [thick,->,>=stealth]
%###MIO###
\tikzstyle{point} = [circle,(0.5cm),minimum with=1cm,draw=black,fill=black!50]




\title{Previo 6: Amplitud Modulada}
\author{Pablo Vivar Colina\\
Grupo 13
}
%\date{Septiembre 2017}

\usepackage{natbib}
\usepackage{graphicx}

\begin{document}

\maketitle

\section{Índice de modulación}

El índice de modulación es una relación sin unidad y se utiliza sólo para describir la profundidad de la modulación lograda para una señal modulada en amplitud y frecuencia dada.\citep{IndiceModulacion}\\

Es el proceso de cambiar la amplitud de una portadora de acuerdo con las variaciones de amplitud de la señal moduladora. La envolvente de la portadora es la información transmitida, y podremos verla en los semiciclos positivos y negativos de la portadora. El porcentaje en que la señal moduladora cambia la portadora senoidal es conocido como el
índice de modulación.\citep{IndiceModulacion}\\

En modulación AM, la señal mensaje y la portadora se combinan en una señal compuesta. La expresión matemática de esta señal es:\citep{IndiceModulacion}\\

\begin{equation}
    s(t)=A_c[1+m(t)]cos(2 \pi f_c(t))
    \label{portadora}
\end{equation}

El índice o porcentaje de modulación de una señal AM es una medida que indica cuanto varía el voltaje de la señal portadora debido a la señal moduladora o mensaje. El índice de modulación toma valores entre 0 y 1. Reescribiendo la ecuación (\ref{portadora}), tenemos:\citep{IndiceModulacion}\\


    \begin{equation}
    s(t)=A_c[1+a*m(t)]cos(2 \pi f_c(t))
    \label{portadoraPorA}
\end{equation}

Siendo a: el índice de modulación, $m_n(t)$ es la señal mensaje normalizada, con la condición:\citep{IndiceModulacion}\\

\begin{equation}
    |m_n(t)|<1
\end{equation}

Si $a > 1$, entonces la señal AM esta sobremodulada, el resultado es una señal que presenta distorsiones. Otra expresión para definir el índice de modulación en circuitos reales, es la siguiente:\citep{IndiceModulacion}\\

\begin{equation}
    a=\frac{k*A_m}{A_c}
\end{equation}

Donde Am es el valor máximo en amplitud de $m(t)$, y $k$ es un parámetro deganancia que se puede ajustar para modificar el índice de modulación a discreción del operador.\citep{IndiceModulacion}\\

 
%####AQUI VAMOS###





\bibliographystyle{plain}
\bibliography{Referencias.bib}


\end{document}