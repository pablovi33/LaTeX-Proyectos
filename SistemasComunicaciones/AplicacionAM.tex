\documentclass{article}
\usepackage[utf8]{inputenc}
\usepackage[spanish.mexico]{babel}
\usepackage[american voltages, american currents,siunitx]{circuitikz}

%Plotting

\usepackage{pgfplots}
\pgfplotsset{width=10cm,compat=1.9} 
 \usepgfplotslibrary{external}
\tikzexternalize 

%Diagrama de bloques

\usetikzlibrary{shapes.geometric, arrows}

\tikzstyle{startstop} = [rectangle, rounded corners, minimum width=3cm, minimum height=1cm,text centered, draw=black, fill=red!30]
\tikzstyle{io} = [trapezium, trapezium left angle=70, trapezium right angle=110, minimum width=3cm, minimum height=1cm, text centered, draw=black, fill=blue!30]
\tikzstyle{process} = [rectangle, minimum width=3cm, minimum height=1cm, text centered, draw=black, fill=orange!30]
\tikzstyle{decision} = [diamond, minimum width=3cm, minimum height=1cm, text centered, draw=black, fill=green!30]


\tikzstyle{arrow} = [thick,->,>=stealth]
%###MIO###
\tikzstyle{point} = [circle,(0.5cm),minimum with=1cm,draw=black,fill=black!50]




\title{Aplicaciones de los tipos de modulación AM }
\author{Pablo Vivar Colina\\
Grupo 13
}
%\date{Septiembre 2017}

\usepackage{natbib}
\usepackage{graphicx}

\begin{document}

\maketitle

\section{Modulación}

Es el proceso de situar una señal de baja frecuencia sobre una señal de más alta frecuencia.\\

\begin{itemize}
    \item  Señal de baja frecuencia   :     moduladora
    \item Señal de alta frecuencia    :     portadora
\end{itemize}

Una portadora puede modificarse por la moduladora en :\\

\begin{itemize}
    \item amplitud
    \item frecuencia
      \item  fase
\end{itemize}

Razones de porqué se emplea la modulación:\\

\begin{itemize}
    \item La Tx directa  resultaría en problemas de interferencia, si las señales de diferentes usuarios  tendrían la misma frecuencia.
   \item  La mayoría de las señales  de información son de banda base, de frecuencias bajas,su Tx directa necesitarían grandes antenas. 
   \item  Con la modulación se simplifica la estructura del transmisor con altas frecuencias 
\end{itemize}

¿Porqué emplear portadora?\\
Si la señal es audio  es de: 3000 Hz\\

\begin{itemize}
    \item $\lambda=  c/f$     muy grande,
                                                                       antena muy grande

    \item En cambio si se emplea una portadora   a 
                                                                       $f = 1000 [Mhz]$
    \item  Una antena con $\lambda/4$ es posible (teléfonos celulares)\citep{Tomasi}\\
\end{itemize}


Los sistemas son generalmente bidireccionales . Se emplea un modulador a la transmisión y un demodulador la recepción. Para posibilitar un diálogo  se emplea u MODEM en cada extremo .\\

El método utilizado para producir una señal AM es aplicar la portadora y la moduladora en un dispositivo no lineal.\\

En este caso a la salida del modulador  se tendrá  :\\

\begin{itemize}
    \item un nivel de C.D.

 \item componentes de las frecuencias originales

 \item componentes de intermodulación.

 \item componentes de  distorsión armónica.
\end{itemize}

\section{Índice de modulación}

El índice de modulación es una relación sin unidad y se utiliza sólo para describir la profundidad de la modulación lograda para una señal modulada en amplitud y frecuencia dada.\citep{IndiceModulacion}\\

Es el proceso de cambiar la amplitud de una portadora de acuerdo con las variaciones de amplitud de la señal moduladora. La envolvente de la portadora es la información transmitida, y podremos verla en los semiciclos positivos y negativos de la portadora. El porcentaje en que la señal moduladora cambia la portadora senoidal es conocido como el
índice de modulación.\citep{IndiceModulacion}\\

En modulación AM, la señal mensaje y la portadora se combinan en una señal compuesta. La expresión matemática de esta señal es:\citep{IndiceModulacion}\\

\begin{equation}
    s(t)=A_c[1+m(t)]cos(2 \pi f_c(t))
    \label{portadora}
\end{equation}

El índice o porcentaje de modulación de una señal AM es una medida que indica cuanto varía el voltaje de la señal portadora debido a la señal moduladora o mensaje. El índice de modulación toma valores entre 0 y 1. Reescribiendo la ecuación (\ref{portadora}), tenemos:\citep{IndiceModulacion}\\


    \begin{equation}
    s(t)=A_c[1+a*m(t)]cos(2 \pi f_c(t))
    \label{portadoraPorA}
\end{equation}

Siendo a: el índice de modulación, $m_n(t)$ es la señal mensaje normalizada, con la condición:\citep{IndiceModulacion}\\

\begin{equation}
    |m_n(t)|<1
\end{equation}

Si $a > 1$, entonces la señal AM esta sobremodulada, el resultado es una señal que presenta distorsiones. Otra expresión para definir el índice de modulación en circuitos reales, es la siguiente:\citep{IndiceModulacion}\\

\begin{equation}
    a=\frac{k*A_m}{A_c}
\end{equation}

Donde Am es el valor máximo en amplitud de $m(t)$, y $k$ es un parámetro deganancia que se puede ajustar para modificar el índice de modulación a discreción del operador.\citep{IndiceModulacion}\\

 \section{Modulación de amplitud AM}
 
 A amplitud de la portadora cambia en proporción directa a la amplitud instantánea de la moduladora, como se puede apreciar en la figura \ref{fig:sistemaAM}



\begin{figure}[h!]
    \centering
    \begin{tikzpicture}[node distance=4cm]

%startstop-->ROJO
\node (Portadora) [process] {Portadora};



\node (Moduladora) [process, below of=Portadora] {Moduladora};



\node (Sistema) [process, right of=Portadora, ] {Sistema};
%yshift=-0.5cm

\node (PortadoraModulada) [process,right of=Sistema] {Portadora Modulada};

%\node (PortadoraModulada) [process,yshift=-2cm,above of=Canal2] {Portadora Modulada};

%Líneas
\draw [arrow] (Portadora) -- (Sistema);

 
\draw [arrow] (Moduladora) -| (Sistema);

\draw [arrow] (Sistema) --(PortadoraModulada);


\end{tikzpicture}
    \caption{El modulador en AM tiene dos entradas y una salida}
    \label{fig:sistemaAM}
\end{figure}

\subsection{Modulación de amplitud convencional doble banda lateral con portadora completa}

\begin{itemize}
    \item Portadora                      $v_c(t)=E_c sen(2\pi f_c t)$

     \item Moduladora                  $v_m(t)=E_m sen(2\pi f_m t)$

     \item Portadora modulada      $V_{AM}(t)$
\end{itemize}


\begin{figure}[h!]
    \centering

\begin{tikzpicture}
\begin{axis}[
    axis lines = left,
    xlabel = {t[s]},
    ylabel = {V[V]},
]

%ENVOLVENTE SUPERIOR
\addplot
[thick=0.1cm,
    domain=0:0.05, 
    samples=300, 
    color=blue,
]
{sin(2*3.14159*x*1000)+2};



\addplot
[thick=0.1cm,
    domain=0:0.05, 
    samples=300, 
    color=red,
]
{sin(2*3.14159*x*10000)};





%Se añade nota :D
\addlegendentry{Moduladora}
\addlegendentry{Portadora}

\end{axis}
\end{tikzpicture}
\caption{Modulación AM}
\label{modulacionAM}
\end{figure}


%SEGUNDA FIGURA DE AM


\begin{figure}[h!]
    \centering

\begin{tikzpicture}
\begin{axis}[
    axis lines = left,
    xlabel = {t[s]},
    ylabel = {V[V]},
]

%ENVOLVENTE SUPERIOR
\addplot
[thick=0.1cm,
    domain=0:0.05, 
    samples=300, 
    color=blue,
]
{sin(2*3.14159*x*1000)+1.5};

%ENVOLVENTE INFERIOR


\addplot
[thick=0.1cm,
    domain=0:0.05, 
    samples=300, 
    color=blue,
]
{-sin(2*3.14159*x*1000)-1.5};


%\addplot
%[thick=0.1cm,
 %   domain=0:0.05, 
  %  samples=300, 
   % color=red,
%]
%{sin(2*3.14159*x*10000)};

%CONVOLUCION
\addplot
[thick=0.1cm,
    domain=0:0.05, 
    samples=300, 
    color=green,
]
{(sin(2*3.14159*x*1000)+1.5)*sin(2*3.14159*x*10000)};

%NO MODULACION
\addplot
[thick=0.1cm,
    domain=-0.025:0, 
    samples=300, 
    color=blue,
]
{+1.5};

\addplot
[thick=0.1cm,
    domain=-0.025:0, 
    samples=300, 
    color=blue,
]
{-1.5};

\addplot
[thick=0.1cm,
    domain=-0.025:0, 
    samples=300, 
    color=green,
]
{1.5*sin(2*3.14159*x*10000)};






%Se añade nota :D
\addlegendentry{Envolvente superior}
\addlegendentry{Envolvente inferior}
\addlegendentry{Convolución}

\end{axis}
\end{tikzpicture}
\caption{No hay modulación hasta llegar al tiempo cero}
\label{modulacionAM}
\end{figure}


\begin{itemize}
    \item sin señal  :  sólo existe la portadora
    \item la amplitud de la portadora  varía con la moduladora.
   \item la envolvente de la amplitud de salida tiene la misma forma de la moduladora. 
   \item sin embargo, la forma de onda AM no contiene ninguna componente de la señal moduladora
   \item la envolvente no es una componente de la forma de onda AM y no se vería en un osciloscopio.
\end{itemize}

\subsection{Ancho de Banda en Amplitud modulada de doble landa lateral con portadora completa}

Un modulador es un dispositivo no lineal. A la salida de AM  se consideran útiles  (si la señal moduladora contiene un frecuencia única fm)  :

\begin{itemize}
    \item frecuencia portadora
    \item frecuencias:(fc+fm)  (fc-fm)    

\end{itemize}

En general, la mezcla de dos frecuencias en un dispositivo no lineal genera un número mayor de componentes además de las deseadas.

El efecto de la modulación es trasladar  la señal moduladora   a una frecuencia más alta.
 
%####AQUI VAMOS###





\bibliographystyle{plain}
\bibliography{Referencias.bib}


\end{document}