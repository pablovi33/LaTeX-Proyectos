
\documentclass{article}
\usepackage[utf8]{inputenc}
\usepackage[spanish.mexico]{babel}
%\title{Dispositivos}
\author{Pablo Vivar Colina}
%\date{Septiembre 2017}

\usepackage{natbib}
\usepackage{graphicx}

%Circuitos
\usepackage{tikz}

\usepackage[american voltages, american currents,siunitx]{circuitikz}

%Diagrama de bloques

\usetikzlibrary{shapes.geometric, arrows}

\tikzstyle{startstop} = [rectangle, rounded corners, minimum width=3cm, minimum height=1cm,text centered, draw=black, fill=red!30]
\tikzstyle{io} = [trapezium, trapezium left angle=70, trapezium right angle=110, minimum width=3cm, minimum height=1cm, text centered, draw=black, fill=blue!30]
\tikzstyle{process} = [rectangle, minimum width=3cm, minimum height=1cm, text centered, draw=black, fill=orange!30]
\tikzstyle{decision} = [diamond, minimum width=3cm, minimum height=1cm, text centered, draw=black, fill=green!30]


\tikzstyle{arrow} = [thick,->,>=stealth]
%###MIO###
\tikzstyle{point} = [circle,(0.5cm),minimum with=1cm,draw=black,fill=black!50]



%Plotting

\usepackage{pgfplots}
\pgfplotsset{width=10cm,compat=1.9} 
 \usepgfplotslibrary{external}
\tikzexternalize 

%#####Fracciones DIAGONALES :B #####

\usepackage{amsmath}
\usepackage{mathtools}

%running fraction with slash - requires math mode.
\newcommand*\rfrac[2]{{}^{#1}\!/_{#2}}


%\maketitle

%\usepackage[top=2cm,bottom=2cm,left=1cm,right=1cm]{geometry}


\begin{titlepage}
     \begin{center}
	\includegraphics[width=0.09\textwidth]{UNAM}\Large Universidad Nacional Autónoma de México
        	\includegraphics[width=0.09\textwidth]{FI}\\[1cm]
        \Large Facultad de Ingeniería\\[1cm]
       % \Large División de Ciencias Básicas\\[1cm]
         \Large Laboratorio de Dispositivos y Circuitos Electrónicos (6654)\\[1cm]
         %la clave antes era:4314
         \footnotesize Profesor: Zapata Rosales Arturo Ing.\\[1cm]
        \footnotesize Semestre 2018-1\\[1cm]
        %\Large Práctica No. 1\\[1cm]
    
        %\Large Práctica No. 2\\[1cm]
        
        %\Large Práctica No. 3\\[1cm]
       
        %\Large Práctica No. 4\\[1cm]
         
               
         %\Large Práctica No. 5\\[1cm]
         
         
         %\Large Práctica No. 6\\[1cm]
         
         %\Large Práctica No. 7\\[1cm]
         
             %\Large Práctica No. 8\\[1cm]
       

        \Large Práctica No. 9\\[1cm]
        
           %####AQUI VAMOS#### ya ahora sii
           
        %\Large Práctica No. 11\\[1cm]
        %\Large Práctica No. 12\\[1cm]
        %\Large Práctica No. 13\\[1cm]
        
        %\Large Amplificador Operacional como Integrador\\[1cm]
        %\Large{Filtros}\\[1cm]
         %\Large{Medición de  corrientes en un circuito}\\[1cm]
         %practica 4
         %Large{Amplificador operacional como seguidor de voltaje en entrada inversora}\\[1cm]
         %practica5
         %\Large{Amplificador operacional como integrador}
         
         %Practica 7
%Comportamiento de un diodo Zener

\Large Diodo Zener
        
         %Texto a la derecha
          \begin{flushright}
\footnotesize  Grupo 13\\[0.5cm]
\footnotesize Brigada: 7\\[0.5cm]

\footnotesize Vivar Colina Pablo\\[0.5cm]
 \end{flushright}
    %Texto a la izquierda
          \begin{flushleft}
        \footnotesize Ciudad Universitaria Abril de 2018.\\
          \end{flushleft}
         
          
        %\vfill
        %\today
   \end{center}
\end{titlepage}
 %agregar portada

\begin{document}



\tableofcontents  % Write out the Table of Contents

\listoffigures  % Write out the List of Figures


\section{Marco Teórico}

\subsection{Distorsión}

\subsubsection{Distorsión}

Es cualquier cambio en una señal que altera su forma de onda básica (en el dominio del  tiempo)  o  bien,  altera  la  relación  entre  sus  componentes  espectrales  (domino  de  la  frecuencia).  La distorsión puede ser del tipo lineal o del tipo no lineal.\citep{Distorsion}


\subsubsection{Distorsión lineal}

Es la alteración de la forma de onda de la señal 
transmitida y se debe a la 
respuesta en frecuencia no plana del medio de trans
misión, que trabaja como filtro y tiende a atenuar 
o a resaltar algunas frecuencias del mensaje. El efecto en telefonía es que a veces se no reconocemos la voz del 
que nos habla porque se modifica su timbre.\citep{Distorsion} 

\subsubsection{Distorsión alineal}

Es un tipo de distorsión no lineal y ocurre cuando un sistema, debido a su ganancia  no línea,  genera  nuevas  componentes  espectrales  en  frecuencias  múltiplo  de  las  frecuencias  ya presente (armónicas) o bien, genera nuevas componentes espectrales en frecuencias suma y diferencia de las  frecuencias  ya  presentes  en  la  señal  (intermodulación).\citep{Distorsion}

%\section{Circuito}


 %\begin{figure}[h!]
  %  \centering
   % \begin{circuitikz}
    
    %\draw
    
    %Generador de funciones
    
     % (-6,0.5) to   (-6,-0.5) node[ground]{}
    %(-6,0.5)--(-5,0.5)
     %  (-5,0.5)to[vco,l=$G$](-4,0.5)
    
    
    %resistencia1
    %(-1,0.5)to[R,l=$R_1$](-4,0.5)
    
    %REsistencia2
    %(-1,0.5)--(-1,1.5)
    %(-1,1.5)to[R,l=$R_2$](2,1.5)
    %(2,1.5)--(2,0)
    
    %salida
%    (1,0)--(3,0)
    
    %tierra a no invesora
 %   (-1.25,-0.5)  to  (-1.25,-1) node[ground]{}
    
    %Fuente
    
  %  (6,3) to[V,l=$15V$](6,-1)
   % (6,3)--(5,3)
    %(5,3)to[R,l=$R_3$](5,1)
    %(5,1)to[R,l=$R_4$](5,-1)
    %(6,-1)--(5,-1)
    
    %(5,1)--(4,1)
    %(4,1) to (4,1) node[ground]{}
    
    %Positivo
    %(5,3)--(4,3)
    %(3.5,3) node{+}
    
    %Negativo
    %(5,-1)--(4,-1)
    %(3.5,-1) node{-}
    

    %;
    %\draw (0,0) node[op amp] (opamp1) {741};
 
  
    %\end{circuitikz}
    %\caption{741 como seguidor de voltaje polarizado con 1 fuente (divisor de voltaje)}
    %\label{fig:OpAmpBuffer1fuente}
%\end{figure}



\section{Desarrollo}

\subsection{Diagrama de bloques}



\begin{figure}[h!]
    \centering
    \begin{tikzpicture}[node distance=4cm]

%startstop-->ROJO
\node (start) [process] {Generador de señales};



\node (ReP) [process, right of=start] {Red en Prueba};



\node (Volt) [process, right of=ReP, ] {Voltímetro};
%yshift=-0.5cm

\node (Canal1) [process,yshift=-2cm,above of=Canal2] {Canal 1 Osciloscopio};

\node (Canal2) [process, above,yshift=-2cm, above of=Volt] {Canal 2 Osciloscopio};

\node (AeE) [process,yshift=2cm, below of=Volt] {Analizador de Espectros};


%Línea inicio a red en prueba
\draw [arrow] (start) -- (ReP);


%Línea de red en prueba aVoltimetro 
\draw [arrow] (ReP) |- (Volt);

\draw [arrow] (start) --(Canal1);

\draw [arrow] (Volt) -- (Canal2);
\draw [arrow] (ReP) |- (AeE);

\end{tikzpicture}
    \caption{Sistema de pruebas}
    \label{fig:sistPruebas}
\end{figure}

En ésta sesión del laboratorio se utilizó el equipo para generar una señal sinusoidal en el generador de funciones, y a través de distintos circuitos se fueron revisando los voltajes de entrada y de salida así como también se midió el ángulo de defasamiento.\\

En el cuadro \ref{lineaTelefonica} se puede apreciar que en el experimento en el cual se utiliza el circuito de línea telefónica en el cuadro de Red en prueba expuesto en, en la figura \ref{fig:sistPruebas}. De igual forma en éste mismo diagrama se colocó el circuito corrector y los datos se registraron en el cuadro \ref{circuitoCorrector}\\

%####TABLAS PRACTICA 4 #######

\subsection{Circuito de línea telefónica}

Para el primer experimento se utilizó el circuito en la figura \ref{fig:circuitoLineaTelefonica}, es importante mencionar que la numeración de los componentes se respeta para los circuitos siguientes, y en éste ejercicio los resultados se pueden apreciar en el cuadro \ref{lineaTelefonica}.\\


\begin{figure}[h!]
    \centering
    \begin{circuitikz}
    
    \draw
    
    (0,0)to[R,l=$R_1$](3,0)
    
    (3,0)to[C,l=$C_1$](3,-3)
    
    (3,0)to[R,l=$R_2$](6,0) 
    
    (7,0)to[R,l=$R_3$](7,-3)
    
    (6,0)--(9,0)
    
    (0,-3)--(9,-3)
    

    %Leyenda teléfono
    (9,-1.5) node{Teléfono}
    
    %Leyenda línea telefónica
    (0,-1.5) node{Línea Telefónica}

    ;
   
    \end{circuitikz}
    \caption{Circuito línea telefónica}
    \label{fig:circuitoLineaTelefonica}
    
    \end{figure}


\begin{table}[h!]
\centering

\begin{tabular}{|c|c|c|c|}
\hline
Frecuencia [Hz] & $V_{entrada}$ & $V_{salida}$ & Desfasamiento [o] \\ \hline
100                 & 4.31          & 1.73         & 5.78                 \\ \hline
250                 & 4.31          & 1.64         & 7.2                  \\ \hline
500                 & 4.33          & 1.65         & 12.5                 \\ \hline
750                 & 4.32          & 1.66         & 12                   \\ \hline
1000                & 4.34          & 1.63         & 24.4                 \\ \hline
%copiado del anteror
1250                & 4.33          & 1.6          & 24.4                      \\ \hline
1500                & 4.32          & 1.56         & 24.0                 \\ \hline
1750                & 4.318         & 1.417        & 18.1                 \\ \hline
2000                & 4.318         & 1.47         & 28.8                 \\ \hline
2250                & 4.32          & 1.417        & 28.8                     \\ \hline
2500                & 4.309         & 1.36         & 37.6                 \\ \hline
2750                & 4.3063        & 1.301        & 41.1                 \\ \hline
3000                & 4.3032        & 1.2502       &  41.1                    \\ \hline
3250                & 4.3003        & 1.2          & 39.1                 \\ \hline
3500                & 4.2968        & 1.1516       & -52.8                \\ \hline
3750                & 4.2931        & 1.1066       & 53.3                 \\ \hline
4000                & 4.2893        & 1.064        & 56.7                 \\ \hline
4250                & 4.2852        & 1.03         & 45.9                 \\ \hline
4500                & 4.2774        & 1.08         & 49.2                 \\ \hline
4750                & 4.277         & 0.955        & 37.5                 \\ \hline
5000                & 4.27          & 0.92         & 84.2                 \\ \hline
5250                & 4.26          & 0.92         & 66.1                 \\ \hline
5500                & 4.27          & 0.856        & 55.1                 \\ \hline
5750                & 4.27          & 0.831        & 45.9                 \\ \hline
6000                & 4.26          & 0.804        & 73.2                 \\ \hline
6250                & 4.251         & 0.774        & 41.4                 \\ \hline
6500                & 4.254         & 0.753        & 44.6                 \\ \hline
6750                & 4.25          & 0.734        & 56.9                 \\ \hline
7000                & 4.249         & 0.713        & 56.6                 \\ \hline
\end{tabular}

\caption{Línea Telefónica}
\label{lineaTelefonica}

\end{table}

\subsection{Cicuito Corrector}

En ésta sección de la práctica se utilizó el circuito mostrado en la figura \ref{fig:circuitoCorrector} el cuál se conectó de manera independiente del circuito mostrado en la figura \ref{fig:circuitoLineaTelefonica}, y los resultados están mostrados en el cuadro \ref{circuitoCorrector}.\\

\begin{figure}[h!]
    \centering
    \begin{circuitikz}
    
    \draw
    
    (-2,0)--(0,0)
    
    (0,0)to[R,l=$R_4$](3,0)
   
    (3,0)to[C,l=$C_2$](6,0) 
    
    (0,-1.5)--(0,0)
     (0,-1.5)to[R,l=$R_5$](3,-1.5)
     (3,-1.5)--(6,-1.5)
     (6,-1.5)--(6,0)
     
     (6,0)--(7,0)
    
    (7,0)to[L,l=$L_1$](7,-3)
    
    (6,0)--(9,0)
    
    (-2,-3)--(9,-3)
    

    %Leyenda teléfono
    (9,-1.5) node{Teléfono}
    
    %Leyenda línea telefónica
    (-2,-1.5) node{Línea Telefónica}

    ;
   
    \end{circuitikz}
    \caption{Circuito Corrector}
    \label{fig:circuitoCorrector}
    
    \end{figure}

\begin{table}[h!]
\centering

\begin{tabular}{|c|c|c|c|}
\hline
Frecuencia {[}Hz{]} & $V_{entrada}$ & $V_{salida}$ & Desfasamiento  [o]\\ \hline
100                 & 4.70          & 0.757        & 2.88          \\ \hline
250                 & 4.70          & 0.762        & 5.40          \\ \hline
500                 & 4.70          & 0.778        & 10.8          \\ \hline
750                 & 4.69          & 0.804        & 19.8          \\ \hline
1000                & 4.7           & 0.838        & 25.8          \\ \hline
1250                & 4.69          & 0.881        & 28.8          \\ \hline
1500                & 4.69          & 0.932        & 31            \\ \hline
1750                & 4.68          & 0.992        & 36            \\ \hline
2000                & 4.68          & 1.06         & 46.7          \\ \hline
2250                & 4.67          & 1.13         & 43.7          \\ \hline
2500                & 4.67          & 1.21         & 48            \\ \hline
2750                & 4.66          & 1.30         & 48.3          \\ \hline
3000                & 4.66          & 1.39         & 51            \\ \hline
3250                & 4.64          & 1.48         & 52.8          \\ \hline
3500                & 4.64          & 1.58         & 39.1          \\ \hline
3750                & 4.63          & 1.68         & 46.2          \\ \hline
4000                & 4.63          & 1.79         & 55.8          \\ \hline
4250                & 4.62          & 1.89         & 52.6          \\ \hline
4500                & 4.6           & 2.01         & 52.8          \\ \hline
4750                & 4.6           & 2.11         & 54.9          \\ \hline
5000                & 4.59          & 2.23         & 50.9          \\ \hline
5250                & 4.58          & 2.33         & 53.7          \\ \hline
5500                & 4.57          & 2.44         & 51.4          \\ \hline
5750                & 4.58          & 2.54         & 51.1          \\ \hline
6000                & 4.56          & 2.65         & 50.9          \\ \hline
6250                & 4.56          & 2.74         & 48.4          \\ \hline
6500                & 4.55          & 2.84         & 48.7          \\ \hline
6750                & 4.55          & 2.93         & 46            \\ \hline
7000                & 4.54          & 3.02         & 46.2          \\ \hline
\end{tabular}

\caption{Resultados Circuito Corrector}
\label{circuitoCorrector}

\end{table}


\subsection{Cicuito en Cascada}

Finalmente se utilizaron los circuitos mostrados en las figuras \ref{fig:circuitoLineaTelefonica} y \ref{fig:circuitoCorrector} en combinación en circuito tipo cascada, ésta representación puede ser apreciada en la figura \ref{fig:circuitoEnCascada}, y en consecuencia los resultados correspondientes se pueden apreciar en el cuadro \ref{circuitoCascada}.\\

Los resultados obtenidos en voltaje de salida y de entrada se pueden apreciar gráficamente en la figura \ref{fig:ComportamientoCircuitoCascada}. Lo importante de los 3 experimentos realizados es comparar su comportamiento entre sí, y lo que se busca al estar intercambiando los circuitos es buscar que el desfasamiento entre las señales sea lo menor posible, es por eso que se tomó en cuenta en cada experimento anotar el ángulo de desfasamiento entre las señales, y para verlo de forma gráfica se incluyó la figura \ref{fig:ComparacionDesfasamientos} en dónde se puede apreciar los 3 experimentos en conjunto.

\begin{figure}[h!]
    \centering
    \begin{circuitikz}
    
    \draw
    

    
    (-6,0)to[R,l=$R_1$](-3,0)
    
     (-3,0)to[C,l=$C_1$](-3,-3)
    
    (-3,0)to[R,l=$R_2$](0,0)

    (0,0)to[R,l=$R_4$](3,0)
   
    (3,0)to[C,l=$C_2$](6,0) 
    
    (0,-1.5)--(0,0)
     (0,-1.5)to[R,l=$R_5$](3,-1.5)
     (3,-1.5)--(6,-1.5)
     (6,-1.5)--(6,0)
     
    
    
    (7,0)to[L,l=$L_1$](7,-3)
    
     (8,0)to[R,l=$R_3$](8,-3)
    
    (6,0)--(9,0)
    
    (-6,-3)--(9,-3)
    

    %Leyenda teléfono
    (8,1) node {Teléfono}
    
    %Leyenda línea telefónica
    (-6,-1.5) node {Línea Telefónica}

    ;
   
    \end{circuitikz}
    \caption{Circuito en Cascada}
    \label{fig:circuitoEnCascada}
    
    \end{figure}



\begin{table}[h!]
\centering

\begin{tabular}{|c|c|c|c|}
\hline
Frecuencia [Hz] & $V_{entrada}$ & $V_{salida}$ & Desfasamiento [o] \\ \hline
100        & 4.18          & 0.491        & 1.44         \\ \hline
250        & 4.15          & 0.524        & -6.3         \\ \hline
500        & 4.16          & 0.618        & -9           \\ \hline
750        & 4.17          & 0.688        & -19.3        \\ \hline
1000       & 4.2           & 0.744        & -24.4        \\ \hline
1250       & 4.21          & 0.795        & -32.4        \\ \hline
1500       & 4.23          & 0.848        & -35.4        \\ \hline
1750       & 4.24          & 0.907        & -38.8        \\ \hline
2000       & 4.26          & 0.973        & -38.8        \\ \hline
2250       & 4.27          & 1.090        & -42.1        \\ \hline
2500       & 4.27          & 1.11         & -45          \\ \hline
2750       & 4.29          & 1.19         & -44.5        \\ \hline
3000       & 4.29          & 1.28         & -49.5        \\ \hline
3250       & 4.31          & 1.36         & -47.9        \\ \hline
3500       & 4.31          & 1.45         & -52          \\ \hline
%Fragmentado aqui
3750 & 4.62   & 0.370 & -52.0 \\ \hline
4000 & 4.6084 & 0.366 & -5.84 \\ \hline
4250 & 4.59   & 0.367 & -7.66 \\ \hline
4500 & 4.59   & 0.368 & -12.9 \\ \hline
4750 & 4.58   & 0.370 & -9.08 \\ \hline
5000 & 4.58   & 0.371 & -14.3 \\ \hline
5250 & 4.58   & 0.372 & -17.6 \\ \hline
5500 & 4.57   & 0.372 & -11.8 \\ \hline
5750 & 4.57   & 0.373 & -17.2 \\ \hline
6000 & 4.56   & 0.373 & -15.8 \\ \hline
6250 & 4.56   & 0.373 & -18   \\ \hline
6500 & 4.56   & 0.373 & -21.4 \\ \hline
6750 & 4.56   & 0.372 & -22   \\ \hline
7000 & 4.55   & 0.371 & -23.5 \\ \hline
\end{tabular}

\caption{Circuito en Cascada}
\label{circuitoCascada}

\end{table}


\begin{figure}[h!]
    \centering
 
\begin{tikzpicture}
\begin{axis}[
   axis lines = left,
    title={Amplitud},
    xlabel={Frecuencia[Hz] },
    ylabel={$V_{entrada}$[V]},
    xmin=50, xmax=7050,
    ymin=0, ymax=5,
    %xtick={0,2,4,6},
    %ytick={0,0.06},
    legend pos=north west,
    ymajorgrids=true,
    %grid style=dashed,
]



 
\addplot[
    color=green,
    mark=square,
    ]
    coordinates {
    (100, 4.18)
(250,4.15)
(500,4.16)
(750,4.17)
(1000,4.2)
(1250,4.21)
(1500,4.23)
(1750,4.24)
(2000,4.26)
(2250,4.27)
(2500,4.27)
(2750,4.29)
(3000,4.29)
(3250,4.31)
(3500,4.31)
%Fragmentado aqui
(3750,4.62)
(4000,4.6084)
(4250,4.59)
(4500,4.59)
(4750,4.58)
(5000,4.58)
(5250,4.58)
(5500,4.57)
(5750,4.57)
(6000,4.56)
(6250,4.56)
(6500,4.56)
(6750,4.56)
(7000,4.55)
    };
    
    
    \addplot[
    color=blue,
    mark=square,
    ]
       coordinates {
    (100, 0.491)
(250,0.524)
(500,0.618)
(750,0.688)
(1000,0.744)
(1250,0.795)
(1500,0.848)
(1750,0.907)
(2000,0.973)
(2250,1.090)
(2500,1.11)
(2750,1.19)
(3000,1.28)
(3250,1.36)
(3500,1.45)
%Fragmentado aqui
(3750,0.370)
(4000,0.366)
(4250,0.367)
(4500,0.368)
(4750,0.370)
(5000,0.371)
(5250,0.372)
(5500,0.372)
(5750,0.373)
(6000,0.373)
(6250,0.373)
(6500,0.373)
(6750,0.372)
(7000,0.371)
    };
   
    \legend{$V_{entrada}$}
   
 
\end{axis}
\end{tikzpicture}
    
    \caption{Comportamiento Circuito Cascada}
    \label{fig:ComportamientoCircuitoCascada}
\end{figure}

\section{Comparación desplazamientos}

En los experimentos anteriores se ha medido el defasamiento entres esñales (entrada y salida) por lo que se ha elaborado un gráfico en la figura \ref{fig:ComparacionDesfasamientos} en donde se incluye un eje que indica las frecuencias utilizadas y otro eje en dónde se incluyen una escala de ángulos, ésto se hizo con el propósito de hacer un mejor análisis de los cuadros \ref{lineaTelefonica},\ref{circuitoCorrector}, y \ref{circuitoCascada}. 

\begin{figure}[h!]
    \centering
 
\begin{tikzpicture}
\begin{axis}[
   axis lines = left,
    title={Adelanto o Atraso},
    xlabel={Frecuencia[Hz] },
    ylabel={Desfasamiento [o]},
    xmin=50, xmax=7050,
    ymin=-60, ymax=90,
    %xtick={0,2,4,6},
    %ytick={0,0.06},
    legend pos=north west,
    ymajorgrids=true,
    %grid style=dashed,
]


    \addplot[
    color=green,
    mark=square,
    ]
      coordinates {
(100,5.78)
(250,7.2)
(500,12.5)
(750,12)
(1000,24.4)
%copiado del anterior
(1250,24.4)
(1500,24.0)
(1750,18.1)
(2000,28.8)
%copiado del anterior
(2250,28.8)
(2500,37.6)
(2750,41.1)
%copiado del anterior
(3000,41.1)
(3250,39.1)
(3500,-52.8)
(3750,53.3)
(4000,56.7)
(4250,45.9)
(4500,49.2)
(4750,37.5)
(5000,84.2)
(5250,66.1)
(5500,55.1)
(5750,45.9)
(6000,73.2)
(6250,41.4)
(6500,44.6)
(6750,56.9)
(7000,56.6)
    };
   
   \addplot[
    color=orange,
    mark=square,
    ]
    coordinates {
(100,2.88)
(250,5.40)
(500,10.8)
(750,19.8)
(1000,25.8)
(1250,28.8)
(1500,31)
(1750,36)
(2000,46.7)
(2250,43.7)
(2500,48)
(2750,48.3)
(3000,51)
(3250,52.8)
(3500,39.1)
(3750,46.2)
(4000,55.8)
(4250,52.6)
(4500,52.8)
(4750,54.9)
(5000,50.9)
(5250,53.7)
(5500,51.4)
(5750,51.1)
(6000,50.9)
(6250,48.4)
(6500,48.7)
(6750,46)
(7000,46.2)
    };
    
    \addplot[
    color=blue,
    mark=square,
    ]
    coordinates {
(100,1.44)
(250,-6.3)
(500,-9)
(750,-19.3)
(1000,-24.4)
(1250,-32.4)
(1500,-35.4)
(1750,-38.8)
(2000,-38.8)
(2250,-42.1)
(2500,-45)
(2750,-44.5)
(3000,-49.5)
(3250,-47.9)
(3500,-52)
(3750,-5.84)
(4000,-7.66)
(4250,-12.9)
(4500,-9.08)
(4750,-14.3)
(5000,-17.6)
(5250,-11.8)
(5500,-17.2)
(5750,-15.8)
(6000,-18)
(6250,-21.4)
(6500,-21.4)
(6750,-22)
(7000,-23.5)
    };
    
    
   
   
    \legend{Tel,Corr,Casc}
   
 
\end{axis}
\end{tikzpicture}
    
    \caption{Comparación desfasamientos entre circuitos}
    \label{fig:ComparacionDesfasamientos}
\end{figure}

\section{Conclusiones Martínez Hernández Fernando}

En esta practica conocimos la distorsión que sufren las señales que se propagan en los cables telefónicos y una de las formas de corregirla, aprendimos que en la distorsión lineal solo cambia la fase y la amplitud de la entrada con respecto a la salida, mediante filtros observamos como se corregía la distorsión.

\section{Conclusiones Pablo Vivar Colina}

Es importante mencionar que la práctica se realizó en 2 sesiones es por eso que hay considerables variaciones el las figuras \ref{fig:ComportamientoCircuitoCascada} y \ref{fig:ComparacionDesfasamientos}, se puede apreciar que a partir de la frecuencia 3500 [Hz] se nota este cambio, sin embargo el comportamiento entre los valores medidos se restringe entre ciertos parámetros. Y es por eso que el análisis se realizará antes y después de ésta frecuencia.\\

En los tres experimentos se pudo comparar los diferentes comportamientos de las señales, como se aprecia en la figura \ref{fig:ComparacionDesfasamientos}. Para los 2 primeros circuitos podemos notar que el ángulo de desfasamiento alcanza valores hasta de los 90 grados de desfasamiento, además de que el circuito de la línea telefónica varía bastante sus resultados, con el circuito corrector podemos ver que aunque los valores son más estables siguen siendo altos; alcanzando valores entre los 40 y 60 grados. Podemos ver finalmente que los valores del circuito corrector son más estables, aunque si sólo tomamos en cuenta los valores después de los 3500 [Hz], podríamos decir que el ángulo de desfasamiento está entre los valores de entre 0 y 20 grados.\\

Según lo anteriormente mencionado podemos decir que la aplicación del circuito corrector en cascada con la línea telefónica estabiliza los valores de entrada y de salida además de que minimiza el ángulo de desfasamiento entre señales.\\[10cm]
%La práctica de distorsión lineal no pudo ser completada correctamente debido a motivos de tiempo, es por eso que el cuadro \ref{circuitoCorrector} no tiene la misma cantidad de datos que el cuadro \ref{lineaTelefonica}, aún así es posible apreciar los cambios en la señal de entrada y de salida sobre los diferentes circuitos, en el primer circuito el voltaje de entrada iba disminuyendo conforme la frecuencia subía, y además el ángulo de desfasamiento iba aumentando,l es decir la señal sufría un desfasamiento que se iba pronunciando conforme la frecuencia aumentara.

\section{Comentarios}

%No se pueden realizar conclusiones sobre el experimento del circuito corrector, sin embargo se pueden analizar los datos del cuadro \ref{circuitoCorrector}, en él podemos apreciar que a diferencia con el experimento de la línea telefónica el ángulo de desfasamiento tiene un cierto adelanto y parece ser un poco más consistente que en el primer experimento, en cuánto al voltaje de salida, podemos notar que es pequeño en el inicio y va aumentando, cosa que no sucede en el primer experimento (iba disminuyendo).
%.\\[100cm]

En la práctica se pudieron notar las diferencias entre las distintas señales y porque es importante diseñar un circuito que ayude a la transmisión de señales con los parámetros indicados, en éste caso el parámetro de interés fue el ángulo de desfasamiento para que la señal tuviera el menor atraso posible.

También es importante realizar los ejercicios con pleno cuidado y de ser posible en la misma sesión para que los resultados sean los más consistentes y con valores estables entre ellos.

\bibliographystyle{plain}
\bibliography{Referencias.bib}

\end{document}
