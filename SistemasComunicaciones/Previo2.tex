ve\documentclass{article}
\usepackage[utf8]{inputenc}
\usepackage[spanish.mexico]{babel}
\usepackage[american voltages, american currents,siunitx]{circuitikz}


\title{Previo 2: Análisis espectral de una señal senoidal}
\author{Pablo Vivar Colina\\
Grupo 13
}
%\date{Septiembre 2017}

\usepackage{natbib}
\usepackage{graphicx}

\begin{document}

\maketitle

\section{Señal Senoidal y cosenoidal}

Se solicita obtener el espectro teórico de una señal senoidal y cosenoidal de $7 [V_{RMS}]$ y $1 [kHz]$

\begin{figure}[h!]
    \centering
    
   
\begin{tikzpicture}
\begin{axis}[
    axis lines = left,
    xlabel = {t[ms]},
    ylabel = {V[V]},
]

\addplot
[thick=0.1cm,
    domain=0:2, 
    samples=100, 
    color=red,
]
{4.9497*sin(deg((2*3.1459*x))};
\addlegendentry{$4.9497 V_{pp}$}
%Here the blue parabloa is defined
%\addplot [
 %   domain=-10:10, 
  %  samples=100, 
    %color=blue,
   % ]
    %{x^2 + 2*x + 1};
%\addlegendentry{$x^2 + 2x + 1$}
 
\end{axis}
\end{tikzpicture}

\caption{Señal senoidal 4.9497 Vpp}
    \label{fig:seno4.9497.1khzprim}
\end{figure}



\bibliographystyle{plain}
\bibliography{Previo1.bib}


\end{document}