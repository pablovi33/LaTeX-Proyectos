\documentclass{article}
\usepackage[utf8]{inputenc}
\usepackage[spanish.mexico]{babel}

\title{Progreso en el desarrollo computacional}
\author{Pablo Vivar Colina\\
Grupo 5\\
Tarea 2
}
%\date{Septiembre 2017}

\usepackage{natbib}
\usepackage{graphicx}

\begin{document}

\maketitle

\section{Ábaco}

Las evidencias del uso del ábaco surgen en comentarios de los antiguos escritores griegos. Por ejemplo, Demóstenes (384-322 a. C.) escribió acerca de la necesidad del uso de piedras para realizar cálculos difíciles de efectuar mentalmente. \citep{Abaco}\\

%Otro ejemplo son los métodos de cálculo encontrados en los comentarios de Heródoto (484-425 a. C.), que hablando de los egipcios decía: "Los egipcios mueven su mano de derecha a izquierda en los cálculos, mientras los griegos lo hacen de izquierda a derecha".\citep{Abaco}\\

\section{Edad Media}





\section{William Gilbert: materiales eléctricos y materiales aneléctricos (1600)}

El científico inglés William Gilbert (1544-1603) publicó su libro De Magnete, en donde utiliza la palabra latina electricus, derivada del griego elektron, que significa ámbar, para describir los fenómenos descubiertos por los griegos. \citep{HDE}\\

%Gilbert estableció las diferencias entre ambos fenómenos a raíz de que la reina Isabel I de Inglaterra le ordenara estudiar los imanes para mejorar la exactitud de las brújulas usadas en la navegación, consiguiendo con este trabajo la base principal para la definición de los fundamentos de la electrostática y magnetismo. A través de sus experiencias clasificó los materiales en eléctricos (conductores) y aneléctricos (aislantes) e ideó el primer electroscopio. Descubrió la imantación por influencia, y observó que la imantación del hierro se pierde cuando se calienta al rojo. Estudió la inclinación de una aguja magnética concluyendo que la Tierra se comporta como un gran imán. El Gilbert es la unidad de medida de la fuerza magnetomotriz.\citep{HDE}\\

\section{Pascalina}

La pascalina fue la primera calculadora que funcionaba a base de ruedas y engranajes, inventada en 1642 por el filósofo y matemático francés Blaise Pascal (1623-1662). El primer nombre que le dio a su invención fue «máquina de aritmética». Luego la llamó «rueda pascalina», y finalmente «pascalina». Este invento es el antepasado remoto del actual ordenador.\citep{Pascalina}

\section{Benjamin Franklin: el pararrayos (1752)}

1752 El polifacético estadounidense Benjamin Franklin (1706-1790) investigó los fenómenos eléctricos naturales. Es particularmente famoso su experimento en el que, haciendo volar una cometa durante una tormenta, demostró que los rayos eran descargas eléctricas de tipo electrostático. Como consecuencia de estas experimentaciones inventó el pararrayos.\citep{HDE}\\ 

%También formuló una teoría según la cual la electricidad era un fluido único existente en toda materia y calificó a las substancias en eléctricamente positivas y eléctricamente negativas, de acuerdo con el exceso o defecto de ese fluido.\\

\section{Alessandro Volta: la pila de Volta (1800)}

El físico italiano Alessandro Volta (1745-1827) inventa la pila, precursora de la batería eléctrica. Con un apilamiento de discos de zinc y cobre, separados por discos de cartón humedecidos con un electrólito, y unidos en sus extremos por un circuito exterior, Volta logró, por primera vez, producir corriente eléctrica continua a voluntad.\citep{HDE}\\

%Dedicó la mayor parte de su vida al estudio de los fenómenos eléctricos, inventó el electrómetro y el eudiómetro y escribió numerosos tratados científicos. Por su trabajo en el campo de la electricidad, Napoleón le nombró conde en 1801. La unidad de tensión eléctrica o fuerza electromotriz, el Volt (símbolo V), castellanizado como Voltio, recibió ese nombre en su honor.\citep{HDE}\\

\section{Hans Christian Ørsted: el electromagnetismo (1819)}

El físico y químico danés Hans Christian Ørsted (1777-1851) fue un gran estudioso del electromagnetismo. En 1813 predijo la existencia de los fenómenos electromagnéticos y en 1819 logró demostrar su teoría empíricamente al descubrir, junto con Ampère, que una aguja imantada se desvía al ser colocada en dirección perpendicular a un conductor por el que circula una corriente eléctrica. Este descubrimiento fue crucial en el desarrollo de la electricidad, ya que puso en evidencia la relación existente entre la electricidad y el magnetismo.\citep{HDE}\\

%En homenaje a sus contribuciones se denominó Oersted (símbolo Oe) a la unidad de intensidad de campo magnético en el sistema Gauss. Se cree que también fue el primero en aislar el aluminio, por electrólisis, en 1825. En 1844 publicó su Manual de Física Mecánica.\citep{HDE}\\

\section{André-Marie Ampère: el solenoide (1822)}

El físico y matemático francés André-Marie Ampère (1775-1836) está considerado como uno de los descubridores del electromagnetismo. Es conocido por sus importantes aportaciones al estudio de la corriente eléctrica y el magnetismo que constituyeron, junto con los trabajos del danés Hans Chistian Oesterd, el desarrollo del electromagnetismo. Sus teorías e interpretaciones sobre la relación entre electricidad y magnetismo se publicaron en 1822, en su Colección de observaciones sobre electrodinámica y en 1826, en su Teoría de los fenómenos electrodinámicos. Ampère descubrió las leyes que determinan el desvío de una aguja magnética por una corriente eléctrica, lo que hizo posible el funcionamiento de los actuales aparatos de medida. Descubrió las acciones mutuas entre corrientes eléctricas, al demostrar que dos conductores paralelos por los que circula una corriente en el mismo sentido, se atraen, mientras que si los sentidos de la corriente son opuestos, se repelen. \citep{HDE}\\

%La unidad de intensidad de corriente eléctrica, el Ampère (símbolo A), castellanizada como Amperio, recibe este nombre en su honor.\citep{HDE}\\

\section{Georg Simon Ohm: la ley de Ohm (1827)}

Georg Simon Ohm (1789-1854) fue un físico y matemático alemán que estudió la relación entre el voltaje V aplicado a una resistencia R y la intensidad de corriente I que circula por ella. En 1827 formuló la ley que lleva su nombre (la ley de Ohm), cuya expresión matemática es V = I · R. También se interesó por la acústica, la polarización de las pilas y las interferencias luminosas. \citep{HDE}\\

%En su honor se ha bautizado a la unidad de resistencia eléctrica con el nombre de Ohm (símbolo Ω), castellanizado a Ohmio.\citep{HDE}\\


\section{Johann Carl Friedrich Gauss: Teorema de Gauss de la electrostática}

1832-1835. El matemático, astrónomo y físico alemán Johann Carl Friedrich Gauss (1777-1855), hizo importantes contribuciones en campos como la teoría de números, el análisis matemático, la geometría diferencial, la geodesia, la electricidad, el magnetismo y la óptica. Considerado uno de los matemáticos de mayor y más duradera influencia, se contó entre los primeros en extender el concepto de divisibilidad a conjuntos diferentes de los numéricos. En 1831 se asoció al físico Wilhelm Weber durante seis fructíferos años durante los cuales investigaron importantes problemas como las Leyes de Kirchhoff y del magnetismo, construyendo un primitivo telégrafo eléctrico. Su contribución más importante a la electricidad es la denominada Ley de Gauss, que relaciona la carga eléctrica q contenida en un volumen V con el flujo del campo eléctrico 

%E → {\displaystyle {\vec {E}}} {\vec {E}} sobre la cerrada superficie S que encierra el volumen V.\citep{HDE}\\

\section{Gustav Robert Kirchhoff: leyes de Kirchhoff (1845)}

Las principales contribuciones a la ciencia del físico alemán Gustav Robert Kirchhoff (1824-1887), estuvieron en el campo de los circuitos eléctricos, la teoría de placas, la óptica, la espectroscopia y la emisión de radiación de cuerpo negro. Kirchhoff propuso el nombre de radiación de cuerpo negro en 1862. Es responsable de dos conjuntos de leyes fundamentales en la teoría clásica de circuitos eléctricos y en la emisión térmica. Aunque ambas se denominan Leyes de Kirchhoff, probablemente esta denominación es más común en el caso de las Leyes de Kirchhoff de la ingeniería eléctrica. Estas leyes permiten calcular la distribución de corrientes y tensiones en las redes eléctricas con derivaciones y establecen lo siguiente: 1ª) La suma algebraica de las intensidades que concurren en un punto es igual a cero. 2ª) La suma algebraica de los productos parciales de intensidad por resistencia, en una malla, es igual a la suma algebraica de las fuerzas electromotrices en ella existentes, cuando la intensidad de corriente es constante. \citep{HDE}\\

%Junto con los químicos alemanes Robert Wilhelm Bunsen y Joseph von Fraunhofer, fue de los primeros en desarrollar las bases teóricas y experimentales de la espectroscopia, desarrollando el espectroscopio moderno para el análisis químico. En 1860 Kirchhoff y Bunsen descubrieron el cesio y el rubidio mediante la espectroscopia. Kirchhoff también estudio el espectro solar y realizó importantes investigaciones sobre la transferencia de calor.\citep{HDE}\\

\section{James Clerk Maxwell: las cuatro ecuaciones de Maxwell (1875)}

El físico y matemático escocés James Clerk Maxwell (1831-1879) es conocido principalmente por haber desarrollado un conjunto de ecuaciones que expresan las leyes fundamentales de la electricidad y el magnetismo así como por la estadística de Maxwell-Boltzmann en la teoría cinética de gases. También se dedicó a la investigación de la visión de los colores y los principios de la termodinámica. Formuló teóricamente que los anillos de Saturno estaban formados por materia disgregada. Maxwell amplió las investigaciones que Michael Faraday había realizado sobre los campos electromagnéticos, formulando la relación matemática entre los campos eléctricos y magnéticos por medio de cuatro ecuaciones diferenciales (llamadas hoy "las ecuaciones de Maxwell") que relacionan el campo eléctrico y el magnético para una distribución espacial de cargas y corrientes. También demostró que la naturaleza de los fenómenos luminosos y electromagnéticos era la misma y que ambos se propagan a la velocidad de la luz.\citep{HDE}\\ 

%Su obra más importante es el Treatise on Electricity and Magnetism (Tratado de electricidad y magnetismo, 1873), en el que publicó sus famosas ecuaciones. También escribió: Matter and motion (Materia y movimiento, 1876) y Theory of Heat (Teoría del calor, 1877). La teoría de Maxwell obtuvo su comprobación definitiva cuando Heinrich Rudolf Hertz obtuvo en 1888 las ondas electromagnéticas de radio. \citep{HDE}\\


%Sus investigaciones posibilitaron la invención del telégrafo sin cables y la radio. La unidad de flujo magnético en el sistema cegesimal, el maxwell, recibe este nombre en su honor.

\section{Alexander Graham Bell: el teléfono (1876)}

El escocés-estadounidense Alexander Graham Bell, científico, inventor y logopeda (1847-1922), se disputó con otros investigadores la invención del teléfono y consiguió la patente oficial en los Estados Unidos en 1876 Previamente habían sido desarrollados dispositivos similares por otros investigadores, entre quienes destacó Antonio Meucci (1871), que entabló pleitos fallidos con Bell hasta su muerte, y a quien suele reconocerse actualmente la prelación en el invento.\citep{HDE}\\

%Bell contribuyó de un modo decisivo al desarrollo de las telecomunicaciones a través de su empresa comercial (Bell Telephone Company, 1877, posteriormente AT&T). También fundó en la ciudad de Washington el Laboratorio Volta, donde, junto con sus socios, inventó un aparato que transmitía sonidos mediante rayos de luz (el fotófono, 1880); y desarrolló el primer cilindro de cera para grabar (1886), lo que sentó las bases del gramófono. Participó en la fundación de la National Geographic Society y de la revista Science.\citep{HDE}\\

\section{Thomas Alva Edison: desarrollo de la lámpara incandescente (1879), Menlo Park y comercialización}

En el ámbito científico descubrió el efecto Edison, patentado en 1883, que consistía en el paso de electricidad desde un filamento a una placa metálica dentro de un globo de lámpara incandescente. Aunque ni él ni los científicos de su época le dieron importancia, este efecto sería uno de los fundamentos de la válvula de la radio y de la electrónica. En 1880 se asoció con el empresario J. P. Morgan para fundar la General Electric.\citep{HDE}\\

\section{John Hopkinson: el sistema trifásico (1882)}

El ingeniero y físico inglés John Hopkinson (1849-1898) contribuyó al desarrollo de la electricidad con el descubrimiento del sistema trifásico para la generación y distribución de la corriente eléctrica, sistema que patentó en 1882. Un sistema de corrientes trifásicas es el conjunto de tres corrientes alternas monofásicas de igual frecuencia y amplitud (y por consiguiente, valor eficaz) que presentan un desfase entre ellas de 120$^0$ (un tercio de ciclo).
%\citep{HDE}\\

%Cada una de las corrientes monofásicas que forman el sistema se designa con el nombre de fase. También trabajó en muchas áreas del electromagnetismo y la electrostática. De sus investigaciones estableció que "el flujo de inducción magnética es directamente proporcional a la fuerza magnetomotriz e inversamente proporcional a la reluctancia", expresión muy parecida a la establecida en la Ley de Ohm para la electricidad, y que se conoce con el nombre de Ley de Hopkinson También se dedicó al estudio de los sistemas de iluminación, mejorando su eficiencia, así como al estudio de los condensadores. Profundizó en los problemas de la teoría electromagnética, propuestos por James Clerk Maxwell. En 1883 dio a conocer el principio de los motores síncronos.\citep{HDE}\\

\section{Heinrich Rudolf Hertz: demostración de las ecuaciones de Maxwell y la teoría electromagnética de la luz (1887)}

El físico alemán Heinrich Rudolf Hertz (1857-1894) demostró la existencia de las ondas electromagnéticas predichas por las ecuaciones de Maxwell. Fue el primer investigador que creó dispositivos que emitían ondas radioeléctricas y también dispositivos que permitía detectarlas. Hizo numerosos experimentos sobre su modo y velocidad de propagación (hoy conocida como velocidad de la luz), en los que se fundamentan la radio y la telegrafía sin hilos, que él mismo descubrió. En 1887 descubrió el efecto fotoeléctrico. La unidad de medida de la frecuencia fue llamada Hertz (símbolo Hz) en su honor, castellanizada como Hercio.\citep{HDE}\\

\section{George Westinghouse: el suministro de corriente alterna (1886)}

Westinghouse compró al científico croata Nikola Tesla su patente para la producción y transporte de corriente alterna, que impulsó y desarrolló. Posteriormente perfeccionó el transformador, desarrolló un alternador y adaptó para su utilización práctica el motor de corriente alterna inventado por Tesla. \citep{HDE}\\

%En 1886 fundó la compañía eléctrica Westinghouse Electric & Manufacturing Company, que contó en los primeros años con la decisiva colaboración de Tesla, con quien logró desarrollar la tecnología necesaria para desarrollar un sistema de suministro de corriente alterna. Westinghouse también desarrolló un sistema para transportar gas natural, y a lo largo de su vida obtuvo más de 400 patentes, muchas de ellas de maquinaria de corriente alterna.

\section{Nikola Tesla: desarrollo de máquinas eléctricas, la bobina de Tesla (1884-1891) y el radiotransmisor (1893)}

El ingeniero e inventor de origen croata Nikola Tesla (1856-1943) emigró en 1884 a los Estados Unidos. Es reconocido como uno de los investigadores más destacados en el campo de la energía eléctrica. El Gobierno de Estados Unidos lo consideró una amenaza por sus opiniones pacifistas y sufrió el maltrato de otros investigadores mejor reconocidos como Marconi o Edison.\citep{HDE}\\

Desarrolló la teoría de campos rotantes, base de los generadores y motores polifásicos de corriente alterna. En 1887 logra construir el motor de inducción de corriente alterna y trabaja en los laboratorios Westinghouse, donde concibe el sistema polifásico para transmitir la electricidad a largas distancias. En 1893 consigue transmitir energía electromagnética sin cables, construyendo el primer radiotransmisor (adelantándose a Guglielmo Marconi). Ese mismo año en Chicago hizo una exhibición pública de la corriente alterna, demostrando su superioridad sobre la corriente continua de Edison. Los derechos de estos inventos le fueron comprados por George Westinghouse, que mostró el sistema de generación y transmisión por primera vez en la World's Columbian Exposition de Chicago de 1893. \citep{HDE}\\

%Dos años más tarde los generadores de corriente alterna de Tesla se instalaron en la central experimental de energía eléctrica de las cataratas del Niágara. Entre los muchos inventos de Tesla se encuentran los circuitos resonantes de condensador más inductancia, los generadores de alta frecuencia y la llamada bobina de Tesla, utilizada en el campo de las comunicaciones por radio.\citep{HDE}\\

%La unidad de inducción magnética del sistema MKS recibe el nombre de Tesla en su honor.\citep{HDE}\\

%\section{Guglielmo Marconi: la telegrafía inalámbrica (1899)}

%El ingeniero y físico italiano Guglielmo Marconi (1874-1937), es conocido, principalmente, como el inventor del primer sistema práctico de señales telegráficas sin hilos, que dio origen a la radio actual. En 1899 logró establecer comunicación telegráfica sin hilos a través del canal de la Mancha entre Inglaterra y Francia, y en 1903 a través del océano Atlántico entre Cornualles, y Saint John's en Terranova, Canadá. En 1903 estableció en los Estados Unidos la estación WCC, en cuya inauguración cruzaron mensajes de salutación el presidente Theodore Roosevelt y el rey Eduardo VIII de Inglaterra. \citep{HDE}\\

%En 1904 llegó a un acuerdo con el Servicio de Correos británico para la transmisión comercial de mensajes por radio. Las marinas italiana y británica pronto adoptaron su sistema y hacia 1907 había alcanzado tal perfeccionamiento que se estableció un servicio trasatlántico de telegrafía sin hilos para uso público. Para la telegrafía fue un gran impulso el poder usar el código Morse sin necesidad de cables conductores.

%Aunque se le atribuyó la invención de la radio, ésta fue posible gracias a una de las patentes de Nikola Tesla, tal y como fue reconocido por la alta corte de los Estados Unidos, seis meses después de la muerte de Tesla, hacia el año 1943. También inventó la antena Marconi. En 1909 Marconi recibió, junto con el físico alemán Karl Ferdinand Braun, el Premio Nobel de Física por su trabajo.\citep{HDE}\\

\section{Albert Einstein: El efecto fotoeléctrico (1905)}

Al alemán nacionalizado norteamericano Albert Einstein (1879 – 1955) se le considera el científico más conocido e importante del siglo XX. El resultado de sus investigaciones sobre la electricidad llegó en 1905 (fecha trascendental que se conmemoró en el Año mundial de la física 2005), cuando escribió cuatro artículos fundamentales sobre la física de pequeña y gran escala. En ellos explicaba el movimiento browniano, el efecto fotoeléctrico y desarrollaba la relatividad especial y la equivalencia entre masa y energía.\citep{HDE}\\

El efecto fotoeléctrico consiste en la emisión de electrones por un material cuando se le ilumina con radiación electromagnética (luz visible o ultravioleta, en general). Ya había sido descubierto y descrito por Heinrich Hertz en 1887, pero la explicación teórica no llegó hasta que Albert Einstein le aplicó una extensión del trabajo sobre los cuantos de Max Planck. \citep{HDE}\\

%En el artículo dedicado a explicar el efecto fotoeléctrico, Einstein exponía un punto de vista heurístico sobre la producción y transformación de luz, donde proponía la idea de quanto de radiación (ahora llamados fotones) y mostraba cómo se podía utilizar este concepto para explicar el efecto fotoeléctrico. Una explicación completa del efecto fotoeléctrico solamente pudo ser elaborada cuando la teoría cuántica estuvo más avanzada. A Albert Einstein se le concedió el Premio Nobel de Física en 1921.\citep{HDE}\\

%El efecto fotoeléctrico es la base de la producción de energía eléctrica por radiación solar y de su aprovechamiento energético. Se aplica también para la fabricación de células utilizadas en los detectores de llama de las calderas de las grandes usinas termoeléctricas. También se utiliza en diodos fotosensibles tales como los que se utilizan en las células fotovoltaicas y en electroscopios o electrómetros. En la actualidad (2008) los materiales fotosensibles más utilizados son, aparte de los derivados del cobre (ahora en menor uso), el silicio, que produce corrientes eléctricas mayores.\citep{HDE}\\

\section{Años 1950}





\bibliographystyle{plain}
\bibliography{LineaTiempo.bib}


\end{document}