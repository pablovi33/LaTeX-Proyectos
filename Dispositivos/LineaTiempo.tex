\documentclass{article}
\usepackage[utf8]{inputenc}
\usepackage[spanish.mexico]{babel}

\title{Progreso en el desarrollo computacional}
\author{Pablo Vivar Colina\\
Grupo 5\\
Tarea 2
}
%\date{Septiembre 2017}

\usepackage{natbib}
\usepackage{graphicx}

\begin{document}

\maketitle

\section{Ábaco}

Las evidencias del uso del ábaco surgen en comentarios de los antiguos escritores griegos. Por ejemplo, Demóstenes (384-322 a. C.) escribió acerca de la necesidad del uso de piedras para realizar cálculos difíciles de efectuar mentalmente. \citep{Abaco}\\

%Otro ejemplo son los métodos de cálculo encontrados en los comentarios de Heródoto (484-425 a. C.), que hablando de los egipcios decía: "Los egipcios mueven su mano de derecha a izquierda en los cálculos, mientras los griegos lo hacen de izquierda a derecha".\citep{Abaco}\\

\section{Edad Media}

\enumarate{

El matemático y astrónomo persa Musa al-Juarismi (780-850), inventó el algoritmo, es decir, la resolución metódica de problemas de álgebra y cálculo numérico mediante una lista bien definida, ordenada y finita de operaciones.\citep{HDC}\\

}





\section{William Gilbert: materiales eléctricos y materiales aneléctricos (1600)}

El científico inglés William Gilbert (1544-1603) publicó su libro De Magnete, en donde utiliza la palabra latina electricus, derivada del griego elektron, que significa ámbar, para describir los fenómenos descubiertos por los griegos. \citep{HDE}\\

%Gilbert estableció las diferencias entre ambos fenómenos a raíz de que la reina Isabel I de Inglaterra le ordenara estudiar los imanes para mejorar la exactitud de las brújulas usadas en la navegación, consiguiendo con este trabajo la base principal para la definición de los fundamentos de la electrostática y magnetismo. A través de sus experiencias clasificó los materiales en eléctricos (conductores) y aneléctricos (aislantes) e ideó el primer electroscopio. Descubrió la imantación por influencia, y observó que la imantación del hierro se pierde cuando se calienta al rojo. Estudió la inclinación de una aguja magnética concluyendo que la Tierra se comporta como un gran imán. El Gilbert es la unidad de medida de la fuerza magnetomotriz.\citep{HDE}\\

\section{Pascalina}

La pascalina fue la primera calculadora que funcionaba a base de ruedas y engranajes, inventada en 1642 por el filósofo y matemático francés Blaise Pascal (1623-1662). El primer nombre que le dio a su invención fue «máquina de aritmética». Luego la llamó «rueda pascalina», y finalmente «pascalina». Este invento es el antepasado remoto del actual ordenador.\citep{Pascalina}

\section{Benjamin Franklin: el pararrayos (1752)}

1752 El polifacético estadounidense Benjamin Franklin (1706-1790) investigó los fenómenos eléctricos naturales. Es particularmente famoso su experimento en el que, haciendo volar una cometa durante una tormenta, demostró que los rayos eran descargas eléctricas de tipo electrostático. Como consecuencia de estas experimentaciones inventó el pararrayos.\citep{HDE}\\ 

%También formuló una teoría según la cual la electricidad era un fluido único existente en toda materia y calificó a las substancias en eléctricamente positivas y eléctricamente negativas, de acuerdo con el exceso o defecto de ese fluido.\\

\section{Alessandro Volta: la pila de Volta (1800)}

El físico italiano Alessandro Volta (1745-1827) inventa la pila, precursora de la batería eléctrica. Con un apilamiento de discos de zinc y cobre, separados por discos de cartón humedecidos con un electrólito, y unidos en sus extremos por un circuito exterior, Volta logró, por primera vez, producir corriente eléctrica continua a voluntad.\citep{HDE}\\

%Dedicó la mayor parte de su vida al estudio de los fenómenos eléctricos, inventó el electrómetro y el eudiómetro y escribió numerosos tratados científicos. Por su trabajo en el campo de la electricidad, Napoleón le nombró conde en 1801. La unidad de tensión eléctrica o fuerza electromotriz, el Volt (símbolo V), castellanizado como Voltio, recibió ese nombre en su honor.\citep{HDE}\\

\section{Hans Christian Ørsted: el electromagnetismo (1819)}

El físico y químico danés Hans Christian Ørsted (1777-1851) fue un gran estudioso del electromagnetismo. En 1813 predijo la existencia de los fenómenos electromagnéticos y en 1819 logró demostrar su teoría empíricamente al descubrir, junto con Ampère, que una aguja imantada se desvía al ser colocada en dirección perpendicular a un conductor por el que circula una corriente eléctrica. Este descubrimiento fue crucial en el desarrollo de la electricidad, ya que puso en evidencia la relación existente entre la electricidad y el magnetismo.\citep{HDE}\\

%En homenaje a sus contribuciones se denominó Oersted (símbolo Oe) a la unidad de intensidad de campo magnético en el sistema Gauss. Se cree que también fue el primero en aislar el aluminio, por electrólisis, en 1825. En 1844 publicó su Manual de Física Mecánica.\citep{HDE}\\

\section{André-Marie Ampère: el solenoide (1822)}

El físico y matemático francés André-Marie Ampère (1775-1836) está considerado como uno de los descubridores del electromagnetismo. Es conocido por sus importantes aportaciones al estudio de la corriente eléctrica y el magnetismo que constituyeron, junto con los trabajos del danés Hans Chistian Oesterd, el desarrollo del electromagnetismo. Sus teorías e interpretaciones sobre la relación entre electricidad y magnetismo se publicaron en 1822, en su Colección de observaciones sobre electrodinámica y en 1826, en su Teoría de los fenómenos electrodinámicos. Ampère descubrió las leyes que determinan el desvío de una aguja magnética por una corriente eléctrica, lo que hizo posible el funcionamiento de los actuales aparatos de medida. Descubrió las acciones mutuas entre corrientes eléctricas, al demostrar que dos conductores paralelos por los que circula una corriente en el mismo sentido, se atraen, mientras que si los sentidos de la corriente son opuestos, se repelen. \citep{HDE}\\

%La unidad de intensidad de corriente eléctrica, el Ampère (símbolo A), castellanizada como Amperio, recibe este nombre en su honor.\citep{HDE}\\

\section{Georg Simon Ohm: la ley de Ohm (1827)}

Georg Simon Ohm (1789-1854) fue un físico y matemático alemán que estudió la relación entre el voltaje V aplicado a una resistencia R y la intensidad de corriente I que circula por ella. En 1827 formuló la ley que lleva su nombre (la ley de Ohm), cuya expresión matemática es V = I · R. También se interesó por la acústica, la polarización de las pilas y las interferencias luminosas. \citep{HDE}\\

%En su honor se ha bautizado a la unidad de resistencia eléctrica con el nombre de Ohm (símbolo Ω), castellanizado a Ohmio.\citep{HDE}\\


\section{Johann Carl Friedrich Gauss: Teorema de Gauss de la electrostática}

1832-1835. El matemático, astrónomo y físico alemán Johann Carl Friedrich Gauss (1777-1855), hizo importantes contribuciones en campos como la teoría de números, el análisis matemático, la geometría diferencial, la geodesia, la electricidad, el magnetismo y la óptica. Considerado uno de los matemáticos de mayor y más duradera influencia, se contó entre los primeros en extender el concepto de divisibilidad a conjuntos diferentes de los numéricos. En 1831 se asoció al físico Wilhelm Weber durante seis fructíferos años durante los cuales investigaron importantes problemas como las Leyes de Kirchhoff y del magnetismo, construyendo un primitivo telégrafo eléctrico. Su contribución más importante a la electricidad es la denominada Ley de Gauss, que relaciona la carga eléctrica q contenida en un volumen V con el flujo del campo eléctrico E → {\displaystyle {\vec {E}}} {\vec {E}} sobre la cerrada superficie S que encierra el volumen V.\citep{HDE}\\

\section{Gustav Robert Kirchhoff: leyes de Kirchhoff (1845)}

Las principales contribuciones a la ciencia del físico alemán Gustav Robert Kirchhoff (1824-1887), estuvieron en el campo de los circuitos eléctricos, la teoría de placas, la óptica, la espectroscopia y la emisión de radiación de cuerpo negro. Kirchhoff propuso el nombre de radiación de cuerpo negro en 1862. Es responsable de dos conjuntos de leyes fundamentales en la teoría clásica de circuitos eléctricos y en la emisión térmica. Aunque ambas se denominan Leyes de Kirchhoff, probablemente esta denominación es más común en el caso de las Leyes de Kirchhoff de la ingeniería eléctrica. Estas leyes permiten calcular la distribución de corrientes y tensiones en las redes eléctricas con derivaciones y establecen lo siguiente: 1ª) La suma algebraica de las intensidades que concurren en un punto es igual a cero. 2ª) La suma algebraica de los productos parciales de intensidad por resistencia, en una malla, es igual a la suma algebraica de las fuerzas electromotrices en ella existentes, cuando la intensidad de corriente es constante. \citep{HDE}\\

%Junto con los químicos alemanes Robert Wilhelm Bunsen y Joseph von Fraunhofer, fue de los primeros en desarrollar las bases teóricas y experimentales de la espectroscopia, desarrollando el espectroscopio moderno para el análisis químico. En 1860 Kirchhoff y Bunsen descubrieron el cesio y el rubidio mediante la espectroscopia. Kirchhoff también estudio el espectro solar y realizó importantes investigaciones sobre la transferencia de calor.\citep{HDE}\\

\section{James Clerk Maxwell: las cuatro ecuaciones de Maxwell (1875)}

El físico y matemático escocés James Clerk Maxwell (1831-1879) es conocido principalmente por haber desarrollado un conjunto de ecuaciones que expresan las leyes fundamentales de la electricidad y el magnetismo así como por la estadística de Maxwell-Boltzmann en la teoría cinética de gases. También se dedicó a la investigación de la visión de los colores y los principios de la termodinámica. Formuló teóricamente que los anillos de Saturno estaban formados por materia disgregada. Maxwell amplió las investigaciones que Michael Faraday había realizado sobre los campos electromagnéticos, formulando la relación matemática entre los campos eléctricos y magnéticos por medio de cuatro ecuaciones diferenciales (llamadas hoy "las ecuaciones de Maxwell") que relacionan el campo eléctrico y el magnético para una distribución espacial de cargas y corrientes. También demostró que la naturaleza de los fenómenos luminosos y electromagnéticos era la misma y que ambos se propagan a la velocidad de la luz.\citep{HDE}\\ 

%Su obra más importante es el Treatise on Electricity and Magnetism (Tratado de electricidad y magnetismo, 1873), en el que publicó sus famosas ecuaciones. También escribió: Matter and motion (Materia y movimiento, 1876) y Theory of Heat (Teoría del calor, 1877). La teoría de Maxwell obtuvo su comprobación definitiva cuando Heinrich Rudolf Hertz obtuvo en 1888 las ondas electromagnéticas de radio. \citep{HDE}\\


%Sus investigaciones posibilitaron la invención del telégrafo sin cables y la radio. La unidad de flujo magnético en el sistema cegesimal, el maxwell, recibe este nombre en su honor.

\section{Alexander Graham Bell: el teléfono (1876)}

El escocés-estadounidense Alexander Graham Bell, científico, inventor y logopeda (1847-1922), se disputó con otros investigadores la invención del teléfono y consiguió la patente oficial en los Estados Unidos en 1876 Previamente habían sido desarrollados dispositivos similares por otros investigadores, entre quienes destacó Antonio Meucci (1871), que entabló pleitos fallidos con Bell hasta su muerte, y a quien suele reconocerse actualmente la prelación en el invento.\citep{HDE}\\

%Bell contribuyó de un modo decisivo al desarrollo de las telecomunicaciones a través de su empresa comercial (Bell Telephone Company, 1877, posteriormente AT&T). También fundó en la ciudad de Washington el Laboratorio Volta, donde, junto con sus socios, inventó un aparato que transmitía sonidos mediante rayos de luz (el fotófono, 1880); y desarrolló el primer cilindro de cera para grabar (1886), lo que sentó las bases del gramófono. Participó en la fundación de la National Geographic Society y de la revista Science.\citep{HDE}\\

\section{Thomas Alva Edison: desarrollo de la lámpara incandescente (1879), Menlo Park y comercialización}

En el ámbito científico descubrió el efecto Edison, patentado en 1883, que consistía en el paso de electricidad desde un filamento a una placa metálica dentro de un globo de lámpara incandescente. Aunque ni él ni los científicos de su época le dieron importancia, este efecto sería uno de los fundamentos de la válvula de la radio y de la electrónica. En 1880 se asoció con el empresario J. P. Morgan para fundar la General Electric.\citep{HDE}\\

\section{John Hopkinson: el sistema trifásico (1882)}

El ingeniero y físico inglés John Hopkinson (1849-1898) contribuyó al desarrollo de la electricidad con el descubrimiento del sistema trifásico para la generación y distribución de la corriente eléctrica, sistema que patentó en 1882. Un sistema de corrientes trifásicas es el conjunto de tres corrientes alternas monofásicas de igual frecuencia y amplitud (y por consiguiente, valor eficaz) que presentan un desfase entre ellas de 120° (un tercio de ciclo). \citep{HDE}\\

%Cada una de las corrientes monofásicas que forman el sistema se designa con el nombre de fase. También trabajó en muchas áreas del electromagnetismo y la electrostática. De sus investigaciones estableció que "el flujo de inducción magnética es directamente proporcional a la fuerza magnetomotriz e inversamente proporcional a la reluctancia", expresión muy parecida a la establecida en la Ley de Ohm para la electricidad, y que se conoce con el nombre de Ley de Hopkinson También se dedicó al estudio de los sistemas de iluminación, mejorando su eficiencia, así como al estudio de los condensadores. Profundizó en los problemas de la teoría electromagnética, propuestos por James Clerk Maxwell. En 1883 dio a conocer el principio de los motores síncronos.\citep{HDE}\\

\section{Heinrich Rudolf Hertz: demostración de las ecuaciones de Maxwell y la teoría electromagnética de la luz (1887)}

El físico alemán Heinrich Rudolf Hertz (1857-1894) demostró la existencia de las ondas electromagnéticas predichas por las ecuaciones de Maxwell. Fue el primer investigador que creó dispositivos que emitían ondas radioeléctricas y también dispositivos que permitía detectarlas. Hizo numerosos experimentos sobre su modo y velocidad de propagación (hoy conocida como velocidad de la luz), en los que se fundamentan la radio y la telegrafía sin hilos, que él mismo descubrió. En 1887 descubrió el efecto fotoeléctrico. La unidad de medida de la frecuencia fue llamada Hertz (símbolo Hz) en su honor, castellanizada como Hercio.\citep{HDE}\\

\section{George Westinghouse: el suministro de corriente alterna (1886)}

Westinghouse compró al científico croata Nikola Tesla su patente para la producción y transporte de corriente alterna, que impulsó y desarrolló. Posteriormente perfeccionó el transformador, desarrolló un alternador y adaptó para su utilización práctica el motor de corriente alterna inventado por Tesla. \citep{HDE}\\

%En 1886 fundó la compañía eléctrica Westinghouse Electric & Manufacturing Company, que contó en los primeros años con la decisiva colaboración de Tesla, con quien logró desarrollar la tecnología necesaria para desarrollar un sistema de suministro de corriente alterna. Westinghouse también desarrolló un sistema para transportar gas natural, y a lo largo de su vida obtuvo más de 400 patentes, muchas de ellas de maquinaria de corriente alterna.

\section{Nikola Tesla: desarrollo de máquinas eléctricas, la bobina de Tesla (1884-1891) y el radiotransmisor (1893)}

El ingeniero e inventor de origen croata Nikola Tesla (1856-1943) emigró en 1884 a los Estados Unidos. Es reconocido como uno de los investigadores más destacados en el campo de la energía eléctrica. El Gobierno de Estados Unidos lo consideró una amenaza por sus opiniones pacifistas y sufrió el maltrato de otros investigadores mejor reconocidos como Marconi o Edison.\citep{HDE}\\

Desarrolló la teoría de campos rotantes, base de los generadores y motores polifásicos de corriente alterna. En 1887 logra construir el motor de inducción de corriente alterna y trabaja en los laboratorios Westinghouse, donde concibe el sistema polifásico para transmitir la electricidad a largas distancias. En 1893 consigue transmitir energía electromagnética sin cables, construyendo el primer radiotransmisor (adelantándose a Guglielmo Marconi). Ese mismo año en Chicago hizo una exhibición pública de la corriente alterna, demostrando su superioridad sobre la corriente continua de Edison. Los derechos de estos inventos le fueron comprados por George Westinghouse, que mostró el sistema de generación y transmisión por primera vez en la World's Columbian Exposition de Chicago de 1893. \citep{HDE}\\

%Dos años más tarde los generadores de corriente alterna de Tesla se instalaron en la central experimental de energía eléctrica de las cataratas del Niágara. Entre los muchos inventos de Tesla se encuentran los circuitos resonantes de condensador más inductancia, los generadores de alta frecuencia y la llamada bobina de Tesla, utilizada en el campo de las comunicaciones por radio.\citep{HDE}\\

%La unidad de inducción magnética del sistema MKS recibe el nombre de Tesla en su honor.\citep{HDE}\\

%\section{Guglielmo Marconi: la telegrafía inalámbrica (1899)}

%El ingeniero y físico italiano Guglielmo Marconi (1874-1937), es conocido, principalmente, como el inventor del primer sistema práctico de señales telegráficas sin hilos, que dio origen a la radio actual. En 1899 logró establecer comunicación telegráfica sin hilos a través del canal de la Mancha entre Inglaterra y Francia, y en 1903 a través del océano Atlántico entre Cornualles, y Saint John's en Terranova, Canadá. En 1903 estableció en los Estados Unidos la estación WCC, en cuya inauguración cruzaron mensajes de salutación el presidente Theodore Roosevelt y el rey Eduardo VIII de Inglaterra. \citep{HDE}\\

%En 1904 llegó a un acuerdo con el Servicio de Correos británico para la transmisión comercial de mensajes por radio. Las marinas italiana y británica pronto adoptaron su sistema y hacia 1907 había alcanzado tal perfeccionamiento que se estableció un servicio trasatlántico de telegrafía sin hilos para uso público. Para la telegrafía fue un gran impulso el poder usar el código Morse sin necesidad de cables conductores.

%Aunque se le atribuyó la invención de la radio, ésta fue posible gracias a una de las patentes de Nikola Tesla, tal y como fue reconocido por la alta corte de los Estados Unidos, seis meses después de la muerte de Tesla, hacia el año 1943. También inventó la antena Marconi. En 1909 Marconi recibió, junto con el físico alemán Karl Ferdinand Braun, el Premio Nobel de Física por su trabajo.\citep{HDE}\\

\section{Albert Einstein: El efecto fotoeléctrico (1905)}

Al alemán nacionalizado norteamericano Albert Einstein (1879 – 1955) se le considera el científico más conocido e importante del siglo XX. El resultado de sus investigaciones sobre la electricidad llegó en 1905 (fecha trascendental que se conmemoró en el Año mundial de la física 2005), cuando escribió cuatro artículos fundamentales sobre la física de pequeña y gran escala. En ellos explicaba el movimiento browniano, el efecto fotoeléctrico y desarrollaba la relatividad especial y la equivalencia entre masa y energía.\citep{HDE}\\

El efecto fotoeléctrico consiste en la emisión de electrones por un material cuando se le ilumina con radiación electromagnética (luz visible o ultravioleta, en general). Ya había sido descubierto y descrito por Heinrich Hertz en 1887, pero la explicación teórica no llegó hasta que Albert Einstein le aplicó una extensión del trabajo sobre los cuantos de Max Planck. \citep{HDE}\\

%En el artículo dedicado a explicar el efecto fotoeléctrico, Einstein exponía un punto de vista heurístico sobre la producción y transformación de luz, donde proponía la idea de quanto de radiación (ahora llamados fotones) y mostraba cómo se podía utilizar este concepto para explicar el efecto fotoeléctrico. Una explicación completa del efecto fotoeléctrico solamente pudo ser elaborada cuando la teoría cuántica estuvo más avanzada. A Albert Einstein se le concedió el Premio Nobel de Física en 1921.\citep{HDE}\\

%El efecto fotoeléctrico es la base de la producción de energía eléctrica por radiación solar y de su aprovechamiento energético. Se aplica también para la fabricación de células utilizadas en los detectores de llama de las calderas de las grandes usinas termoeléctricas. También se utiliza en diodos fotosensibles tales como los que se utilizan en las células fotovoltaicas y en electroscopios o electrómetros. En la actualidad (2008) los materiales fotosensibles más utilizados son, aparte de los derivados del cobre (ahora en menor uso), el silicio, que produce corrientes eléctricas mayores.\citep{HDE}\\

\section{Años 1950}

\enumarate{

    \item 1951: comienza a operar la EDVAC, a diferencia de la ENIAC, no era decimal, sino binaria y tuvo el primer programa diseñado para ser almacenado.
     
    \item 1951: Eckert y Mauchly entregan a la Oficina del Censo su primer computador: el UNIVAC I.
    
    \item 1951: el Sistema A-0 fue inventado por Grace Murray Hopper. Fue el compilador desarrollado para una computadora electrónica.
    
    \item 1952: Claude Elwood Shannon desarrolla el primer ratón eléctrico capaz de salir de un laberinto, considerada la primera red neural.
    
    \item 1953: IBM fabrica su primera computadora a escala industrial, la IBM 650. Se amplía el uso del lenguaje ensamblador para la programación de las computadoras.
    
    \item 1953: se crean memorias a base de magnetismo (conocidas como memorias de núcleos magnéticos).
    
    \item 1954: se desarrolla el lenguaje de programación de alto nivel Fortran.
    
    \item 1956: Darthmouth da una conferencia en a partir de la que nace el concepto de inteligencia artificial.
    
    \item 1956: Edsger Dijkstra inventa un algoritmo eficiente para descubrir las rutas más cortas en grafos como una demostración de las habilidades de la computadora ARMAC.
    
    \item 1957: IBM pone a la venta la primera impresora de matriz de puntos.
    
    \item 1957: se funda la compañía Fairchild Semiconductor.
    
    \item 1957: Jack S. Kilby construye el primer circuito integrado.
    
    \item 1957: la organización ARPA es creada como consecuencia tecnológica de la llamada Guerra Fría.
    
    \item 1957: puesta en marcha de SAPO, el primer ordenador con tolerancia a fallos.
    
    \item 1958: comienza la segunda generación de computadoras, caracterizados por usar circuitos transistorizados en vez de válvulas al vacío.
    
    \item 1960: se desarrolla COBOL, el primer lenguaje de programación de alto nivel transportable entre modelos diferentes de computadoras.
    
    \item 1960: aparece ALGOL, el primer lenguaje de programación estructurado y orientado a los procedimientos.
    
    \item 1960: se crea el primer compilador de computador.
    
    \item 1960: C. Antony R. Hoare desarrolla el algoritmo de ordenamiento o clasificación llamado quicksort.
}

%\section{Tarjetas Perforadas}

%La tarjeta perforada o simplemente tarjeta es una lámina hecha de cartulina que contiene información en forma de perforaciones según un código binario. Estos fueron los primeros medios utilizados para ingresar información e instrucciones a una computadora en los años 1960 y 1970. Las tarjetas perforadas fueron usadas con anterioridad por Joseph Marie Jacquard en los telares de su invención, de donde pasó a las primeras computadoras electrónicas. Con la misma lógica se utilizaron las cintas perforadas.\citep{TP}\\

%Actualmente las tarjetas perforadas han sido reemplazadas por medios magnéticos y ópticos de ingreso de información. Sin embargo, muchos de los dispositivos de almacenamiento actuales, como por ejemplo el CD-ROM también se basan en un método similar al usado por las tarjetas perforadas, aunque por supuesto los tamaños, velocidades de acceso y capacidad de los medios actuales no admiten comparación con los antiguos medios.\citep{TP}\\

\section{Años 1960}

\enumarate{

     \item 1961: en IBM, Kenneth Iverson inventa el lenguaje de programación APL.
    \item  1961: T. Kilburn y D. J. Howart describen por primera vez el concepto de paginación de memoria.
    \item  1962: en el MIT, Ivan Sutherland desarrolla los primeros programas gráficos que dejan que el usuario dibuje interactivamente en una pantalla.
    \item  1962: en el MIT, Hart y Levin inventan para Lisp el primer compilador autocontenido, es decir, capaz de compilar su propio código fuente.
    \item  1962: un equipo de la Universidad de Mánchester completa la computadora ATLAS. Esta máquina introdujo muchos conceptos modernos como interrupciones, pipes (tuberías), memoria entrelazada, memoria virtual y memoria paginada. Fue la máquina más poderosa del mundo en ese año.
    \item  1962: el estudiante del MIT Steve Russell escribe el primer juego de computadora, llamado Spacewar!.

Caracteres ASCII imprimibles, del 32 al 126.

    \item  1962: un comité industrial-gubernamental define el código estándar de caracteres ASCII.
    \item  1963: DEC (Digital Equipment Corporation) lanza el primer minicomputador comercialmente exitoso.
    \item  1964: la aparición del IBM 360 marca el comienzo de la tercera generación de computadoras. Las placas de circuito impreso con múltiples componentes elementales pasan a ser reemplazadas con placas de circuitos integrados.
    \item  1964: aparece el CDC 6600, la primera supercomputadora comercialmente disponible.
    \item 1964: en el Dartmouth College, John George Kemeny y Thomas Eugene Kurtz desarrollan el lenguaje BASIC (el Dartmouth BASIC).
    \item  1965: Gordon Moore publica la famosa Ley de Moore.
    %\item  1965 - Ted Nelson teoriza sobre la red de documentos hiperenlazados o documentos web.
    %\item 1965: La lógica difusa, diseñada por Lofti Zadeh, se usa para procesar datos aproximados.
    %\item 1965: J. B. Dennis introduce por primera vez el concepto de segmentación de memoria.
    %\item 1965: en los clásicos documentos de Dijkstra se tratan por primera vez los algoritmos de exclusión mutua (sistemas operativos).
    \item 1966: la mayoría de ideas y conceptos que existían sobre redes se aplican a la red militar ARPANET.
    %\item 1966: aparecen los primeros ensayos que más tarde definirían lo que hoy es la programación estructurada.
    \item 1967: en el MIT, Richard Greenblatt inventa los primeros programas exitosos de ajedrez.
    \item 1967: en IBM, David Noble ―bajo la dirección de Alan Shugart― inventa el disquete (disco flexible).
    \item 1968: Robert Noyce y Gordon Moore fundan la corporación Intel.
    %\item 1968: Douglas Engelbart lleva a cabo "The Mother of All Demos".
    \item 1969: el protocolo de comunicaciones NCP se crea para controlar la red militar ARPANET.
    \item 1969: Data General Corporation distribuye la primera minicomputadora de 16-bit.
    \item 1969: en los laboratorios Bell, Ken Thompson y Dennis Ritchie desarrollan el lenguaje de programación B.
    \item 1969: en los laboratorios Bell de AT&T, un grupo de empleados de dicho laboratorio entre los que se encuentran Ken Thompson, Dennis Ritchie y Douglas Mcllroy― crean el sistema operativo UNICS.
    En abril de 1969 se publica el RFC 1, que describe la primera Internet (entonces ARPANET). La libre disposición de las RFC, y en particular de las especificaciones de los protocolos usados en Internet fue un factor clave de su desarrollo.

}

\section{Años 1970}

\enumarate{

    \item 1970: el sistema UNICS es renombrado como Unix.
    %\item 1970: la empresa Corning Glass Works vende comercialmente el primer cable de fibra óptica.
    %\item 1970: E. F. Codd se publica el primer modelo de base de datos relacional.
    \item 1970: el profesor suizo Niklaus Wirth desarrolla el lenguaje de programación Pascal.
    %\item 1970: Brinch Hansen utiliza por primera vez la comunicación interprocesos en el sistema RC 400.
    \item 1970: Intel crea la primera memoria dinámica RAM. Se le llamó i1103 y tenía una capacidad de 1024 bits (1 kbits).
    \item 1970: Se funda La división de investigación Xerox PARC
    \item 1971: Intel presenta el primer procesador comercial y a la vez el primer chip microprocesador, el Intel 4004.
    \item 1971: Ray Tomlinson crea el primer programa para enviar correo electrónico. Como consecuencia, la arroba se usa por primera vez con fines informáticos.
    \item 1971: en el MIT, un grupo de investigadores presentan la propuesta del primer "Protocolo para la transmisión de archivos en Internet" (FTP).
    \item 1971: Texas Instruments vende la primera calculadora electrónica portátil.
    \item 1971: John Blankenbaker presenta el Kenbak-1, considerado como el primer ordenador personal de la historia, sin un procesador, solo con puertas lógicas. Solo vende 40 unidades en centros de enseñanza.
    \item 1972: aparecen los disquetes de 5 1/4 pulgadas.

    \item 1972: Intel desarrolla y pone a la venta el procesador Intel 8008.
    1972: sale al mercado la primera consola de videojuegos, la Magnavox Odyssey.
    \item 1972: C. A. R. Hoare y Per Brinch Hansen desarrollan el concepto de región crítica.
    \item 1973: ARPA cambia su nombre por DARPA.
    \item 1973: La división de investigación Xerox PARC desarrollo el primer ordenador que utilizó el concepto de Computadora de Escritorio llamado Xerox Alto, además de ser el primer ordenador en utilizar una GUI y un mouse.
    \item 1974: Vint Cerf y Robert Kahn crean el TCP (protocolo de control de transmisión).
    \item 1974: se crea el sistema Ethernet para enlazar a través de un cable único a las computadoras de una LAN (red de área local).
    \item 1974: Gary Kildall crea el sistema operativo CP/M (Control Program for Microcomputer).
    \item 1974: Federico Faggin funda la compañía ZiLOG inc. la cual crea un procesador el Zilog Z80 que está basado en el Intel 8080 (que Faggin mismo diseñó cuando era empleado de esa compañía), fue lanzado en 1976; éste alcanzó gran popularidad durante los próximos años.
    \item 1975: en enero la revista Popular Electronics hace el lanzamiento del Altair 8800, el primer microcomputador personal reconocible como tal.
    \item 1975: se funda la empresa Microsoft.
    \item 1975: Raphael Finkell publica una colección de términos técnicos usados por las comunidades de hackers desde principios de los años 60, llamada jargon-1.1​
    \item 1976: se funda la empresa Apple.
    \item 1977: Apple presenta el primer computador personal que se vende a gran escala, el Apple II, desarrollado por Steve Jobs y Steve Wozniak en un garaje.
    \item 1977: sale al mercado el ordenador TRS-80 el primero que utiliza un procesador Zilog Z80.
    %\item 1978: se desarrolla el famoso procesador de textos WordStar, originalmente para plataforma CP/M (Control Program for Microcomputer).
    \item 1978: Donald Knuth comienza a trabajar en el proyecto TeX, un sistema de tipografía en código libre.
    %\item 1979: Dan Bricklin crea la primera hoja de cálculo, que más tarde sería denominada VisiCalc.
    \item 1979: Toru Iwatani, de la empresa Namco, crea el juego Pacman.
    \item 1980: en IBM, un grupo de investigación desarrolla el primer prototipo de RISC (Computadora de Instrucción Reducida).
    \item 1980: la empresa Mycron lanza la primera microcomputadora de 16 bits, llamada Mycron 2000.
    \item 1980: Laboratorios Bell desarrolla el primer microprocesador de 32-bit en un solo chip, llamado Bellmac-32.

}

\section{Años 1980}


\enumarate{

    \item 1981: se lanza al mercado el IBM PC, que se convertiría en un éxito comercial, marcaría una revolución en el campo de la computación personal y definiría nuevos estándares.
    \item  1981: se termina de definir el protocolo TCP/IP.
    \item  1981 (3 de abril): Adam Osborne lanza el Osborne 1, primer ordenador portable (no portátil ya que no usaba baterías).
    \item  1981: Sony crea los disquetes de 3 1/2 pulgadas.
    \item  1982: Microsoft saca al mercado el sistema operativo MS-DOS, que sería el más utilizado en los IBM PC compatibles hasta mediados de los años 90.
    \item  1982: la Asociación Internacional MIDI publica el MIDI (protocolo para comunicar computadoras con instrumentos musicales).
    \item  1982: Rod Canion, Jim Harris y Bill Murto fundan Compaq (Computer Corporation), una compañía de computadoras personales.
    \item  1982: aparece el Sinclair ZX Spectrum.
    \item 1982: creación del protocolo SMTP, que permitía por primera vez el intercambio de correos electrónicos en ARPANET.



    \item 1983: Microsoft ofrece la versión 1.0 del procesador de textos Word para DOS.
    \item 1983: Compaq fabrica el primer compatible IBM PC, el Compaq portable.
    \item 1983: ARPANET se separa de la red militar que la originó, de modo que, ya sin fines militares, se puede considerar esta fecha como el nacimiento de Internet.
    \item 1983: el 27 de septiembre, Richard Stallman anuncia públicamente el proyecto GNU. Este texto inicial fue desarrollado en el Manifiesto GNU, publicado en 19852​. Poco tiempo después Richard Stallman libera el Emacs con licencia GPL.
    %\item 1983: Bjarne Stroustrup publica el lenguaje de programación C++.
    \item 1983: Sun lanza su primer sistema operativo, llamado SunOS.
    %\item 1983: la compañía Lotus Software lanza el famoso programa de hoja de cálculo Lotus 1-2-3.
    %\item 1983: el sistema DNS (de Internet) ya posee 1000 hosts.
    \item 1983: se funda Borland.
%    \item 1984: IBM presenta el IBM Personal Computer/AT, con procesador Intel 80286, bus de expansión de 16 bits y 6 MHz de velocidad. Tenía hasta 512 kB de memoria RAM, un disco duro de 20 MB y un monitor monocromático. Su precio en ese momento era de 5795 dólares.
    \item 1984: Apple Computer presenta su Macintosh 128K con el sistema operativo Mac OS, el cual introduce la interfaz gráfica ideada en Xerox.
    \item 1984: las compañías Philips y Sony crean los CD-Roms para computadores.
    \item 1984: se desarrolla el sistema de ventanas X bajo el nombre X1 para dotar de una interfaz gráfica a los sistemas Unix.
    \item 1984: aparece el lenguaje LaTeX para procesamiento de documentos.
    \item 1984: Hewlett-Packard lanza su popular impresora láser llamada LaserJet.
    \item 1984: Leonard Bosack y Sandra Lerner fundan Cisco Systems que es líder mundial en soluciones de red e infraestructuras para Internet.
    \item 1985: Microsoft presenta el sistema operativo Windows 1.0.
    \item 1985: Compaq saca a la venta la Compaq Deskpro 286, una PC IBM Compatible de 16-bits con microprocesador Intel 80286 corriendo a 6 MHz y con 7 MB de RAM, fue considerablemente más rápida que una PC IBM. Fue la primera de la línea de computadoras Compaq Deskpro.
    \item 1985: en octubre se crea la organización Free Software Foundation (FSF), receptor de fondos para proyecto GNU.
    \item 1985: Bertrand Meyer crea el lenguaje de programación Eiffel.
    \item 1985: Adobe crea el PostScript.
    \item 1985: el ruso Alekséi Pázhitnov crea el juego Tetris.
 %   \item 1986: ISO estandariza SGML, lenguaje en que posteriormente se basaría XML.
    \item 1986: Compaq lanza el primer computador basado en el procesador de 32 bits Intel 80386, adelantándose a IBM.
    \item 1986: HP lanza al mercado el HP PA RISC con un sistema basado en BSD llamado HP-UX. También AT&T incorpora Vi y librerías curses propiedad de BSD en su Unix System V Release 1
    \item 1986: IBM lanza al mercado AIX v1 basada en un System V Release 3, que corría en la IBM RT/PC (AIX/RT).
    \item 1986: el lenguaje SQL es estandarizado por ANSI.
    \item 1986: aparece el programa de cálculo algebraico de computadora MathCad.
    \item 1986: se registra la primera patente base de codificación de lo que hoy conocemos como MP3 (un método de compresión de audio).
  %  \item 1986: Compaq pone en venta la PC compatible Compaq Portable II, mucho más ligera y pequeña que su predecesora, que utilizaba microprocesador de 8 MHz y 10 MB de disco duro, y fue 30\% más barata que la IBM PC/AT con disco rígido.
   % \item 1986: Mark Crispin crea el protocolo IMAP.
    %\item 1987: se desarrolla la primera versión del actual protocolo X11.
    \item 1987: Larry Wall crea el lenguaje de programación Perl.
    \item 1987: el proyecto GNU crea el conjunto de compiladores llamado "GNU Compiler Collection".
    \item 1987: Compaq introduce la primera PC basada en el nuevo microprocesador de Intel; el 80386 de 32 bits, con la Compaq Portable 386 y la Compaq Portable III. Aún IBM no estaba usando este procesador. Compaq marcaba lo que se conocería como la era de los clones de PC.
    \item 1988: Soft Warehouse desarrolla el programa de álgebra computacional llamado Derive.
    \item 1988: Stephen Wolfram y su equipo sacan al mercado la primera versión del programa Mathematica.
    \item 1988: aparece el primer documento que describe lo que hoy se conoce como firewalls (cortafuegos).
    \item 1988: aparece el estándar XMS.
    %\item 1989: Creative Labs presenta la reconocida tarjeta de sonido Sound Blaster.
    \item 1989: T. E. Anderson estudia las cuestiones sobre el rendimiento de las hebras o hilos en sistemas operativos (threads).
    \item 1989: en enero UNIX Software Operation se organiza como división separada de AT&T; en noviembre UNIX Software Operation lanza al mercado UNIX System V Release 4.0.
    %\item 1989: Se funda Cygnus Solutions más tarde Cygnus Software, la primera empresa dedicada fundamentalmente a proporcionar servicios comerciales para el software libre (entre ellos, soporte, desarrollo y adaptación de programas libres).
    \item 1990: Tim Berners-Lee idea el hipertexto para crear el World Wide Web (www) una nueva manera de interactuar con Internet. También creó las bases del protocolo de transmisión HTTP, el lenguaje de documentos HTML y el concepto de los URL.
    \item 1990: en AT&T (Laboratorios Bell) se construye el primer prototipo de procesador óptico.3​
    \item 1990: La FSF anuncia su intento de construir un kernel que se llamará GNU Hurd. La meta de este proyecto es completar el mayor hueco que queda en la estrategia del proyecto GNU de construir un sistema operativo completo.
    \item 1990: Guido van Rossum crea el lenguaje de programación Python.

}

\section{Años 1990}

\enumarate{

    \item 1991: Linus Torvalds comenzó a desarrollar Linux, un sistema operativo compatible con Unix.
    \item 1991: Comienza a popularizarse la programación orientada a objetos.
    %\item 1991: Surge la primera versión del estándar Unicode.
    %\item 1991: Adobe Systems Incorporated lanza al mercado la primera versión de Adobe Premiere, una aplicación orientada a la Edición de vídeos.
    \item 1991: Compaq pone a la venta al por menor con la Compaq Presario, y fue uno de los primeros fabricantes en los mediados de los años noventa en vender una PC de menos de 1000 dólares estadounidenses. Compaq se convirtió en una de los primeros fabricantes en usar micros de AMD y Cyrix.
    %\item 1991: El protocolo Gopher es creado en la Universidad de Minnesota.4​
    \item 1992: Es introducida la arquitectura de procesadores Alpha diseñada por DEC bajo el nombre AXP, como reemplazo para la serie de microcomputadores VAX que comúnmente utilizaban el sistema operativo VMS y que luego originaría el openVMS. El procesador Alpha 21064 de 64 bits y 200 MHz es declarado como el más rápido del mundo.
    %\item 1992: Un grupo de estudiantes de la Universidad de Helsinki crean el OTH-Erwise, el primer navegador gráfico.
    %\item 1992: Pei-Yuan Wei crea el ViolaWWW, un navegador gráfico anterior al Mosaic, pero con funcionalidades e integración como los actuales navegadores.
    \item 1992: Microsoft lanza Windows 3.1.
    \item 1992: Aparece la primera versión del sistema operativo Solaris.
    \item 1992: GNU comienza a utilizar el núcleo Linux.
    \item 1993: un grupo de investigadores descubren que un rasgo de la mecánica cuántica, llamado entrelazamiento, podía utilizarse para superar las limitaciones de la teoría del cuanto aplicada a la construcción de computadoras cuánticas y a la teleportación.
    \item 1993: Microsoft lanza al mercado la primera versión del sistema operativo multiusuario de 32 bits (cliente-servidor) Windows NT.
    \item 1993: Marc Andreessen y Eric Bina crean en el NCSA el navegador web gráfico Mosaic para Unix, base de las primeras versiones de Mozilla y Spyglass (más tarde adquirido por Microsoft y renombrado Internet Explorer).
    \item 1993: se funda SuSE, en Alemania, que comienza sus negocios distribuyendo Slackware Linux, traducida al alemán.
    %\item 1993: se crea la lista TOP500 que recopila los 500 ordenadores más potentes de la tierra.
    %\item 1994: Marc Andreessen crea el famoso navegador web Netscape Navigator.
    \item 1994: es diseñado el PHP, originalmente en lenguaje Perl, seguidos por la escritura de un grupo de CGI binarios escritos en el lenguaje C por el programador danés-canadiense Rasmus Lerdorf.
    \item 1994: Jerry Yang Chih-Yuan y David Filo fundan Yahoo! Incorporated, referente en portales de servicios y directorios web.
    \item 1995: Microsoft lanza al mercado el sistema operativo Windows 95, junto con su navegador web predeterminado, Windows Internet Explorer; fue (al menos en su época) el sistema operativo más ""incluyente"" al traer consigo hasta 12 paquetes de idiomas distintos .5​
    \item 1995: aparece la primera versión de MySQL.
    \item 1995: se inicia el desarrollo del servidor Apache.
    %\item 1995: la implementación original y de referencia del compilador, la máquina virtual y las librerías de clases de Java fueron desarrollados por Sun Microsystems.
    \item 1995: se presenta públicamente el lenguaje de programación Ruby.
    \item 1995: se especifica la versión 1.5 del DVD, base actual del DVD.
    \item 1995: Sony Computer Entertaiment, lanza al mercado la Playstation, el primer controlador de videojuego en utilizar un CD-ROM como dispositivo de entrada para acceder a los datos que constituyen el videojuego.
    %\item 1996: se crea Internet2, más veloz que la Internet original.
    \item 1996: se publica la primera versión del navegador web Opera.
    \item 1996: se inicia el proyecto KDE.
    \item 1996: la tecnología de DjVu fue originalmente desarrollada en los laboratorios de AT&T.
    \item 1996: aparece la primera versión de SuperCollider.
    %\item 1996: Sabeer Bhatia y Jack Smith fundan Hotmail.
    \item 1997: la empresa estadounidense Nullsoft distribuye gratuitamente el reproductor multimedia Winamp.
    \item 1997: aparece la primera versión pública de FlightGear.
    \item 1997: Spencer Kimball y Peter Mattis crean la inicial librería GTK+.
    \item 1997: EL IEEE crea la primera standard WLAN y la llamaron 802.11.
    \item 1998: la W3C publica la primera versión de XML.
    \item 1998: Microsoft lanza al mercado el sistema Windows 98.
    %\item 1998: Compaq adquiere Digital Equipment Corporation, la compañía líder en la anterior generación de las computadoras durante los años setenta y principios de los ochenta. Esta adquisición convertiría a Compaq en el segundo más grande fabricante de computadoras, en términos de ingresos.
    \item 1998: Larry Page y Serguéi Brin fundan Google Inc.
    \item 1998: Netscape Communications libera el código fuente de su navegador Nestcape Navigator e inicia el proyecto Mozilla.6​

GNOME.

    \item 1999: aparece el entorno de escritorio GNOME.
    \item 1999: Sean Parker y Shawn Fanning fundan Napster un servicio p2p orientado a compartir archivos mp3 de un usuario a otro.
    \item 1999: Microsoft publica la primera versión de MSN Messenger.
    \item 1999: Macintosh lanza Mac OS 9.
    \item 1999: IEEE crea el 802.11b expandiendo así las limitaciones de su predecesor 802.11, como por ejemplo soportar un ancho de banda de hasta 11 Mbps.
    \item 2000: Sony Computer Entertaiment, lanza al mercado la Playstation 2
    \item 2000: un equipo de investigadores de IBM construye el prototipo de computador cuántico.
    \item 2000: Microsoft lanza el sistema operativo Windows 2000.
    \item 2000: Microsoft lanza el sistema operativo Windows Me.
    \item 2000: Macintosh lanza el sistema operativo Mac OS X.

}

\section{Siglo XXI}

\enumarate{
\item 2005: creación Youtube
\item 2007: creación Facebook
}



\bibliographystyle{plain}
\bibliography{LineaTiempo.bib}


\end{document}