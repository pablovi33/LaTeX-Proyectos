\documentclass{article}
\usepackage[utf8]{inputenc}
\usepackage[spanish.mexico]{babel}

\title{Dispositivos y circuitos Electrónicos}
\author{Pablo Vivar Colina}
\date{Septiembre 2017}

\usepackage{natbib}
\usepackage{graphicx}

\usepackage{tikz}
\usepackage[american voltages, american currents,siunitx]{circuitikz}

\begin{document}


\maketitle

%%\usepackage[top=2cm,bottom=2cm,left=1cm,right=1cm]{geometry}


\begin{titlepage}
     \begin{center}
	\includegraphics[width=0.09\textwidth]{UNAM}\Large Universidad Nacional Autónoma de México
        	\includegraphics[width=0.09\textwidth]{FI}\\[1cm]
        \Large Facultad de Ingeniería\\[1cm]
       % \Large División de Ciencias Básicas\\[1cm]
         \Large Laboratorio de Dispositivos y Circuitos Electrónicos (6654)\\[1cm]
         %la clave antes era:4314
         \footnotesize Profesor: Zapata Rosales Arturo Ing.\\[1cm]
        \footnotesize Semestre 2018-1\\[1cm]
        %\Large Práctica No. 1\\[1cm]
    
        %\Large Práctica No. 2\\[1cm]
        
        %\Large Práctica No. 3\\[1cm]
       
        %\Large Práctica No. 4\\[1cm]
         
               
         %\Large Práctica No. 5\\[1cm]
         
         
         %\Large Práctica No. 6\\[1cm]
         
         %\Large Práctica No. 7\\[1cm]
         
             %\Large Práctica No. 8\\[1cm]
       

        \Large Práctica No. 9\\[1cm]
        
           %####AQUI VAMOS#### ya ahora sii
           
        %\Large Práctica No. 11\\[1cm]
        %\Large Práctica No. 12\\[1cm]
        %\Large Práctica No. 13\\[1cm]
        
        %\Large Amplificador Operacional como Integrador\\[1cm]
        %\Large{Filtros}\\[1cm]
         %\Large{Medición de  corrientes en un circuito}\\[1cm]
         %practica 4
         %Large{Amplificador operacional como seguidor de voltaje en entrada inversora}\\[1cm]
         %practica5
         %\Large{Amplificador operacional como integrador}
         
         %Practica 7
%Comportamiento de un diodo Zener

\Large Diodo Zener
        
         %Texto a la derecha
          \begin{flushright}
\footnotesize  Grupo 13\\[0.5cm]
\footnotesize Brigada: 7\\[0.5cm]

\footnotesize Vivar Colina Pablo\\[0.5cm]
 \end{flushright}
    %Texto a la izquierda
          \begin{flushleft}
        \footnotesize Ciudad Universitaria Abril de 2018.\\
          \end{flushleft}
         
          
        %\vfill
        %\today
   \end{center}
\end{titlepage}
 %agregar portada

\section{Previo 10}

\section{Circuito}

En el experimento de la práctica 8 se construyó un rectificador de onda completa, que se puede apreciar en la figura \ref{fig:rectificadorOndaCompleta} y se intercambiaron los valores del resistor y del capacitor en orden de obtener una mejor señal del generador par qué el voltaje de rizado fuera menor.\\

\begin{figure}[h!]
    \centering
    \begin{circuitikz}
    
        \draw (0,0) node [transformer](T){};
        \draw  (1,0) to[D,v=A](3,0); 
        \draw  (1,-2.1) to[D,v=B](3,-2.1)
        (3,-2.1)--(3,-1.3)
        (3,-1.3) arc (-90:90:0.25) 
        (3,0)--(3,-0.8)
        
        ; 
        
        \draw (4,0)to[C,l=C](4,-1.05)
        (3,0)--(6,0)
        (0,-1.05)--(6,-1.05)
        
       ;
        \draw  (5.5,0) to[R,l=R](5.5,-1.05)
        
        ;
        
    \end{circuitikz}
    \caption{Rectificador Onda Completa}
    \label{fig:rectificadorOndaCompleta}
\end{figure}


\begin{table}[ht!]
\centering
\begin{tabular}{|c|c|c|c|c|c|}
\hline
Prueba & V rizo       & V directa & V pico   & C       & R    \\ \hline
1      & 7V           & 5V        & 12V      & 0.047uF & 1k   \\ \hline
2      & 3V           & 6V        & 9V       & 10uF    & 1k   \\ \hline
3      & 50mV         & 6V        & 7V       & 1000uF  & 1k   \\ \hline
4      & 9mV          & 7V        & 7V       & 1000uF  & 1k   \\ \hline
5      & 0.5V         & 6V        & 6V       & 10uF    & 10k  \\ \hline
6      & 6.8V         & 4V        & 10.8V    & 0.047uF & 10k  \\ \hline
7      & 5V           & 5V        & 10V      & 0.047uF & 100k \\ \hline
8      & 0.5mV        & 7V        & 7.005V   & 0.057uF & 100k \\ \hline
9      & \textless5mV & 8V        & aprox 8V & 1000uF  & 100k \\ \hline
\end{tabular}


\caption{Registro de datos}
\label{datosRectificadorOC}
\end{table}

De acuerdo a la tabla del experiemnto del rectificador de onda completa \ref{datosRectificadorOC} se puede apreciar que la combinación de valores uqe ha resultado mejor es un capacitor de 1000 $\mu$F y un resitor de 100k$\Omega$ ya que presenta un valor de voltaje de rizo menor a los demás y un voltaje de directa mayor a los demás.\\

\section{Circuito Práctica 9}

\begin{figure}[h!]
    \centering
    \begin{circuitikz}
    
    
      \draw
      (0,-3)--(0,-2)
      (0,-3)--(6,-3)
      (0,0) to [battery1,l=F](0,-2)
      ;
       \draw
       (3,0)to[R,l=R](6,0)
       (6,0)to[zzD,l=D](6,-3)
      ;
        \draw
    
        (0,0)to[ammeter](3,0);
        
    \end{circuitikz}
    \caption{Regulador diodo Zener Directa}
    \label{fig:reguladorZ}
\end{figure}

%Determine el valor de Rx para el circuito regulador de la práctica 9 del diodo zener

%¿Qué no hace un diodo?

Es importante recordar en los parámetros de diseño que un diodo no amplifica ni altera señales.\\

%Circuito rectificador de media onda con mejor filtro de la práctica 8




%\bibliographystyle{plain}
%\bibliography{referenciasPrev10}
\end{document}