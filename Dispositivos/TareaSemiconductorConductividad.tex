\documentclass{article}
\usepackage[utf8]{inputenc}
\usepackage[spanish.mexico]{babel}
\usepackage[american voltages, american currents,siunitx]{circuitikz}

%para fotos

\usepackage{graphicx}
\usepackage{subcaption}

%\title{Modelo atómico}
\author{Pablo Vivar Colina A7\\
Grupo 5 Tarea 7
}
%\date{Septiembre 2017}

\usepackage{natbib}
\usepackage{graphicx}

\begin{document}

\maketitle

\section{Teoría de bandas}

En física de estado sólido, teoría según la cual se describe la estructura electrónica de un material como una estructura de bandas electrónicas, o simplemente estructura de bandas de energía. La teoría se basa en el hecho de que en una molécula los orbitales de un átomo se solapan produciendo un número discreto de orbitales moleculares.\citep{TeorB}\\

 \subsection{Bandas de energía}
 
 \begin{itemize}
    
     \item La banda de valencia (BV): está ocupada por los electrones de valencia de los átomos, es decir, aquellos electrones que se encuentran en la última capa o nivel energético de los átomos. Los electrones de valencia son los que forman los enlaces entre los átomos, pero no intervienen en la conducción eléctrica.
     \citep{TeorB}
     \item  La banda de conducción (BC): está ocupada por los electrones libres, es decir, aquellos que se han desligado de sus átomos y pueden moverse fácilmente. Estos electrones son los responsables de conducir la corriente eléctrica.
     \citep{TeorB}

 \end{itemize}
 
 En consecuencia, para que un material sea buen conductor de la corriente eléctrica debe haber poca o ninguna separación entre la BC y la BV (que pueden llegar a solaparse), de manera que los electrones puedan saltar entre las bandas. Cuando la separación entre bandas sea mayor, el material se comportará como un aislante. En ocasiones, la separación entre bandas permite el salto entre las mismas de solo algunos electrones. En estos casos, el material se comportará como un semiconductor. Para que el salto de electrones entre bandas en este caso se produzca deben darse alguna o varias de las siguientes situaciones: que el material se encuentre a altas presiones, a una temperatura elevada o se le añadan impurezas (que aportan más electrones).\citep{TeorB}

%Entre la banda de valencia y la de conducción existe una zona denominada banda prohibida o gap, que separa ambas bandas y en la cual no pueden encontrarse los electrones.\citep{TeorB}

\begin{figure}
    \centering
    \includegraphics[scale=0.4]{Imagenes/bandasEnergia.png}
    \caption{Representación esquemática de las bandas de energía en un sólido.}
    \label{fig:bandasEnergia}
\end{figure}

\section{Conductividad Semiconductores}

\begin{table}[h!]
\centering

\begin{tabular}{|c|c|}
\hline
Semiconductores & Conductividad Eléctrica {[}S/m{]} \\ \hline
Carbono 	      & 80 × 104                          \\ \hline
Germanio 	     & 20 × 10−2                         \\ \hline
Silicio 	      & 60 × 10−5                         \\ \hline
\end{tabular}

\caption{Conductividad Semiconductores\citep{CondE}}
\label{tabla-conductividad-semiconductores}

\end{table}


\section{Banda Prohibida}

La banda prohibida, brecha de bandas o brecha energética (en inglés bandgap), en la física del estado sólido y otros campos relacionados, es la diferencia de energía entre la parte superior de la banda de valencia y la parte inferior de la banda de conducción. Esta cantidad se encuentra presente en aislantes y semiconductores, su predicción puede llegar a ser un reto para muchos de los métodos teóricos relacionados con la Teoría de bandas.\citep{BandaPBD}\\

La conductividad eléctrica de un semiconductor intrínseco (puro) depende en gran medida de la anchura del gap. Los únicos portadores útiles para conducir son los electrones que tienen suficiente energía térmica para poder saltar la banda prohibida, la cual se define como la diferencia de energía entre la banda de conducción y la banda de valencia. La probabilidad de que un estado de energía $E_0$ esté ocupado por un electrón se calcula mediante las estadísticas de Fermi-Dirac.\citep{BandaPBD}\\

%\section{Circuito a resolver}

%\begin{figure}[h!]
 %   \centering
   % \includegraphics{}
  %  \begin{circuitikz}
%\draw

%(-1,0)--(-1,-1)
%(-1,-1) to[V,l=$28v$](-1,-2) 
% (-1,-2)--(-1,-3)
 
 %(-1,-3)--(2,-3)
 
% (-1,0) to[R,l=$50 \Omega $](2,0)
 
 
% (2,-3)to[R,l=$5 \Omega$](2,0)
 
% (2,0)--(4,0)
% (2,-3)--(4,-3)
 
 % (4,-3)to[R,l=$10 \Omega$](4,0)
 
 
%;

%\draw[thick,arrows=->]
%(2.5,0)--(2.5,-1)
%;
 
%\end{circuitikz}
 %   \caption{Circuito a resolver}
  %  \label{fig:circuito}
%\end{figure}

%\subsection{Resultados}

%Para resolver el circuito se hicieron equivalente los resistores de 5 y 10 $\Omegas$ para obtener el voltaje en esas ramas, así se pudo deducir la corriente en la rama del resistor de 50 $\Omega$ y el circuito equivalente, posterirmente se usaron las leyes de Kirchhoff para deducir los voltajes y las corrientes respectivamente.

%\begin{table}[h!]
%\centering

%\begin{tabular}{|c|c|c|}
%\hline
%Componente & V {[}V{]} & I {[}A{]} \\ \hline
%R 50       & 26.25     & 0.525     \\ \hline
%R 10       & 1.75      & 0.173     \\ \hline
%R 5        & 1.75      & 0.352     \\ \hline
%V fuente   & 28        & 0.525     \\ \hline
%\end{tabular}
%\caption{Tabla de Resultados}
%\label{tabla-resultados}

%\end{table}

%.\\[100cm]
\bibliographystyle{plain}
\bibliography{Referencias.bib}


\end{document}