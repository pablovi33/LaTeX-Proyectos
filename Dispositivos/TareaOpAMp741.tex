\documentclass{article}
\usepackage[utf8]{inputenc}
\usepackage[spanish.mexico]{babel}
\usepackage[american voltages, american currents,siunitx]{circuitikz}

%para fotos

\usepackage{graphicx}
\usepackage{subcaption}

\title{Amplificador Operacional 741}
\author{Pablo Vivar Colina A7\\
Grupo 5\\
Tarea 9
}
%\date{Septiembre 2017}

\usepackage{natbib}
\usepackage{graphicx}

\begin{document}

\maketitle

\section{Amplificador Operacional}

Un amplificador operacional, a menudo conocido op-amp por sus siglas en inglés (operational amplifier) es un dispositivo amplificador electrónico de alta ganancia acoplado en corriente continua que tiene dos entradas y una salida. En esta configuración, la salida del dispositivo es, generalmente, de cientos de miles de veces mayor que la diferencia de potencial entre sus entradas.\citep{AmplificadorOperacional}


\subsection{Comparador}


Aplicación sin retroalimentación que compara señales entre las dos entradas y presenta una salida en función de qué entrada sea mayor. Se puede usar para adaptar niveles lógicos.\citep{AmplificadorOperacional}

\begin{equation}
    {\displaystyle V_{\rm {out}}=\left\{{\begin{matrix}V_{S+}&V_{1}>V_{2}\\V_{S-}&V_{1}<V_{2}\end{matrix}}\right.} 
\end{equation}

Una aplicación simple pero útil, es la de proporcionar un sistema de control ON-OFF. Por ejemplo un control de temperatura, cuya entrada no inversora se conecta un termistor (sensor de temperatura) y en la entrada inversora un divisor resistivo con un preset (resistencia variables) para ajustar el valor de tensión de referencia. Cuando en la pata no inversora exista una tensión mayor a la tensión de referencia, la salida activara alguna señalización o un actuador.\citep{AmplificadorOperacional}

\subsection{Seguidor de Voltaje o tensión}

Es aquel circuito que proporciona a la salida la misma tensión que a la entrada. Presenta la ventaja de que la impedancia de entrada es elevada, la de salida prácticamente nula, y es útil como un buffer, para eliminar efectos de carga o para adaptar impedancias (conectar un dispositivo con gran impedancia a otro con baja impedancia y viceversa) y realizar mediciones de tensión de un sensor con una intensidad muy pequeña que no afecte sensiblemente a la medición.\citep{AmplificadorOperacional}

\begin{figure}[h!]
    \centering
    \includegraphics[width=0.5\textwidth]{Imagenes/Buffer.png}
    \caption{Amplificador operacional en modo seguidor de tensión \citep{AmplificadorOperacional}}
    \label{fig:buffer}
\end{figure}

\subsection{Amplificador no inversor}

En el modo amplificador no inversor, el voltaje de salida cambia en la misma dirección del voltaje de entrada.\citep{AmplificadorOperacional}

La ecuación de ganancia para esta configuración es:

\begin{equation}
    {\displaystyle V_{\text{out}}=A_{OL}\,(V_{\!+}-V_{\!-})} 
\end{equation}

Sin embargo, en este circuito $V−$ es una función de $V_{out}$ debido a la realimentación negativa a través de la red constituida por $R1$ y $R2$, donde $R1$ y 
$R2$ forman un divisor de tensión, y como V− es una entrada de alta impedancia, no hay efecto de carga. Por consiguiente:

\begin{equation}
    {\displaystyle V_{\!-}\,\,=\beta \cdot V_{\text{out}}} 
\end{equation}


Donde:

\begin{equation}
{\displaystyle \beta ={\frac {R_{1}}{R_{1}+R_{2}}}}
\end{equation}

Sustituyendo esto en la ecuación de ganancia, se obtiene:


\begin{equation}
{\displaystyle V_{\text{out}}=A_{OL}(V_{\text{in}}-\beta \cdot V_{\text{out}})}
\end{equation}


Resolviendo para ${\displaystyle V_{\text{out}}} {\displaystyle V_{\text{out}}}:$

\begin{equation}
{\displaystyle V_{\text{out}}=V_{\text{in}}\left({\frac {1}{\beta +1/A_{OL}}}\right)} 
\end{equation}


Si ${\displaystyle A_{OL}} {\displaystyle A_{OL}}$ es muy grande, se simplifica a



\begin{equation}
{\displaystyle V_{\text{out}}\approx {\frac {V_{\text{in}}}{\beta }}={\frac {V_{\text{in}}}{\frac {R_{\text{1}}}{R_{\text{1}}+R_{\text{2}}}}}=V_{\text{in}}\left(1+{\frac {R_{2}}{R_{1}}}\right)}
\end{equation}

\begin{figure}[h!]
    \centering
    \includegraphics[width=0.5\textwidth]{Imagenes/Opampintegrating.png}
    \caption{Amplificador operacional en modo no inversor\citep{AmplificadorOperacional}}
    \label{fig:OpAmpInvert}
\end{figure}

\subsection{Sumador inversor}

Aplicación en la cual la salida es de polaridad opuesta a la suma de las señales de entrada.\citep{AmplificadorOperacional}\\

Para resistencias independientes $R1, R2,... Rn
$

\begin{equation}
{\displaystyle V_{\mathrm {out} }=-R_{f}\left({\frac {V_{1}}{R_{1}}}+{\frac {V_{2}}{R_{2}}}+\dots +{\frac {V_{n}}{R_{n}}}\right)}
\end{equation}

La expresión se simplifica bastante si se usan resistencias del mismo valor
Impedancias de entrada: $Zn = Rn$\citep{AmplificadorOperacional}\\

\begin{figure}
    \centering
    \includegraphics[width=0.5\textwidth]{Imagenes/Opampsumming.png}
    \caption{Amplificador sumador de n entradas.\citep{AmplificadorOperacional}}
    \label{fig:opAmpSumming}
\end{figure}


\subsection{Restador Inversor}

Para resistencias independientes R1,R2,R3,R4 la salida se expresa como:\citep{AmplificadorOperacional}

\begin{equation}[h!]
    {\displaystyle V_{\mathrm {out} }=V_{2}\left({\left(R_{3}+R_{1}\right)R_{4} \over \left(R_{4}+R_{2}\right)R_{1}}\right)-V_{1}\left({R_{3} \over R_{1}}\right)}
\end{equation}

La impedancia diferencial entre dos entradas es:

\begin{equation}
{\displaystyle Z_{\rm {in}}=R_{1}+R_{2}+R_{\rm {in}}}
\end{equation}

donde ${\displaystyle R_{in}}$ ${\displaystyle R_{in}}$ representa la resistencia de entrada diferencial del amplificador, ignorando las resistencias de entrada del amplificador de modo común. Este tipo de configuración tiene una resistencia de entrada baja en comparación con otro tipo de restadores como el amplificador de instrumentación.\citep{AmplificadorOperacional}

\begin{figure}[h!]
    \centering
    \includegraphics[width=0.5\textwidth]{Imagenes/Opampdifferencing.png}
    \caption{Amplificador restador-inversor\citep{AmplificadorOperacional}}
    \label{fig:my_label}
\end{figure}

\subsection{Integrador ideal}

Este montaje integra e invierte la señal de entrada ${\displaystyle V_{in}} {\displaystyle V_{in}}$ produciendo como salida:

\begin{equation}
    {\displaystyle V_{\rm {out}}=\int _{0}^{t}-{V_{\rm {in}} \over RC}\,dt+V_{\rm {inicial}}} 
\end{equation}

En esta ecuación ${\displaystyle V_{\rm {inicial}}}$  es la tensión de origen al iniciarse el funcionamiento.\citep{AmplificadorOperacional}\\

Este integrador no se usa en la práctica de forma discreta ya que cualquier señal pequeña de corriente directa en la entrada puede ser acumulada en el condensador hasta saturarlo por completo; sin mencionar la característica de desplazamiento de tensión del amplificador operacional, que también es acumulada. Este circuito se usa de forma combinada en sistemas retroalimentados que son modelos basados en variables de estado (valores que definen el estado actual del sistema) donde el integrador conserva una variable de estado en el voltaje de su condensador.\citep{AmplificadorOperacional}

\begin{figure}[h!]
    \centering
    \includegraphics[width=0.5\textwidth]{Imagenes/Opampintegrating.png}
    \caption{Amplificador integrador.\citep{AmplificadorOperacional}}
    \label{fig:opAmpIntegrador}
\end{figure}

%\begin{figure}
 %   \centering
  %  \includegraphics{Imagenes/OpAmpLazoAbierto.png}
   
   % \caption{Amplificador operacional en modo de lazo abierto, configuración usada como comparador.}
    %\label{fig:opAmpLazoAbierto}
%\end{figure}

\subsection{Derivador ideal}

Este circuito deriva e invierte la señal de entrada, produciéndose como salida:\citep{AmplificadorOperacional}

\begin{equation}
    {\displaystyle V_{\rm {out}}=-RC\,{dV_{\rm {in}} \over dt}} 
\end{equation}

Además de lo anterior, este circuito también se usa como filtro, sin embargo no es estable. Esto se debe a que al amplificar más las señales de alta frecuencia, se termina amplificando mucho el ruido.\citep{AmplificadorOperacional}

\begin{figure}[h!]
    \centering
    \includegraphics[width=0.5\textwidth]{Imagenes/Opampdifferentiating.png}
    \caption{Amplificador derivador}
    \label{fig:OpAmpDerivador}
\end{figure}

\subsection{Amplificador Ideal}

\begin{figure}
    \centering
    \includegraphics[width=0.5\textwidth]{Imagenes/OpAMpIdeal.png}
    \caption{Circuito equivalente de un amplificador operacional\citep{AmplificadorOperacional}}
    \label{fig:opAmpIdeal}
\end{figure}

\section{Filtro Digital}

Un filtro digital es un tipo de filtro que opera sobre señales discretas y cuantizadas, implementado con tecnología digital, bien como un circuito digital o como un programa informático.\citep{FiltroDigital}\\

\subsection{Filtro paso alto}

Un filtro paso alto (HPF) es un tipo de filtro electrónico en cuya respuesta en frecuencia se atenúan los componentes de baja frecuencia pero no los de alta frecuencia, éstas incluso pueden amplificarse en los filtros activos. La alta o baja frecuencia es un término relativo que dependerá del diseño y de la aplicación. En particular la función de transferencia de un filtro pasa alta de primer orden corresponde a:\citep{FiltroPasoAlto}

\begin{equation}
    {\displaystyle H(s)={{\frac {s}{w_{c}}} \over 1+{\frac {s}{w_{c}}}}} {\displaystyle H(s)={{\frac {s}{w_{c}}} \over 1+{\frac {s}{w_{c}}}}}
\end{equation}

\subsubsection{Implementación}

El filtro paso alto es un circuito RC en serie en el cual la salida es la caída de tensión en la resistencia.\citep{FiltroPasoAlto}\\

Si se estudia este circuito con componentes ideales para frecuencias muy bajas -continua por ejemplo- se tiene que el condensador se comporta como un circuito abierto, por lo que no dejará pasar la corriente a la resistencia, y su diferencia de tensión será cero. Para una frecuencia muy alta, idealmente infinita, el condensador se comportará como un circuito cerrado, es decir, como si no estuviera, por lo que la caída de tensión de la resistencia será la misma tensión de entrada, lo que significa que dejaría pasar toda la señal. Por otra parte, el desfase entre la señal de entrada y la de salida si que varía, como puede verse en la imagen.\citep{FiltroPasoAlto}\\

El producto de resistencia por condensador $(RxC)$ es la constante de tiempo, cuyo recíproco es la frecuencia de corte, es decir, donde el módulo de la respuesta en frecuencia baja 3dB respecto a la zona pasante:\citep{FiltroPasoAlto}\\

\begin{equation}
    {\displaystyle f_{c}={1 \over 2\pi RC}}
\end{equation}


Donde fc es la frecuencia de corte en Hertzs, R es la resistencia de la aplicación en ohm y C es la capacidad en farads.\citep{FiltroPasoAlto}\\

El desfase depende de la frecuencia f de la señal y sería:\citep{FiltroPasoAlto}\\

\begin{equation}
    {\displaystyle \theta \ =\tan ^{-1}{\frac {f_{c}}{f}}} {\displaystyle \theta \ =\tan ^{-1}{\frac {f_{c}}{f}}}
\end{equation}

\begin{figure}[h!]
    \centering
    \begin{subfigure}[b]{0.45\textwidth}
        \includegraphics[width=\textwidth]{Imagenes/circuitoPasoAlto.png}
        \caption{Filtro pasivo analógico de primer orden con circuito RC.}
        \label{fig:circuitoPasoAlto}
    \end{subfigure}
    ~ %add desired spacing between images, e. g. ~, \quad, \qquad, \hfill etc. 
      %(or a blank line to force the subfigure onto a new line)
    \begin{subfigure}[b]{0.45\textwidth}
        \includegraphics[width=\textwidth]{Imagenes/ondaFiltroPasoALto.png}
        \caption{Comportamiento de onda con filtro paso alto}
        \label{fig:ondaPasoAlto}
    \end{subfigure}
   
    \caption{Filtro Paso Alto}\label{fig:filtroPasoAlto}
\end{figure}

\subsection{Filtro Paso Bajo}

Un filtro paso bajo corresponde a un filtro electrónico caracterizado por permitir el paso de las frecuencias más bajas y atenuar las frecuencias más altas. El filtro requiere de dos terminales de entrada y dos de salida, de una caja negra, también denominada cuadripolo o bipuerto, así todas las frecuencias se pueden presentar a la entrada, pero a la salida solo estarán presentes las que permita pasar el filtro. De la teoría se obtiene que los filtros están caracterizados por sus funciones de transferencia, así cualquier configuración de elementos activos o pasivos que consigan cierta función de transferencia serán considerados un filtro de cierto tipo.\citep{FiltroPasoBajo}\\

En particular la función de transferencia de un filtro paso bajo de primer orden corresponde a ${\displaystyle H(s)=k{\frac {1}{1+{\frac {s}{\omega _{c}}}}}\,\!} {\displaystyle H(s)=k{\frac {1}{1+{\frac {s}{\omega _{c}}}}}\,\!}$, donde la constante ${\displaystyle k\,\!} {\displaystyle k\,\!}$ es sólo una ponderación correspondiente a la ganancia del filtro, y la real importancia reside en la forma de la función de transferencia ${\displaystyle {\frac {1}{1+{\frac {s}{\omega _{c}}}}}\,\!} {\displaystyle {\frac {1}{1+{\frac {s}{\omega _{c}}}}}\,\!}$, la cual determina el comportamiento del filtro. En la función de transferencia anterior ${\displaystyle \omega _{c}\,\!} {\displaystyle \omega _{c}\,\!}$ corresponde a la frecuencia de corte propia del filtro, aquel valor de frecuencia para la cual la relación entre la señal de salida y la señal de entrada es exactamente ${\displaystyle {\frac {\sqrt {2}}{2}}\,\!} {\displaystyle {\frac {\sqrt {2}}{2}}\,\!}$, relación que se puede aproximar a ${\displaystyle -3\ dB\,\!} {\displaystyle -3\ dB\,\!}$.\citep{FiltroPasoBajo}\\

De forma análoga al caso de primer orden, los filtros de pasa bajo de mayor orden también se caracterízan por su función de transferencia, por ejemplo la función de transferencia de un filtro paso bajo de segundo orden corresponde a ${\displaystyle H(s)=k{\frac {\omega _{o}^{2}}{s^{2}+2\xi \omega _{o}s+\omega _{o}^{2}}}\,\!} {\displaystyle H(s)=k{\frac {\omega _{o}^{2}}{s^{2}+2\xi \omega _{o}s+\omega _{o}^{2}}}\,\!}, donde {\displaystyle \omega _{o}\,\!} {\displaystyle \omega _{o}\,\!}$ es la frecuencia natural del filtro y ${\displaystyle \xi \,\!} {\displaystyle \xi \,\!}$ es el factor de amortiguamiento de este.\citep{FiltroPasoBajo}\\


\subsection{Filtro Paso Banda}

Un filtro paso banda es un tipo de filtro electrónico que deja pasar un determinado rango de frecuencias de una señal y atenúa el paso del resto.\citep{FiltroPasoBanda}


\begin{figure}[h!]
    \centering
    \includegraphics{Imagenes/PasoBanda.png}
    \caption{Respuesta frecuencial de un filtro paso banda\citep{FiltroPasoBanda}}
    \label{fig:filtroPasoBanda}
\end{figure}


%\section{Circuito a resolver}

%\begin{figure}[h!]
 %   \centering
   % \includegraphics{}
  %  \begin{circuitikz}
%\draw

%(-1,0)--(-1,-1)
%(-1,-1) to[V,l=$28v$](-1,-2) 
% (-1,-2)--(-1,-3)
 
 %(-1,-3)--(2,-3)
 
% (-1,0) to[R,l=$50 \Omega $](2,0)
 
 
% (2,-3)to[R,l=$5 \Omega$](2,0)
 
% (2,0)--(4,0)
% (2,-3)--(4,-3)
 
 % (4,-3)to[R,l=$10 \Omega$](4,0)
 
 
%;

%\draw[thick,arrows=->]
%(2.5,0)--(2.5,-1)
%;
 
%\end{circuitikz}
 %   \caption{Circuito a resolver}
  %  \label{fig:circuito}
%\end{figure}

%\subsection{Resultados}

%Para resolver el circuito se hicieron equivalente los resistores de 5 y 10 $\Omegas$ para obtener el voltaje en esas ramas, así se pudo deducir la corriente en la rama del resistor de 50 $\Omega$ y el circuito equivalente, posterirmente se usaron las leyes de Kirchhoff para deducir los voltajes y las corrientes respectivamente.

%\begin{table}[h!]
%\centering

%\begin{tabular}{|c|c|c|}
%\hline
%Componente & V {[}V{]} & I {[}A{]} \\ \hline
%R 50       & 26.25     & 0.525     \\ \hline
%R 10       & 1.75      & 0.173     \\ \hline
%R 5        & 1.75      & 0.352     \\ \hline
%V fuente   & 28        & 0.525     \\ \hline
%\end{tabular}
%\caption{Tabla de Resultados}
%\label{tabla-resultados}

%\end{table}

%.\\[100cm]
\bibliographystyle{plain}
\bibliography{Referencias.bib}


\end{document}