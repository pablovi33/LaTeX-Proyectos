\documentclass{article}
\usepackage[utf8]{inputenc}
\usepackage[spanish.mexico]{babel}

\title{Previo 14}
\author{Pablo Vivar Colina}
\date{Septiembre 2017}

\usepackage{natbib}
\usepackage{graphicx}

%Circuitos
\usepackage{tikz}

\usepackage[american voltages, american currents,siunitx]{circuitikz}

\begin{document}

\maketitle

\begin{figure}[h!]
    \centering
    \begin{circuitikz}
    
\draw
(-0.85,2)--(-0.85,3)
(-0.85,2)to[R,l=$12k\Omega$](-0.85,0)

%Capacitor de salida
(0,0.3)to[C,l=$10\mu F$](2,0.3)


(-0.85,-2)to[R,l=$2k\Omega$](-0.85,0)
(-0.85,-2)--(-0.85,-3)


(0,0.75)to[R,l=$3.3k\Omega$](0,3)
(0,-0.75)to[R,l=$1k\Omega$](0,-3)

%12V
(-0.425,3)--(-0.425,3.5)

%lineas horizontales
(-0.85,3)--(0,3)
(-0.85,-3)--(0,-3)

%tierra resistencias (2k y 1k)
(-0.425,-3)  to  (-0.425,-3.5) node[ground]{}

%Capacitor de entrada
(-3,0)to[C,l=$10\mu F$](-0.8,0)

%Capacitor de entrada
(-3,0)to[sV,l=$S$](-3,-2)

%tierra fuente
(-3,-2)  to (-3,-2.5) node[ground]{}



;
    
    
    \draw
    %NPN
    (0,0)
node [npn,xscale=1] (npn) {} 
(npn.collector) ;
    
    \node[draw] at (-0.5,3.8) {12[V]};
   

        
       
    \end{circuitikz}
    \caption{Circuito operación experimento TBJ}
    \label{fig:circuito TBJ A}
\end{figure}

En el circuito de la figura \ref{fig:circuito TBJ A} se midieron los voltajes de entrada y de salida del circuito, características del circuito.\\

\begin{itemize}
    \item $V_{ent}$ = 3.98 $[V]$
    \item $V_{sal}$ = 1.23 $[V]$
    \item Ganancia = 6.73 
    \item Ánglulo de defasamiento 
\end{itemize}


\section{Experimento}

\begin{figure}[h!]
    \centering
    \begin{circuitikz}
    
\draw
(-0.85,2)--(-0.85,3)
(-0.85,2)to[R,l=$12k\Omega$](-0.85,0)

%Capacitor de salida
(0,-0.3)to[C,l=$10\mu F$](2,-0.3)


(-0.85,-2)to[R,l=$2k\Omega$](-0.85,0)
(-0.85,-2)--(-0.85,-3)


(0,0.75)to[R,l=$3.3k\Omega$](0,3)
(0,-0.75)to[R,l=$1k\Omega$](0,-3)

%12V
(-0.425,3)--(-0.425,3.5)

%lineas horizontales
(-0.85,3)--(0,3)
(-0.85,-3)--(0,-3)

%tierra resitencias (2k y 1k)
(-0.425,-3)  to  (-0.425,-3.5) node[ground]{}

%Capacitor de entrada
(-3,0)to[C,l=$10\mu F$](-0.8,0)

%Fuente
(-3,0)to[sV,l=$S$](-3,-2)

%tierra fuente
(-3,-2)  to (-3,-2.5) node[ground]{}

;
    
    
    \draw
    %NPN
    (0,0)
node [npn,xscale=1] (npn) {} 
(npn.collector) ;
    
    \node[draw] at (-0.5,3.8) {12[V]};
   

        
       
    \end{circuitikz}
    \caption{Circuito operación experimento TBJ}
    \label{fig:circuito TBJ B}
\end{figure}

En el circuito de la figura \ref{fig:circuito TBJ B} se midieron los voltajes de entrada y de salida del circuito, características del circuito.\\

\begin{itemize}
    \item $V_{ent}$ = 3.98 $[V]$
    \item $V_{sal}$ = 1.23 $[V]$
    \item Ganancia = 1
    \item Ánglulo de defasamiento 
\end{itemize}


De los circuitos expuestos en las figuras \ref{fig:circuito TBJ A} y \ref{fig:circuito TBJ B} decir ¿cuál es la diferencia de cada circuito?

%\section{¿Cómo se polariza el T.B.J.?}






\end{document}
