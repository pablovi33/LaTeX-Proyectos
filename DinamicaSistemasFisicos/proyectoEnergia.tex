\documentclass[]{article}
\usepackage[spanish.mexico]{babel}
\usepackage[T1]{fontenc}
\usepackage[utf8]{inputenc}
\usepackage{lmodern}
\usepackage[a4paper]{geometry}

%\usepackage{natbib}
\usepackage{cite}


%Grafico de barras
\usepackage{pgfplots}


\title{Proyecto de Optimización de Energía}
\author{Pablo Vivar Colina}
\date{Mayo 2018}

\begin{document}

\maketitle

\section{Introducción}

En física, energía se define como la capacidad para realizar un trabajo.1​ En tecnología y economía, \textit{energía} se refiere a un recurso natural (incluyendo a su tecnología asociada) para poder extraerla, transformarla y darle un uso industrial o económico.\cite{EnergiaWiki}\\



\section{Dinámica de sistemas físicos}


\begin{figure}[h!]
	\begin{tabular}[]{|c|}
		\hline
		hola \\
		\hline
	\end{tabular}
\label{tabla}
\caption{Resultados}
\end{figure}


\section{Suministro de Energía
}


\begin{figure}[h!]
	\centering
	\begin{tabular}[]{|c|c|}
		\hline
		 Suministro & Precio (MXN)\\
		 	\hline
		 Básico & 0.793 \\
		 Intermedio & 0.956 \\
		\hline
	\end{tabular}
	\label{cuadro:suministros}
	\caption{Suministros}
	
\end{figure}



\section{Historial de consumo de energía}

\begin{figure}[h!]
	\centering
\begin{tikzpicture}
\begin{axis}[
symbolic x coords={A,B,C,D,E,F,G,H,I,J,K},
xtick=data
]
\addplot[ybar,fill=orange] coordinates {
	(A,   291)
	(B,  284)
	(C,   323)
	(D,323)
	(E,339)
	(F,306)
	(G,254)
	(H,282)
	(I,302)
	(J,264)
	(K,272)
};
\end{axis}
\end{tikzpicture}
\caption{Consumo energía eléctrica [kWh]}
\end{figure}

\begin{figure}[h!]
	\centering
	\begin{tabular}[]{|c|c|c|}
		\hline
		Nombre & Periodo & Energía consumida [kWh]\\
		\hline
	I & 11/8/17-12/10/17 & 323 \\
	
	J & 12/10/17-11/10/17 & 284 \\
		K & 11/12/17-12/2/18 & 291 \\
	
		\hline
	\end{tabular}
	\label{cuadro:ultimosPeriodos}
	\caption{Últimos periodos}
	
\end{figure}

\bibliographystyle{plain}
\bibliography{Referencias.bib}
%\addbibresource{Referencias.bib}

\end{document}
