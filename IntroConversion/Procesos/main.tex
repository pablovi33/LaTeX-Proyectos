\documentclass[]{article}
\usepackage[spanish.mexico]{babel}
\usepackage[T1]{fontenc}
\usepackage[utf8]{inputenc}
%\usepackageB.\\{lmodern}
\usepackage[a4paper]{geometry}

%\usepackage{natbib}
\usepackage{cite}


%Grafico de barras
\usepackage{pgfplots}

%Graficos e imagenes
\usepackage{graphicx}


\title{Introducción a la conversión de Energía}
\author{Pablo Vivar Colina}
%\date{Mayo 2018}



\begin{document}
	
%	%\usepackage[top=2cm,bottom=2cm,left=1cm,right=1cm]{geometry}


\begin{titlepage}
     \begin{center}
	\includegraphics[width=0.09\textwidth]{UNAM}\Large Universidad Nacional Autónoma de México
        	\includegraphics[width=0.09\textwidth]{FI}\\[1cm]
        \Large Facultad de Ingeniería\\[1cm]
       % \Large División de Ciencias Básicas\\[1cm]
         \Large Laboratorio de Dispositivos y Circuitos Electrónicos (6654)\\[1cm]
         %la clave antes era:4314
         \footnotesize Profesor: Zapata Rosales Arturo Ing.\\[1cm]
        \footnotesize Semestre 2018-1\\[1cm]
        %\Large Práctica No. 1\\[1cm]
    
        %\Large Práctica No. 2\\[1cm]
        
        %\Large Práctica No. 3\\[1cm]
       
        %\Large Práctica No. 4\\[1cm]
         
               
         %\Large Práctica No. 5\\[1cm]
         
         
         %\Large Práctica No. 6\\[1cm]
         
         %\Large Práctica No. 7\\[1cm]
         
             %\Large Práctica No. 8\\[1cm]
       

        \Large Práctica No. 9\\[1cm]
        
           %####AQUI VAMOS#### ya ahora sii
           
        %\Large Práctica No. 11\\[1cm]
        %\Large Práctica No. 12\\[1cm]
        %\Large Práctica No. 13\\[1cm]
        
        %\Large Amplificador Operacional como Integrador\\[1cm]
        %\Large{Filtros}\\[1cm]
         %\Large{Medición de  corrientes en un circuito}\\[1cm]
         %practica 4
         %Large{Amplificador operacional como seguidor de voltaje en entrada inversora}\\[1cm]
         %practica5
         %\Large{Amplificador operacional como integrador}
         
         %Practica 7
%Comportamiento de un diodo Zener

\Large Diodo Zener
        
         %Texto a la derecha
          \begin{flushright}
\footnotesize  Grupo 13\\[0.5cm]
\footnotesize Brigada: 7\\[0.5cm]

\footnotesize Vivar Colina Pablo\\[0.5cm]
 \end{flushright}
    %Texto a la izquierda
          \begin{flushleft}
        \footnotesize Ciudad Universitaria Abril de 2018.\\
          \end{flushleft}
         
          
        %\vfill
        %\today
   \end{center}
\end{titlepage}
 %agregar portada

\maketitle

%\tableofcontents  % Write out the Table of Contents

%\listoffigures  % Write out the List of Figures



\section{Energía de concepto}

\section{Ingeniería Básica}

\section{Ingeniería de detalle}

Se proporcionan las especificaciones.\\
Se arma la oferta comercial, se puedan dar cotizaciones.\\


\section{Apéndice}

Malfunción: Fallo originado por condiciones internas al sistema.\\ 

Disfunción: un fallo originado por condiciones externas al sistema.\\



\section{Referencias}

%\bibliographystyle{plain}
%\bibliography{Prac1}



\begin{thebibliography}{9}
	%\bibitem{latexcompanion} 
	%Michel Goossens, Frank Mittelbach, and Alexander Samarin. 
	%\textit{The \LaTeX\ Companion}. 
	%Addison-Wesley, Reading, Massachusetts, 1993.
	
	%\bibitem{einstein} 
	%Albert Einstein. 
	%\textit{Zur Elektrodynamik bewegter K{\"o}rper}. (German) 
	%[\textit{On the electrodynamics of moving bodies}]. 
	%Annalen der Physik, 322(10):891–921, 1905.
	
	
	%\bibitem{ebullicion} 
	 %Punto ebullición,
	%\\\texttt{http://www.cie.unam.mx/~ojs/pub/Liquid3/node8.html}
	
	
	%\bibitem{coccion} 
   %UnComo:Tiempo de cocción de los frijoles,
	%\\\texttt{https://comida.uncomo.com/articulo/tiempo-de-coccion-de-los-frijoles-34777.html}
\end{thebibliography}






\end{document}
