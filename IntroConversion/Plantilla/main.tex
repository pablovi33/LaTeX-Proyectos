\documentclass[]{article}
\usepackage[spanish.mexico]{babel}
\usepackage[T1]{fontenc}
\usepackage[utf8]{inputenc}
%\usepackageB.\\{lmodern}
\usepackage[a4paper]{geometry}

%\usepackage{natbib}
\usepackage{cite}


%Grafico de barras
\usepackage{pgfplots}

%Graficos e imagenes
\usepackage{graphicx}


\title{Introducción a la conversión de Energía}
\author{Pablo Vivar Colina}
%\date{Mayo 2018}



\begin{document}
	
%	%\usepackage[top=2cm,bottom=2cm,left=1cm,right=1cm]{geometry}


\begin{titlepage}
     \begin{center}
	\includegraphics[width=0.09\textwidth]{UNAM}\Large Universidad Nacional Autónoma de México
        	\includegraphics[width=0.09\textwidth]{FI}\\[1cm]
        \Large Facultad de Ingeniería\\[1cm]
       % \Large División de Ciencias Básicas\\[1cm]
         \Large Laboratorio de Dispositivos y Circuitos Electrónicos (6654)\\[1cm]
         %la clave antes era:4314
         \footnotesize Profesor: Zapata Rosales Arturo Ing.\\[1cm]
        \footnotesize Semestre 2018-1\\[1cm]
        %\Large Práctica No. 1\\[1cm]
    
        %\Large Práctica No. 2\\[1cm]
        
        %\Large Práctica No. 3\\[1cm]
       
        %\Large Práctica No. 4\\[1cm]
         
               
         %\Large Práctica No. 5\\[1cm]
         
         
         %\Large Práctica No. 6\\[1cm]
         
         %\Large Práctica No. 7\\[1cm]
         
             %\Large Práctica No. 8\\[1cm]
       

        \Large Práctica No. 9\\[1cm]
        
           %####AQUI VAMOS#### ya ahora sii
           
        %\Large Práctica No. 11\\[1cm]
        %\Large Práctica No. 12\\[1cm]
        %\Large Práctica No. 13\\[1cm]
        
        %\Large Amplificador Operacional como Integrador\\[1cm]
        %\Large{Filtros}\\[1cm]
         %\Large{Medición de  corrientes en un circuito}\\[1cm]
         %practica 4
         %Large{Amplificador operacional como seguidor de voltaje en entrada inversora}\\[1cm]
         %practica5
         %\Large{Amplificador operacional como integrador}
         
         %Practica 7
%Comportamiento de un diodo Zener

\Large Diodo Zener
        
         %Texto a la derecha
          \begin{flushright}
\footnotesize  Grupo 13\\[0.5cm]
\footnotesize Brigada: 7\\[0.5cm]

\footnotesize Vivar Colina Pablo\\[0.5cm]
 \end{flushright}
    %Texto a la izquierda
          \begin{flushleft}
        \footnotesize Ciudad Universitaria Abril de 2018.\\
          \end{flushleft}
         
          
        %\vfill
        %\today
   \end{center}
\end{titlepage}
 %agregar portada

\maketitle

%\tableofcontents  % Write out the Table of Contents

%\listoffigures  % Write out the List of Figures



\section{La energía eléctrica}

\subsection{Tipos de cargas}

\subsubsection{Desde el punto de vista eléctrico-técnico las cargas pueden ser}

\begin{itemize}
	\item Resistivas
	\item Capacitivas
	\item Inductivas
	\item Mixtas: resistiva vapacitiva, resistiva inductiva, resistiva, capacitiva e inductiva.
\end{itemize}


\subsubsection{De acuerdo la clasificación técnica económica de cada usuario}

\begin{itemize}
	\item Cargas no críticas
	\item Cargas sensibles
	\item Cargas críticas
	
\end{itemize}

\subsubsection{De acuerdo con la relación Voltaje-Corriente}

\begin{itemize}
	\item Cargas de impedancia constante
	\item Cargas de potencia contante
\end{itemize}

\subsubsection{De acuerdo con el comportamiento de la forma de onda de la corriente con respecto de la forma de onda de la tensión}

\begin{itemize}
	\item Cargas lineales
	\item Cargas no lineales
\end{itemize}



\subsection{Cargas Resistivas}

Convierte la energía eléctrica en calorífica, comúnmente una resistencia donde la potencia de la tensión.\\

\subsection{Cargas Capacitivas}

La potencia que toma la carga en el primer medio ciclo de la fuente de corriente alterna la convierte en camo eléctrico., que en el siguiente medio ciclo regresa la potencia a la fuente. Es decir, que el capacitor se carga y se descarga (toma la potencia de la fuente, la usa y la regresa, pero no la consume).\\


En éste caso no se habla de una resitencia (R), sino de una reactancia.\\

\subsection{Cargas Inductivas}

La potencia, que toma la carga en el primer medio\\

\subsection{Carga Mixta}

Las cargas pueden ser formadas por la combinación de elementos resistivos capacitivos en inductivos.\\

\subsection{Tipos de cargas en base a clasificación técnica-económica} 

\subsubsection{Cargas Sensibles} 

Son aquellas que requieren de un suministro de alta calidad, esto es, libre de variaciones de tensión o frecuencia. Los equipos electrónicos son más susceptibles a estos disturbios.\\

\subsubsection{Cargas críticas} 

Son aquellas que al dejar de funcionar ponen en peligro la vida humana, la seguridad del personal y/o ocasiona grandes perjuicos económicos.\\

Ejemplos: Salas de cirugía, cuidados intensivos, centros de datos y control, telecomunicaciones vitales, sistemas de segurdad pública y provada.\\

\subsection{Cargas de impedancia constante}

En este tipo de cargasd la relacion V/i se mantiene constante, de tal dorma que si varía la tensipon, también lo harán la corriente y la potencia de la misma proporción

(VA o W)

%Las variaciones de tensión de C.A. No afectan el consumo de potencia final en carga, formando así " cargas no críticas" o "cargas de potencia constante" (CPL), a diferencia de las cargas de impedancia constante (CZL) , las CPL tiene características distintas, donde el valor promedio de impediancia es positivo $(V/I>0)$, pero el incremento de impedancia siempre es negativo $\frac{\Delta V}{\DeltaI} < 0$.\\

Un incremento de voltaje a través de CPL resulta en en decremento de la corriente mientras in decremento en el voltaje resulta en in incremento en la corriente mientras un decremento en el voltaje resulta en un incremento en la corriente. En este escenario las CPL generan un efecto en la impedancia negativa.\\

Por otra parte, cuando efecto de Resistencia negativa tiende a predominar en la red, se puede llevar a operar la red de destribución en zonas de inestabilidad provocando oscilaciones de potencia o incluso colapsos de voltaje.\\

\subsection{Cargas lineales}

Cualquier carda conectada a una fuente de tensión alterna.\\

\subsection{Cargas no lineales}

Si entre la fuente de tensión alterna y la carga se interpone un dispositivo eléctrixo (diodo, SCR, TRIAC) que controle el paso de corriente originaria y ésta no tuviera la misma forma de onda que la tensión y, además no se garantiza que la relación, 
para cualquier valor instantáneo (Vi/Ii), fuera constante, en el ciclo positivo, en el caso de tener un diodo, la tensión y la corriente tendrán la misma forma de onda,
pero en la parte negativa habrá tensión y no corriente.

Se produce una deformaciónd de la onda senoidal original.

TEOREMAS DE FOURIER*



\section{Calidad de la energía}

La calidad de la energía se entiende cuando la energía eléctrica es suministrada a los equipos y dispositivos con las características y condiciones adecuadas que les permita mantener su continuidad sin que afecte su desempeño ni provoque alteraciones .\\

\section{Efectos de Cargas No Lineales}

\begin{itemize}
	\item Efectos térmicos.
	\item Oscilaciones torsionales en máquinas eléctricas.
	\item Generan corrientes por los ductos.
	\item Oscilaciones de baja frecuencia en sistemas mecánicos.
	\item Interferencias en señales de control y protección en líneas eléctricas.
	 \item Sobrecargas en filtros paralelos aprmónicos de alot orden.
	 \item Perturbaciones acústicas.
	 \item Saturación en transformadores de corrientes.
\end{itemize}


\section{Efectos económicos de las Interrupciones}

A través del tiempo se han propuesto varias soluciones tecnológicas para mitigar los efectos de la mala calidad de la energía, algunos de éstos están basados en componentes pasivos o con el uso de dispositivos semiconducotres

Tap, banco de capacitores.



\begin{itemize}
	\item Tap
	\item banco de capacitores
	\item DVR
	\item D-STATCOM
	\item UPQC (combinación de DVR y D-STATCOM)
\end{itemize}



\section{Referencias}

%\bibliographystyle{plain}
%\bibliography{Prac1}



\begin{thebibliography}{9}
	%\bibitem{latexcompanion} 
	%Michel Goossens, Frank Mittelbach, and Alexander Samarin. 
	%\textit{The \LaTeX\ Companion}. 
	%Addison-Wesley, Reading, Massachusetts, 1993.
	
	%\bibitem{einstein} 
	%Albert Einstein. 
	%\textit{Zur Elektrodynamik bewegter K{\"o}rper}. (German) 
	%[\textit{On the electrodynamics of moving bodies}]. 
	%Annalen der Physik, 322(10):891–921, 1905.
	
	
	%\bibitem{ebullicion} 
	 %Punto ebullición,
	%\\\texttt{http://www.cie.unam.mx/~ojs/pub/Liquid3/node8.html}
	
	
	%\bibitem{coccion} 
   %UnComo:Tiempo de cocción de los frijoles,
	%\\\texttt{https://comida.uncomo.com/articulo/tiempo-de-coccion-de-los-frijoles-34777.html}
\end{thebibliography}






\end{document}
