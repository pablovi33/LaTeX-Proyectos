\documentclass[]{article}
\usepackage[spanish.mexico]{babel}
\usepackage[T1]{fontenc}
\usepackage[utf8]{inputenc}
%\usepackage{lmodern}
\usepackage[a4paper]{geometry}

%Plotting

\usepackage{pgfplots}
\pgfplotsset{width=10cm,compat=1.9} 
%\usepgfplotslibrary{external}
%\tikzexternalize 

%Graficos e imagenes
\usepackage{graphicx}
%\graphicspath{ Imagenes/ }

\usepackage{natbib}
%\usepackage{cite}

\usepackage{subcaption}

%Grafico de barras
%\usepackage{pgfplots}


\usepackage{tikz}
\usepackage[american voltages, american currents,siunitx]{circuitikz}

\title{Tarea 1 Capítulo 2}
\author{Pablo Vivar Colina}
%\date{Mayo 2018}


\begin{document}
	
%%\usepackage[top=2cm,bottom=2cm,left=1cm,right=1cm]{geometry}


\begin{titlepage}
     \begin{center}
	\includegraphics[width=0.09\textwidth]{UNAM}\Large Universidad Nacional Autónoma de México
        	\includegraphics[width=0.09\textwidth]{FI}\\[1cm]
        \Large Facultad de Ingeniería\\[1cm]
       % \Large División de Ciencias Básicas\\[1cm]
         \Large Laboratorio de Dispositivos y Circuitos Electrónicos (6654)\\[1cm]
         %la clave antes era:4314
         \footnotesize Profesor: Zapata Rosales Arturo Ing.\\[1cm]
        \footnotesize Semestre 2018-1\\[1cm]
        %\Large Práctica No. 1\\[1cm]
    
        %\Large Práctica No. 2\\[1cm]
        
        %\Large Práctica No. 3\\[1cm]
       
        %\Large Práctica No. 4\\[1cm]
         
               
         %\Large Práctica No. 5\\[1cm]
         
         
         %\Large Práctica No. 6\\[1cm]
         
         %\Large Práctica No. 7\\[1cm]
         
             %\Large Práctica No. 8\\[1cm]
       

        \Large Práctica No. 9\\[1cm]
        
           %####AQUI VAMOS#### ya ahora sii
           
        %\Large Práctica No. 11\\[1cm]
        %\Large Práctica No. 12\\[1cm]
        %\Large Práctica No. 13\\[1cm]
        
        %\Large Amplificador Operacional como Integrador\\[1cm]
        %\Large{Filtros}\\[1cm]
         %\Large{Medición de  corrientes en un circuito}\\[1cm]
         %practica 4
         %Large{Amplificador operacional como seguidor de voltaje en entrada inversora}\\[1cm]
         %practica5
         %\Large{Amplificador operacional como integrador}
         
         %Practica 7
%Comportamiento de un diodo Zener

\Large Diodo Zener
        
         %Texto a la derecha
          \begin{flushright}
\footnotesize  Grupo 13\\[0.5cm]
\footnotesize Brigada: 7\\[0.5cm]

\footnotesize Vivar Colina Pablo\\[0.5cm]
 \end{flushright}
    %Texto a la izquierda
          \begin{flushleft}
        \footnotesize Ciudad Universitaria Abril de 2018.\\
          \end{flushleft}
         
          
        %\vfill
        %\today
   \end{center}
\end{titlepage}
 %agregar portada

\maketitle

%\tableofcontents  % Write out the Table of Contents

%\listoffigures  % Write out the List of Figures


1.- Valor 10 puntos: Cuantos neutrones y protones se encuentran en el núcleo de los siguientes átomos:
(a) 7 Li,
(b) 24 Mg,
(c) 135 Xe,
(d) 209 Bi,
(e) 222 Rn ?
2.- Valor 20 puntos: La fisión del núcleo de 235 U libera aproximadamente 200 MeV. Cuanta energía es
liberada por fisión de 1 g de 235 U ? (Calcularlo en kilowatt-hora y megawatt-día).
3.- Valor 20 puntos: a) Comparar el radio nuclear y la densidad del núcleo del 1 H y del
2.4).
235
U (ver sección
b) ¿Cuál es la proporción del espacio total del átomo de aluminio ocupádo por el núcleo (la densidad del
aluminio es 2.7 gr/cm 3 )?
4.- Valor 20 puntos: a). ¿Cuál es la longitud de onda de un electrón con 1 keV de energía cinética?
b). ¿Cuál es la longitud de onda de una pelota de golf de 50 gr. Que viaja a 50 m/s?
5.- Valor 30 puntos: Un acelerador de partículas es usado para crear rayos de electrones de alta energía.
Si esto se logra permitiendo que los electrones se aceleren a través de un potencial de 300 V:
a) Determinar la energía cinética del electrón, la energía total y la velocidad.
b) Repetir los cálculos utilizando un protón en lugar de un electrón.
c) Repetir los cálculos para el electrón y el protón si el voltaje que acelera las partículas se incrementa a 1
MV.


\section{Obejtivo del ejercicio}

Analizar, diseñar, simular y armar el inversor lógico como circuito integrado para generar su función de transferencia y su señalización mediante diodos LED.


\section{Inversor lógico}


\begin{figure}[h!]
	\centering
%	\includegraphics[width=0.2\textwidth]{Puertas_NOT_con_transistores}
	\caption{Compuerta NOT con transistores}
\end{figure}


Una variable lógica. Una variable lógica (A) a la cual se le aplica la negación se pronuncia como "no A" o "A negada".
Puerta NOT con transistores. Se puede definir como una puerta que proporciona el estado inverso del que esté en su entrada. \\

La ecuación característica que describe el comportamiento de la puerta NOT es: $F=\bar{A}$ \\

%\begin{equation}
%    F=\bar{A}
%\end{equation}



\section{Simulación}

Mediante el software de logisim se agregó la compuerta con un un interruptor momentáneo.\\




%\section{Referencias}

%\bibliographystyle{plain}
%\bibliography{Referencias.bib}



\end{document}
