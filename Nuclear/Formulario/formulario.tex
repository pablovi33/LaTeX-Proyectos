\documentclass[]{article}
\usepackage[spanish.mexico]{babel}
\usepackage[T1]{fontenc}
\usepackage[utf8]{inputenc}
%\usepackage{lmodern}
\usepackage[a4paper]{geometry}

%Plotting

\usepackage{pgfplots}
\pgfplotsset{width=10cm,compat=1.9} 
%\usepgfplotslibrary{external}
%\tikzexternalize 

%Graficos e imagenes
\usepackage{graphicx}
%\graphicspath{ Imagenes/ }

\usepackage{natbib}
%\usepackage{cite}

\usepackage{subcaption}

%Grafico de barras
%\usepackage{pgfplots}


\usepackage{tikz}
\usepackage[american voltages, american currents,siunitx]{circuitikz}

\title{Formulario Nuclear}
\author{Pablo Vivar Colina}
%\date{Mayo 2018}


\begin{document}
	
%%\usepackage[top=2cm,bottom=2cm,left=1cm,right=1cm]{geometry}


\begin{titlepage}
     \begin{center}
	\includegraphics[width=0.09\textwidth]{UNAM}\Large Universidad Nacional Autónoma de México
        	\includegraphics[width=0.09\textwidth]{FI}\\[1cm]
        \Large Facultad de Ingeniería\\[1cm]
       % \Large División de Ciencias Básicas\\[1cm]
         \Large Laboratorio de Dispositivos y Circuitos Electrónicos (6654)\\[1cm]
         %la clave antes era:4314
         \footnotesize Profesor: Zapata Rosales Arturo Ing.\\[1cm]
        \footnotesize Semestre 2018-1\\[1cm]
        %\Large Práctica No. 1\\[1cm]
    
        %\Large Práctica No. 2\\[1cm]
        
        %\Large Práctica No. 3\\[1cm]
       
        %\Large Práctica No. 4\\[1cm]
         
               
         %\Large Práctica No. 5\\[1cm]
         
         
         %\Large Práctica No. 6\\[1cm]
         
         %\Large Práctica No. 7\\[1cm]
         
             %\Large Práctica No. 8\\[1cm]
       

        \Large Práctica No. 9\\[1cm]
        
           %####AQUI VAMOS#### ya ahora sii
           
        %\Large Práctica No. 11\\[1cm]
        %\Large Práctica No. 12\\[1cm]
        %\Large Práctica No. 13\\[1cm]
        
        %\Large Amplificador Operacional como Integrador\\[1cm]
        %\Large{Filtros}\\[1cm]
         %\Large{Medición de  corrientes en un circuito}\\[1cm]
         %practica 4
         %Large{Amplificador operacional como seguidor de voltaje en entrada inversora}\\[1cm]
         %practica5
         %\Large{Amplificador operacional como integrador}
         
         %Practica 7
%Comportamiento de un diodo Zener

\Large Diodo Zener
        
         %Texto a la derecha
          \begin{flushright}
\footnotesize  Grupo 13\\[0.5cm]
\footnotesize Brigada: 7\\[0.5cm]

\footnotesize Vivar Colina Pablo\\[0.5cm]
 \end{flushright}
    %Texto a la izquierda
          \begin{flushleft}
        \footnotesize Ciudad Universitaria Abril de 2018.\\
          \end{flushleft}
         
          
        %\vfill
        %\today
   \end{center}
\end{titlepage}
 %agregar portada

\maketitle

%\tableofcontents  % Write out the Table of Contents

%\listoffigures  % Write out the List of Figures

\section{Partículas de interés}

\subsection{Positrón y Negatrón (electrón)}

\begin{itemize}
	\item Masa en reposo $9.10954x10^{31}[Kg]$
	\item Carga $e=1.60219x10^{-19}[C]$ 
\end{itemize}

\subsection{Protón}

\begin{itemize}
	\item Masa en reposo $1.67265x10^{-27}[Kg]$
	\item Carga $1.60219x10^{-19}[C]$
\end{itemize}

\subsection{Neutrón}

\begin{itemize}
	\item Masa en reposo $1.67265x10^{-27}[Kg]$
	\item Carga Neutro
\end{itemize}

\subsection{Neutrino}

\begin{itemize}
	\item Masa en reposo cero
	\item electrón neutrinos y electrón antineutrinos
\end{itemize}

\section{Estructura nuclear}

\begin{figure}
	\centering
	$^A_ZNe$
	\caption{Z=Número Atómico, A=Número de nucleones
}
\end{figure}


	$A= Z+N$,	N=Número de neutrones en el núcleo, Isótopo: Igual Z pero diferente N.

\section{Peso atómico}

\begin{equation}
   M(^AZ)=12x\frac{m(^AZ)}{m(^{12}C)}
\end{equation}

\section{Número de Avogadro}

\begin{equation}
    N_A= 0.6022045 x 10^{24}
\end{equation}

\section{Radio atómico y nuclear}

\subsection{Radio atómico}

\begin{equation}
  2x10^{-10}[m]
\end{equation}

\subsection{Radio nuclear}

\begin{equation}
R = 1.25(fm)A^{\frac{1}{3}} 
\end{equation}

Donde: R está en fentómetros $(fm = 1 x 10^{-13}cm)$ y A es
el número de masa atómica.

\section{Ecuación de Einstein}

\begin{equation}
E=mc^2
\end{equation}

\section{Electrón Volt}

\begin{equation}
1[eV]=1.60219x10^{-19}[C]x1[V]=1.60219x10^{-19}[J]
\end{equation}

Incremento de energía cinética de un
electrón cuando pasa a través de una diferencia de
potencial de un volt.\\

La carga de un electrón es 0.5110 [MeV].\\

\section{Movimiento del átomo}

Cuando un cuerpo entra en movimiento, la relación de
masa se incrementa de acuerdo con la fórmula:\\

\begin{equation}
 m=\frac{m_0}{\sqrt{\frac{1-v^2}{c^2}}}
\end{equation}

Donde $v$ es su velocidad.\\

\subsection{Energía total de una partícula}

\begin{equation}
   E_{total}=mc^2
\end{equation}

\subsection{Energía cinética de una partícula}

\begin{equation}
E=mc^2-m_0c^2=m_0c^2[\frac{1}{\sqrt{1-\frac{v^2}{c^2}}}-1]
\end{equation}

Cuándo $v<<c$ se utiliza la fórmula clásica de la energía cinética $E=\frac{1}{2}m_0v^2$


\begin{itemize}
	\item Para electrones: Mecánica Clásica $E<10 [KeV]$
	\item Para Neutrones: Mecánica Clásica $E<20 [MeV]$
\end{itemize}









\section{Energía de ligadura}

Ionización de un electrón de un átomo

\begin{equation}
  Es=[M_n+M(^{A-1}Z)-M(^AZ)]931MeV
\end{equation}

$Es$ es la suficiente para remover el último neutrón del núcleo sin proveerle energía cinética alguna. Si el proceso fuera al revés, y un neutrón sin energía cinética es absorbido por el núcleo, la energía $E$ es liberada.\\

\begin{equation}
  m_0 n+mC^{12}+mC^{13}
\end{equation}

\begin{equation}
  (1.008664923+12-13.003354838)*931.5MeV=4.946344178MeV
\end{equation}


\section{Constante de Planck}


\begin{equation}
  h=6.62607015x10^{-34}[\frac{kgm^2}{s}]
\end{equation}

\begin{equation}
 E=hf
\end{equation}

Donde h es la constante de planck, y f es la frecuencia de oscilación de la partícula. La longitud de onda de una partícula con un momentum $p$ es:\\


\begin{equation}
   \lambda=\frac{h}{p}
\end{equation}

Para las partículas con energía potencial diferente de cero es :\\

\begin{equation}
   p=mf
\end{equation}

Para energías no relativistas el momentum $p$ se calcula:\\

\begin{equation}
  p=\sqrt{2m_0E}
\end{equation}

Y de la misma forma la longitud de onda de la partícula puede ser descrita como:\\

\begin{equation}
   \lambda=\frac{h}{\sqrt{2m_oE}}
\end{equation}

La longitud de onda del neutrón se obtiene:\\

\begin{equation}
   \lambda=\frac{2.860x10^-9}{\sqrt{E}}
\end{equation}

Donde $\lambda$ está en [cm] y $E$ en [eV].\\

Para el caso relativista $p$ se calcula a partir de:\\

\begin{equation}
p=\frac{\sqrt{E_{total}^2-E_{rep}^2}{c}
\end{equation}

Entonces la longitud de onda la podemos obtener como:\\

\begin{equation}
 \lambda=\frac{hc}{\sqrt{E_{total}^2-E_{rep}^2}
\end{equation}

Para las partículas con energía potencial igual a cero tenemos que:\\

\begin{equation}
p=\frac{E}{c}
\end{equation}

\begin{equation}
\lambda=\frac{hc}{E}
\end{equation}

%\section{Referencias}

%\bibliographystyle{plain}
%\bibliography{Referencias.bib}



\end{document}
