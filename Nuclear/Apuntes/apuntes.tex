\documentclass[]{article}
\usepackage[spanish.mexico]{babel}
\usepackage[T1]{fontenc}
\usepackage[utf8]{inputenc}
%\usepackage{lmodern}
\usepackage[a4paper]{geometry}

%Plotting

\usepackage{pgfplots}
\pgfplotsset{width=10cm,compat=1.9} 
%\usepgfplotslibrary{external}
%\tikzexternalize 

%Graficos e imagenes
\usepackage{graphicx}
%\graphicspath{ Imagenes/ }

\usepackage{natbib}
%\usepackage{cite}

\usepackage{subcaption}

%Grafico de barras
%\usepackage{pgfplots}


\usepackage{tikz}
\usepackage[american voltages, american currents,siunitx]{circuitikz}

\title{Apuntes Nuclear}
\author{Pablo Vivar Colina}
%\date{Mayo 2018}


\begin{document}
	
%%\usepackage[top=2cm,bottom=2cm,left=1cm,right=1cm]{geometry}


\begin{titlepage}
     \begin{center}
	\includegraphics[width=0.09\textwidth]{UNAM}\Large Universidad Nacional Autónoma de México
        	\includegraphics[width=0.09\textwidth]{FI}\\[1cm]
        \Large Facultad de Ingeniería\\[1cm]
       % \Large División de Ciencias Básicas\\[1cm]
         \Large Laboratorio de Dispositivos y Circuitos Electrónicos (6654)\\[1cm]
         %la clave antes era:4314
         \footnotesize Profesor: Zapata Rosales Arturo Ing.\\[1cm]
        \footnotesize Semestre 2018-1\\[1cm]
        %\Large Práctica No. 1\\[1cm]
    
        %\Large Práctica No. 2\\[1cm]
        
        %\Large Práctica No. 3\\[1cm]
       
        %\Large Práctica No. 4\\[1cm]
         
               
         %\Large Práctica No. 5\\[1cm]
         
         
         %\Large Práctica No. 6\\[1cm]
         
         %\Large Práctica No. 7\\[1cm]
         
             %\Large Práctica No. 8\\[1cm]
       

        \Large Práctica No. 9\\[1cm]
        
           %####AQUI VAMOS#### ya ahora sii
           
        %\Large Práctica No. 11\\[1cm]
        %\Large Práctica No. 12\\[1cm]
        %\Large Práctica No. 13\\[1cm]
        
        %\Large Amplificador Operacional como Integrador\\[1cm]
        %\Large{Filtros}\\[1cm]
         %\Large{Medición de  corrientes en un circuito}\\[1cm]
         %practica 4
         %Large{Amplificador operacional como seguidor de voltaje en entrada inversora}\\[1cm]
         %practica5
         %\Large{Amplificador operacional como integrador}
         
         %Practica 7
%Comportamiento de un diodo Zener

\Large Diodo Zener
        
         %Texto a la derecha
          \begin{flushright}
\footnotesize  Grupo 13\\[0.5cm]
\footnotesize Brigada: 7\\[0.5cm]

\footnotesize Vivar Colina Pablo\\[0.5cm]
 \end{flushright}
    %Texto a la izquierda
          \begin{flushleft}
        \footnotesize Ciudad Universitaria Abril de 2018.\\
          \end{flushleft}
         
          
        %\vfill
        %\today
   \end{center}
\end{titlepage}
 %agregar portada

\maketitle

%\tableofcontents  % Write out the Table of Contents

%\listoffigures  % Write out the List of Figures

Prueba de cambios y compilqacion

\section{Energía de ligadura}

Ionización de un electrón de un átomo

\begin{equation}
  Es=[M_n+M(^{A-1}Z)-M(^AZ)]931MeV
\end{equation}

$Es$ es la suficiente para remover el último neutrón del núcleo sin proveerle energía cinética alguna. Si el proceso fuera al revés, y un neutrón sin energía cinética es absorbido por el núcleo, la energía $E$ es liberada.\\

\begin{equation}
  m_0 n+mC^{12}+mC^{13}
\end{equation}

\begin{equation}
  (1.008664923+12-13.003354838)*931.5MeV=4.946344178MeV
\end{equation}

\subsection{Ejercicio}

El tritio $(^3H)$ puede producirse a través de la absorción de neutrones de baja energía por el Deuterio $(^2H)$, la reacción es:\\

\begin{equation}
  ^2H+n->^3H+ \gamma
\end{equation}

Donde los rayos gamma tienen una energía de 6.250 MeV.\\

\begin{enumerate}
	\item Muestre que la energía en retroceso del núcleo de $^3H$ es aprox 7 kEV
	\item Cual es el valor Q de la reacción
	\item Calcule la energía de separación del último neutrón del Tritio
	\item Utilizando la energía de ligadura del $^2H$ como 2.23 MeV y el resueltro del inciso 3, calcule la energía de ligadura total del $^3H$ 
\end{enumerate}

\begin{equation}
Q=931.5*(3.016029310-(2.014001778+1.008664923))=6.27438913MeV
\end{equation}

Se sabe que la energía de ligadura del Deuterio es de 2.23MeV.\\

\begin{equation}
2.23+6.25=8.48 MeV
\end{equation}

Debemos notar que los núcleos que contienen 2,6,8,14,20,28,50,82 o 126 neutrones o protones son especialmente estables.\\



\section{Equivalencias de unidades}

\begin{equation}
   1ev=1.60219x10^{-19} [J]
\end{equation}

\begin{equation}
   1uma=931.5meV
\end{equation}


\section{Partículas}

\subsection{Masa del Hidrógeno (proton y electrón)}

masa del Hidrógeno  1.007825032.\\

\section{Reactor HTGMR}

GT-HMR.\\

reactor fr turbina de gas.\\

reactor modular de helio de turbina de gas, minimiza el uso de tuberías para su refrigeración
la zona activa nunca puede fundirse, debido a la termodinamica y a los materiales con los que se construye.\\

Buques centrales nucleares móviles.\\

akademik lomolosov.\\

KLT40-S 35 MW de electricidad.\\

htgmr combustible en esferas recubrimiento de grafíto pirolitica esferas pequeñas
pvmr cama de esferas ambos de éstos reactores usan gas como refrigerante
el grafito funcikna como moderaqdor del combustible.\\

usar agua solo permite elevar la temperatura entre 200 y 300 grados C.\\

usar gas permite utilizar mayores temperaturas alrededor de 1000 grados o 11000
esto permite tener una mejor eficiencia.\\

velocidad de reactores de neutrones rápidos $14 000 000  \frac{m}{s}$.\\

\section{Reservas Uranio}

Reservas Razonablemente Aseguradas (RRA) 80 USD por kilo de uranio.\\

\section{Sistemas de Seguridad}

ECCS Sistemas de nefriamiento del núcleo.\\
Accidente de base de diseño, ruptura de tubo de bomba de recirculación.\\
grosor de la vasija 16 cm en ABWR.\\
en PWR en 24cm de ancho por mayor presion.\\



%\section{Referencias}

%\bibliographystyle{plain}
%\bibliography{Referencias.bib}



\end{document}
