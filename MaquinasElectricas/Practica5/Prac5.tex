\documentclass[]{article}
\usepackage[spanish.mexico]{babel}
\usepackage[T1]{fontenc}
\usepackage[utf8]{inputenc}
%\usepackageB.\\{lmodern}
\usepackage[a4paper]{geometry}

%\usepackage{natbib}
\usepackage{cite}


%Grafico de barras
\usepackage{pgfplots}

%Graficos e imagenes
\usepackage{graphicx}
\usepackage{subcaption}

%circuitos
\usepackage{tikz}
\usepackage[american voltages, american currents,siunitx]{circuitikz}

%Plotting

\usepackage{pgfplots}
\pgfplotsset{width=10cm,compat=1.9} 


%\title{Proyecto de Optimización de Energía}
%\author{Pablo Vivar Colina}
%\date{Mayo 2018}

%agregar portada

\begin{document}
	
	%\usepackage[top=2cm,bottom=2cm,left=1cm,right=1cm]{geometry}


\begin{titlepage}
     \begin{center}
	\includegraphics[width=0.09\textwidth]{UNAM}\Large Universidad Nacional Autónoma de México
        	\includegraphics[width=0.09\textwidth]{FI}\\[1cm]
        \Large Facultad de Ingeniería\\[1cm]
       % \Large División de Ciencias Básicas\\[1cm]
         \Large Laboratorio de Dispositivos y Circuitos Electrónicos (6654)\\[1cm]
         %la clave antes era:4314
         \footnotesize Profesor: Zapata Rosales Arturo Ing.\\[1cm]
        \footnotesize Semestre 2018-1\\[1cm]
        %\Large Práctica No. 1\\[1cm]
    
        %\Large Práctica No. 2\\[1cm]
        
        %\Large Práctica No. 3\\[1cm]
       
        %\Large Práctica No. 4\\[1cm]
         
               
         %\Large Práctica No. 5\\[1cm]
         
         
         %\Large Práctica No. 6\\[1cm]
         
         %\Large Práctica No. 7\\[1cm]
         
             %\Large Práctica No. 8\\[1cm]
       

        \Large Práctica No. 9\\[1cm]
        
           %####AQUI VAMOS#### ya ahora sii
           
        %\Large Práctica No. 11\\[1cm]
        %\Large Práctica No. 12\\[1cm]
        %\Large Práctica No. 13\\[1cm]
        
        %\Large Amplificador Operacional como Integrador\\[1cm]
        %\Large{Filtros}\\[1cm]
         %\Large{Medición de  corrientes en un circuito}\\[1cm]
         %practica 4
         %Large{Amplificador operacional como seguidor de voltaje en entrada inversora}\\[1cm]
         %practica5
         %\Large{Amplificador operacional como integrador}
         
         %Practica 7
%Comportamiento de un diodo Zener

\Large Diodo Zener
        
         %Texto a la derecha
          \begin{flushright}
\footnotesize  Grupo 13\\[0.5cm]
\footnotesize Brigada: 7\\[0.5cm]

\footnotesize Vivar Colina Pablo\\[0.5cm]
 \end{flushright}
    %Texto a la izquierda
          \begin{flushleft}
        \footnotesize Ciudad Universitaria Abril de 2018.\\
          \end{flushleft}
         
          
        %\vfill
        %\today
   \end{center}
\end{titlepage}
 
	

	%\maketitle
	
	%\tableofcontents  % Write out the Table of Contents
	
	%\listoffigures  % Write out the List of Figures
		
	\section{Introducción}
	
	\subsection{Representación fasorial}
		
	La corriente alterna se puede representar con una flecha girando a velocidad angular $\omega$. Este elemento recibe el nombre de fasor y se representa como un número complejo.\\
	
	Su longitud coincide con el valor máximo de la tensión o corriente (según sea la magnitud que se esté representando). También se utiliza el valor RMS en lugar del valor máximo (ver transformación a fasores). En ese caso habría que dividir el valor máximo por raíz de 2.\\
	
	El ángulo (corrimiento de la señal sobre el eje horizontal) representa la fase. La velocidad de giro $\omega$ está relacionada con la frecuencia de la señal.\cite{FisicaPractica}\\

		
	\section{Objetivos}
	
    Para verificar el diagrama, se aplica al lado de alta tensión un sistema trifásico de voltajes, tomando lecturas con un voltímetro, interconectando a la vez una terminal de alta tensión y una de baja tensión.
     
	\section{Resultados}
	
	
	\begin{table}[h!]
		\centering
		\begin{tabular}{|c|c|c|}
			\hline
			Terminales Alta	& Terminales Baja  & Defasamiento  \\ \hline
		$\Delta$	& $\Delta$ & 0 $^o$  \\ \hline
			& $\lambda$  &  30 $^o$ \\ \hline
		$\lambda$ 	& $\lambda$  & 0 $^o$ \\ \hline
		$\lambda$ 	& $\Delta$ &  30 $^o$\\ \hline
		
		\end{tabular}
		\caption{Conexiones en Transformador}
	\end{table}
	
	\begin{table}[h!]
		\centering
		\begin{tabular}{|c|c|}
			\hline
		$T_2[V]$	& Medición $T_1[V]$  \\ \hline
		286.6	& $H_3-X_2=$ 56 \\ \hline
		286.8	& $H_3-X_3=$ 55.2 \\ \hline
		292.7	& $H_1-H_3=$ 109.7 \\ \hline
		280.3	& $H_2-X_2=$ 56.2 \\ \hline
		290.3	& $H_2-H_3=$ 151.6  \\ \hline
		\end{tabular}
		\caption{Relacion de conexiones}
	\end{table}
	
	Las relaciones mostradas a continuación deben cumplirse para las sitaciones donde existan $30^o$ de defasamiento.\\
	
	\begin{eqnarray}
	H_3-X_2=H_3-X_3\\
	H_3-X_2<H_1-H_3\\
	H_2-X_2<H_2-X_3\\
	H_2-X_2<H_1-H_3
	\end{eqnarray}
	
	Para el primer transformador se obtuvo lo siguiente:\\
	
	\begin{eqnarray}
	56[V]\approx 55.2[V]\\
	56[V]<109.7[V]\\
	56.2<151.6[V]\\
	56.2<109.7[V]
	\end{eqnarray}
	
	Para el segundo transformador se obtuvo lo siguiente:\\
	
	\begin{eqnarray}
	286.6[V] \approx 286.8[V]\\
	286.6[V]<292[V]\\
	280.3[V]<290.3[V]\\
	280.3<292.7[V]
	\end{eqnarray}
	
		Para el tercer transformador se obtuvo lo siguiente:\\
		
		\begin{eqnarray}
		285.2[V] \approx 284.8[V]\\
		285.2[V]<284.8[V]\\
		283.2[V]<293.3[V]\\
		280.3<292.7[V]
		\end{eqnarray}
	
	%AQUI VAMOS
	
	Es importante mencionar que las cargas lineales no deforman la forma de la onda, sin embargo las cargas no lineales sí lo hacen.\\
	
	Ejemplos de cargas lineales pueden ser una resistencia o una inductancia, y ejemplos de cargas no lineales pueden ser la luz led, semiconductores, etc.\\
    
	cd ..
	\section{Conclusiones}
	
	El objetivo de la práctica se cumplió porque logramos verificar de manera presencial el desfasamiento fasorial que se producen entre las conexiones delta y estrela que pueden presentar los transformadores.

	\section{Referencias}
	

	\bibliographystyle{plain}
    \bibliography{Referencias}
    


	
	
\end{document}

