\documentclass[]{article}
\usepackage[spanish.mexico]{babel}
\usepackage[T1]{fontenc}
\usepackage[utf8]{inputenc}
%\usepackageB.\\{lmodern}
\usepackage[a4paper]{geometry}

%\usepackage{natbib}
\usepackage{cite}


%Grafico de barras
\usepackage{pgfplots}

%Graficos e imagenes
\usepackage{graphicx}

%circuitos
\usepackage{tikz}
\usepackage[american voltages, american currents,siunitx]{circuitikz}



%\title{Proyecto de Optimización de Energía}
%\author{Pablo Vivar Colina}
%\date{Mayo 2018}

%agregar portada

\begin{document}
	
	%\usepackage[top=2cm,bottom=2cm,left=1cm,right=1cm]{geometry}


\begin{titlepage}
     \begin{center}
	\includegraphics[width=0.09\textwidth]{UNAM}\Large Universidad Nacional Autónoma de México
        	\includegraphics[width=0.09\textwidth]{FI}\\[1cm]
        \Large Facultad de Ingeniería\\[1cm]
       % \Large División de Ciencias Básicas\\[1cm]
         \Large Laboratorio de Dispositivos y Circuitos Electrónicos (6654)\\[1cm]
         %la clave antes era:4314
         \footnotesize Profesor: Zapata Rosales Arturo Ing.\\[1cm]
        \footnotesize Semestre 2018-1\\[1cm]
        %\Large Práctica No. 1\\[1cm]
    
        %\Large Práctica No. 2\\[1cm]
        
        %\Large Práctica No. 3\\[1cm]
       
        %\Large Práctica No. 4\\[1cm]
         
               
         %\Large Práctica No. 5\\[1cm]
         
         
         %\Large Práctica No. 6\\[1cm]
         
         %\Large Práctica No. 7\\[1cm]
         
             %\Large Práctica No. 8\\[1cm]
       

        \Large Práctica No. 9\\[1cm]
        
           %####AQUI VAMOS#### ya ahora sii
           
        %\Large Práctica No. 11\\[1cm]
        %\Large Práctica No. 12\\[1cm]
        %\Large Práctica No. 13\\[1cm]
        
        %\Large Amplificador Operacional como Integrador\\[1cm]
        %\Large{Filtros}\\[1cm]
         %\Large{Medición de  corrientes en un circuito}\\[1cm]
         %practica 4
         %Large{Amplificador operacional como seguidor de voltaje en entrada inversora}\\[1cm]
         %practica5
         %\Large{Amplificador operacional como integrador}
         
         %Practica 7
%Comportamiento de un diodo Zener

\Large Diodo Zener
        
         %Texto a la derecha
          \begin{flushright}
\footnotesize  Grupo 13\\[0.5cm]
\footnotesize Brigada: 7\\[0.5cm]

\footnotesize Vivar Colina Pablo\\[0.5cm]
 \end{flushright}
    %Texto a la izquierda
          \begin{flushleft}
        \footnotesize Ciudad Universitaria Abril de 2018.\\
          \end{flushleft}
         
          
        %\vfill
        %\today
   \end{center}
\end{titlepage}
 
	

	%\maketitle
	
	%\tableofcontents  % Write out the Table of Contents
	
	%\listoffigures  % Write out the List of Figures
		
	\section{Introducción}
	
	Para determinar la relación de transformación existen tres métodos:\cite{Angelo-kun2015}\\
	
	\begin{itemize}
		\item Método de los voltímetros
		\item Método del transformador patrón.
		\item Método del potenciómetro de resistencia.
	\end{itemize}
	
	Básicamente los tres métodos consisten en aplicar a uno de los devanados una tensión alterna y detectar el voltaje inducido en el otro devanado.\\
		
	\section{Objetivos}
		
	Aplicar alguno de los métodos mencionados en la introducción para verificar experimentalmente la secuencia de fases del mismo.

	\section{Resultados}
	
\subsection{Método del transformador patrón}

	\begin{figure}[h!]
		\centering
		\begin{circuitikz}
			
			\draw
			
			(-2,0)to[vco,l=$G_E$](-2,-2)
			
			(-1,0.25) node {$A$}
			(-2,0)--(0,0)
			(-2,-2)--(0,-2)
			(-1,-2.25) node {$B$}
			
			(0,0)to[L](0,-2)
			(0.4,0)--(0.4,-2)
			(0.7,0)--(0.7,-2)
		    (1,0)to[L](1,-2)
		    
		    (1,0.25) node {$C$}
		    (1,-2.25) node {$D$}
		    
		    (1,0)--(2,0)
		    
		    
		    (0,-2.75) node {$E$}
		    (0,-5.25) node {$F$}
		    
		    (0,-3)to[L](0,-5)
		    (0.4,-3)--(0.4,-5)
		    (0.7,-3)--(0.7,-5)
		    (1,-3)to[L](1,-5)
		
		    (1,-2.75) node {$G$}
		    (1,-5.25) node {$H$}
		
		    (1,-3)--(2,-3)
		    
		    (2,0) to[voltmeter,l=$V_m$](2,-3) 
		    
		     (-2.25,-4) node {$A$}
		      (-2,-4)--(-1.5,-4)
		       (-1.25,-4) node {$E$}
		      
		      (-2.25,-4.5) node {$F$}
		      (-2,-4.5)--(-1.5,-4.5)
		      (-1.25,-4.5) node {$B$}
		       (-2.25,-4.5)
		       
		       (-2.25,-5)
		        node {$D$}
		       (-2,-5)--(-1.5,-5)
		       (-1.25,-5) node {$H$}
			;
			
		\end{circuitikz}
		\caption{Prueba del transformador patrón}
		\label{fig:PruebaTransformadorPatron}
	\end{figure}
	
	En la figura \ref{fig:PruebaTransformadorPatron} se muestra la conexión realizada en el laboratorio, y se comprobó que la relación de transformación del transfomador patrón y de prueba era la misma ya que al energizar el generador $G_E$ no existió diferencia de potencial, esto lo verificamos a través de un voltmetro $V_m$ conectado entre los nodos C y G.\\
    
	
	\section{Conclusiones}
	
	El objetivo de la práctica se cumplió porque logramos verificar de manera presencial, la medición de la relación de transformación con instrumentos de medición

	\section{Referencias}
	

	\bibliographystyle{plain}
    \bibliography{Referencias}
    


	
	
\end{document}

