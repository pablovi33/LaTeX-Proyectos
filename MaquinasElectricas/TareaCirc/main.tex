\documentclass[]{article}
\usepackage[spanish.mexico]{babel}
\usepackage[T1]{fontenc}
\usepackage[utf8]{inputenc}
%\usepackageB.\\{lmodern}
\usepackage[a4paper]{geometry}
 \geometry{
	a4paper,
	total={170mm,257mm},
	left=20mm,
	top=20mm,
}

%columnas itemize
\usepackage{multicol}

%\usepackage{natbib}
\usepackage{cite}


%Grafico de barras
\usepackage{pgfplots}

%Graficos e imagenes
\usepackage{graphicx}


\usepackage{tikz}
\usepackage[american voltages, american currents,siunitx]{circuitikz}


\title{Examen Máquinas Eléctricas}
\author{Pablo Vivar Colina}
%\date{Mayo 2018}



\begin{document}
	
%	%\usepackage[top=2cm,bottom=2cm,left=1cm,right=1cm]{geometry}


\begin{titlepage}
     \begin{center}
	\includegraphics[width=0.09\textwidth]{UNAM}\Large Universidad Nacional Autónoma de México
        	\includegraphics[width=0.09\textwidth]{FI}\\[1cm]
        \Large Facultad de Ingeniería\\[1cm]
       % \Large División de Ciencias Básicas\\[1cm]
         \Large Laboratorio de Dispositivos y Circuitos Electrónicos (6654)\\[1cm]
         %la clave antes era:4314
         \footnotesize Profesor: Zapata Rosales Arturo Ing.\\[1cm]
        \footnotesize Semestre 2018-1\\[1cm]
        %\Large Práctica No. 1\\[1cm]
    
        %\Large Práctica No. 2\\[1cm]
        
        %\Large Práctica No. 3\\[1cm]
       
        %\Large Práctica No. 4\\[1cm]
         
               
         %\Large Práctica No. 5\\[1cm]
         
         
         %\Large Práctica No. 6\\[1cm]
         
         %\Large Práctica No. 7\\[1cm]
         
             %\Large Práctica No. 8\\[1cm]
       

        \Large Práctica No. 9\\[1cm]
        
           %####AQUI VAMOS#### ya ahora sii
           
        %\Large Práctica No. 11\\[1cm]
        %\Large Práctica No. 12\\[1cm]
        %\Large Práctica No. 13\\[1cm]
        
        %\Large Amplificador Operacional como Integrador\\[1cm]
        %\Large{Filtros}\\[1cm]
         %\Large{Medición de  corrientes en un circuito}\\[1cm]
         %practica 4
         %Large{Amplificador operacional como seguidor de voltaje en entrada inversora}\\[1cm]
         %practica5
         %\Large{Amplificador operacional como integrador}
         
         %Practica 7
%Comportamiento de un diodo Zener

\Large Diodo Zener
        
         %Texto a la derecha
          \begin{flushright}
\footnotesize  Grupo 13\\[0.5cm]
\footnotesize Brigada: 7\\[0.5cm]

\footnotesize Vivar Colina Pablo\\[0.5cm]
 \end{flushright}
    %Texto a la izquierda
          \begin{flushleft}
        \footnotesize Ciudad Universitaria Abril de 2018.\\
          \end{flushleft}
         
          
        %\vfill
        %\today
   \end{center}
\end{titlepage}
 %agregar portada

\maketitle

%\tableofcontents  % Write out the Table of Contents

%\listoffigures  % Write out the List of Figures



\section{Transformador}

\begin{multicols}{2}
	\begin{itemize}
		\item P=[15 KVA]
		\item V=2400[V] a 240 [V]
		\item f=60 [Hz]
		\item $R_2$=0.025 [$\Omega$]
		\item $R_C$=32[k$\Omega$]
		\item $X_m$=11.5 [k$\Omega$]
		\item $X_1$=7[$\Omega$]
		\item $X_2$=0.07[$\Omega$]
	\end{itemize}
\end{multicols}

\section{Motor trifásico}

\begin{multicols}{2}
	\begin{itemize}
		\item V=230 [V]
		\item P=5[Hp]=3.7282[W]
		\item f=60 [Hz]
		\item V=1746 [rpm]
		\item I=12[A]
		\item 2.5 $\%$ rotor
		\item 2.5 $\%$ estator
		\item 5 $\%$ núcleo
		\item 3 $\%$ devanado
		\item $X_1$=2.7 [$\Omega$]
		\item $X_R$=37 [$Omega$]
	\end{itemize}

\end{multicols}

\begin{figure}[h!]
	\centering
%	\begin{center}
		\begin{circuitikz}
			
			\draw
			
			
			(0,0)to[R,l=$R_1$25](2,0)
			(2,0)to[L,l=$L_1$7](4,0)
			
			
			(4,0)to[R,l=$R_2$0.025](6,0)
			(6,0)to[L,l=$L_2$11.5](8,0)
			
			(4,0)--(4,-1)
			(3,-1)--(5,-1)
			
			(3,-1)to[R,l=$R_3$0.025](3,-3)
			(5,-1)to[L,l=$L_3$0.007](5,-3)
			
			(3,-3)--(5,-3)
			
			(4,-3)--(4,-4)
			
			(0,-4)--(8,-4)
			
			(8,-4)--(8,0)
			
			%(8,-2)node[anchor=east] {240V}
			(8,-2)node[anchor=west] {240V}
			
			
			(8,-3)rectangle(9,-1)
			
			(9,-4)--(9,0)
			
			%LO MISMO MAS 9
			
			(0+9,0)to[R,l=$R_4$1K](2+9,0)
			(2+9,0)to[L,l=$L_4$27](4+9,0)
			
			
			(4+9,0)to[R,l=$R_5$1K](6+9,0)
			(6+9,0)to[L,l=$L_5$1m](8+9,0)
			
			(4+9,0)--(4+9,-1)
			(3+9,-1)--(5+9,-1)
			
			(3+9,-1)to[R,l=$R_6$1K](3+9,-3)
			(5+9,-1)to[L,l=$L_6$0](5+9,-3)
			
			(3+9,-3)--(5+9,-3)
			
			(4+9,-3)--(4+9,-4)
			
			(9,-4)--(8+9,-4)
			
			(8+9,-4)--(8+9,0)


			
			%(8,-2)node[anchor=east] {240V}
			(8+9,-2)node[anchor=west] {$\frac{1-s}{s}$}
				
          (8+9,-3)rectangle(9+9,-1)
			
		
			
			%tierras
			(4,-3) to   (4,-4) node[ground]{}
			(4+9,-3) to   (4+9,-4) node[ground]{}
			
			
			
			
			%node[scale=1.2,nand port,anchor=in 1] (nand1) {}
			%(nand1.in 2) -- (xor1.in 2|-nand1.in 2)
			
			%(3,0)to[push button,l=$R_1$](3,3)
			
			
			%Generador de funciones
			%(-6,0.5)--(-5,0.5)
			%(0,3)to[V,l=$G$](0,0)
			
			;
			
		\end{circuitikz}
		
%\end{center}
	
		
%\caption{Circuito de acción momentánea}
%\label{fig:CircuitoMomentáneo}
\end{figure}

Determinar

\begin{enumerate}
	\item Velocidad Síncrona
	\item Deslizamiento
	\item La frecuencia de las corrientes del motor
	\item Par de Salida
	\item Corriente de suministro del motor
	\item La eficiencia del motor
	\item Factor de potencia
\end{enumerate}

\section{Desarrollo}

\subsection{Velocidad Síncrona}


\begin{equation}
  n_s=\frac{120(60)}{8}(1-0.03)=900
  \label{ns}
\end{equation}

\subsection{Deslizamiento}

%Del resultado de la ecuación \ref{ns} sustituimos:\\

\begin{equation}
s=\frac{n_s-n_m}{n_s}=\frac{1800-1746}{1800}=0.03
\end{equation}

\subsection{Frecuencia de las corrientes del motor}

\begin{equation}
f_i=sf=(0.03)(60)=1.8[Hz]
\end{equation}

\subsection{Par de Salida}

\begin{equation}
 \tau=\frac{P}{v}=\frac{3.7282[kW]}{1746[rpm](\frac{1[M]}{60[s]})2 \pi}=\frac{3.7282[kW]}{182[\frac{rad}{s}]}=20[Nm]
\end{equation}

\subsection{Corriente de suministro del motor}

\begin{equation}
\eta=\frac{P_{salida}}{P_{entrada}}=\frac{3728}{4285}=0.87=87\%
\end{equation}

\subsection{Factor de potencia}


\begin{equation}
  F.P.=\frac{4285}{(230)(8)\sqrt{3}}=0.59  
\end{equation}

\end{document}
