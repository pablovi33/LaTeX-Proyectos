\documentclass[]{article}
\usepackage[spanish.mexico]{babel}
\usepackage[T1]{fontenc}
\usepackage[utf8]{inputenc}
%\usepackageB.\\{lmodern}
%\usepackage{lmodern}
\usepackage[a4paper]{geometry}
%labels
%\usepackage{hyperref}% http://ctan.org/pkg/hyperref

%\usepackage{natbib}
\usepackage{cite}


%Grafico de barras
\usepackage{pgfplots}

%Graficos e imagenes
\usepackage{graphicx}
\usepackage{subcaption}

%circuitos
\usepackage{tikz}
\usepackage[american voltages, american currents,siunitx]{circuitikz}

%más circuitos
%\usetikzlibrary{circuits}
%\usetikzlibrary{circuits.ee.IEC}

%Plotting

\usepackage{pgfplots}
\pgfplotsset{width=10cm,compat=1.9} 


%\title{Proyecto de Optimización de Energía}
%\author{Pablo Vivar Colina}
%\date{Mayo 2018}

%agregar portada

\begin{document}
	
	%\usepackage[top=2cm,bottom=2cm,left=1cm,right=1cm]{geometry}


\begin{titlepage}
     \begin{center}
	\includegraphics[width=0.09\textwidth]{UNAM}\Large Universidad Nacional Autónoma de México
        	\includegraphics[width=0.09\textwidth]{FI}\\[1cm]
        \Large Facultad de Ingeniería\\[1cm]
       % \Large División de Ciencias Básicas\\[1cm]
         \Large Laboratorio de Dispositivos y Circuitos Electrónicos (6654)\\[1cm]
         %la clave antes era:4314
         \footnotesize Profesor: Zapata Rosales Arturo Ing.\\[1cm]
        \footnotesize Semestre 2018-1\\[1cm]
        %\Large Práctica No. 1\\[1cm]
    
        %\Large Práctica No. 2\\[1cm]
        
        %\Large Práctica No. 3\\[1cm]
       
        %\Large Práctica No. 4\\[1cm]
         
               
         %\Large Práctica No. 5\\[1cm]
         
         
         %\Large Práctica No. 6\\[1cm]
         
         %\Large Práctica No. 7\\[1cm]
         
             %\Large Práctica No. 8\\[1cm]
       

        \Large Práctica No. 9\\[1cm]
        
           %####AQUI VAMOS#### ya ahora sii
           
        %\Large Práctica No. 11\\[1cm]
        %\Large Práctica No. 12\\[1cm]
        %\Large Práctica No. 13\\[1cm]
        
        %\Large Amplificador Operacional como Integrador\\[1cm]
        %\Large{Filtros}\\[1cm]
         %\Large{Medición de  corrientes en un circuito}\\[1cm]
         %practica 4
         %Large{Amplificador operacional como seguidor de voltaje en entrada inversora}\\[1cm]
         %practica5
         %\Large{Amplificador operacional como integrador}
         
         %Practica 7
%Comportamiento de un diodo Zener

\Large Diodo Zener
        
         %Texto a la derecha
          \begin{flushright}
\footnotesize  Grupo 13\\[0.5cm]
\footnotesize Brigada: 7\\[0.5cm]

\footnotesize Vivar Colina Pablo\\[0.5cm]
 \end{flushright}
    %Texto a la izquierda
          \begin{flushleft}
        \footnotesize Ciudad Universitaria Abril de 2018.\\
          \end{flushleft}
         
          
        %\vfill
        %\today
   \end{center}
\end{titlepage}
 
	

	%\maketitle
	
	%\tableofcontents  % Write out the Table of Contents
	
	%\listoffigures  % Write out the List of Figures
		
	\section{Introducción}
	
	El propósito que tiene el ensayo o prueba de cortocircuito es el de determinar:\\
	
	
	\begin{itemize}
		\item  Las pérdidas en los bobinados.
		\item Las pérdidas de voltaje en el secundario cuando el transformador está funcionando nominalmente
		\item La impedancia del transformador principalmente.
		
	\end{itemize}

	Para realizar la prueba se pone el bobinado secundario del transformador en cortocircuito y se alimenta el bobinado primario con un voltaje alterno regulable. El voltaje alterno regulable parte de cero voltios y va incrementándose su valor hasta alcanzar las corrientes nominales en ambos bobinados del transformador. (Ver la corriente alterna C.A.)\\
	
	Con los valores nominales de corriente en ambos bobinados se mide el valor del voltaje en el primario $(Ecc)$ y se determina la impedancia del transformador utilizando la siguiente fórmula:\\
	
	\begin{equation}
	Imp=\frac{Ecc*100}{E1}\\
	\end{equation}
		
	\section{Objetivos}
	
	
	Los devanados sufren calentamiento, cuya energía se disipa al medio ambiente, constituyendo una pérdida. Podemos considerar que las pérdidas de carga tienen dos componentes, una suma de productos que serían las pérdidas óhmicas y otra que constituye las pérdidas indeterminadas.
	La prueba se efectúa poniendo en corto-circuito el lado de baja tensión y alimentando por el lado de alta tensión, logrando que circula en todos los embobinados su respectiva corriente nominal. La potencia que consume el transformador en estas condiciones representa las pérdidas eléctricas
	Los aparatos que se incluyen en el circuito de alimentación son:
	
	\begin{itemize}
	
	\item 	Monofásico
		\item Trifásico
		\item Frecuencímetro
		\item Amperímetro
	\item 	Wattímetro
		\item Voltímetro de valor eficaz
\end{itemize}
 
  
  
  
   
	\section{Resultados}
	
	
	\subsection{Transformadores utilizados}
	
	\subsubsection{Transformador A}
	
	\begin{itemize}
		\item Potencia = 15000 [kVA]
		\item 1 Fase
	\end{itemize}
	
	\begin{equation}
	  I_H=\frac{15000}{\sqrt{3}220}=39.36[A]
	\end{equation}
	
	
	\begin{equation}
	I_X=\frac{15000}{\sqrt{3}208}=41.63[A]
	\end{equation}
	
	\subsubsection{Transformador B}
	
	\begin{itemize}
		\item Potencia = 50000 [kVA]
		\item 1 Fase
	\end{itemize}
	
		\begin{equation}
		I_H=\frac{50000}{\sqrt{3}6000}=4.81[A]
		\end{equation}
		
		
		\begin{equation}
		I_X=\frac{50000}{\sqrt{3}220}=131.21[A]
		\end{equation}
	
	
	
	\subsubsection{Transformador C}
	
		\begin{itemize}
			\item Potencia = 15000 [kVA]
			\item 1 Fase
		\end{itemize}
		
		\begin{equation}
		I_H=\frac{15000}{\sqrt{3}6000}=1.44[A]
		\end{equation}
		
		
		\begin{equation}
		I_X=\frac{15000}{\sqrt{3}240}=36.08[A]
		\end{equation}
		
		Los valores de los transformadores a utilizar se pueden condensar en el cuadro \ref{transformadores}.\\
		
		La corriente de exitación de manera práctica es aproximadamente el 10 porciento de la corriente nominal, y para la seguridad en el momento de realizar el experimento se utilizó el transformador C que que nos da una corriente de exitación aproximada de 3.6 [A].\\
		
		Se utilizaron dispositivos TC para bajar el voltaje por fase.\\
		
		\begin{table}[h!]
			\centering
			\begin{tabular}{|c|c|c|}
				\hline
				Transformador & $I_H [A]$ & $I_X [A]$ \\ \hline
				A             & 39.36      & 41.63      \\ \hline
				B             & 4.81       & 131.21     \\ \hline
				C             & 1.44       & 36.08      \\ \hline
			\end{tabular}
			\caption{Características de los transformadores}
			\label{transformadores}
		\end{table}
	
	
	\begin{figure}[h!]
		\centering
		\begin{circuitikz}
			
			\draw
			
			
			%MEDIDORES
			 (6.25,3) to[esource] (6.25,4.5)
			 (6.25,3.75) node {$V$}
			 
			 (7.25,3) to[esource] (7.25,4.5)
			 (7.25,3.75) node {$VM$}
			
			(6.25,4.5)--(8.25,4.5)
			(8.25,4.5)--(8.25,1.5)
			[black,fill=black] (8.25,1.5)circle (.5ex)
			
			(6.25,3)--(7.5,3)
			(7.5,3)--(7.5,2.5)
			[black,fill=black] (7.5,2.5)circle (.5ex)
			
			
		 (0.5,3.25) node {$Generador$}
           
           (2.5,1.5) to[esource] (2.5,2.5)
           (2.5,2) node {$F$}
           
           %NEUTRO x0
           [black,fill=black] (9,3.4) circle (.5ex)
           (9.4,3.4) node {$X_0$}

            %LINEA 1
            (1.5,2.75) node {$1$}
            (1,2.5)--(3,2.5)

            (3,2.5) to[ammeter] (4.5,2.5)

            (4.5,2.5) to[esource] (6,2.5)
            (5.25,2.5) node {$W$}
            
            (5.5,2.2)--(9,2.2)
            (9,2.2)--(9,2.5)
            (9,2.5)--(6,2.5)
            [black,fill=black] (9,2.4) circle (.5ex)
              (9.4,2.4) node {$X_1$}
              
              (5.5,2.75)--(6.5,2.75)
            
            %LINEA 2
            
              (1.5,1.75) node {$2$}
            (1,1.5)--(3,1.5)
            
            (3,1.5) to[ammeter] (4.5,1.5)
            
            
            (4.5,1.5) to[esource] (6,1.5)
            (5.25,1.5) node {$W$}
            
            (5.5,1.2)--(9,1.2)
            (9,1.2)--(9,1.5)
            (9,1.5)--(6,1.5)
            [black,fill=black] (9,1.4) circle (.5ex)
              (9.4,1.4) node {$X_2$}
              
              %PUNTO de conexion de 3 wattmetros
              (5.5,1.75)--(6.5,1.75)
               [black,fill=black] (6.5,1.75) circle (.5ex)
            
            %LINEA 3
            (1.5,0.75) node {$3$}
			(1,0.5)--(3,0.5)
			
			(3,0.5) to[ammeter] (4.5,0.5)
			
	        (4.5,0.5) to[esource] (6,0.5)
	        (5.25,0.5) node {$W$}
			
			 (5.5,0.2)--(9,0.2)
			 (9,0.2)--(9,0.5)
			 (9,0.5)--(6,0.5)
			 [black,fill=black] (9,0.4) circle (.5ex)
			  (9.4,0.4) node {$X_3$}
			  
			   (5.5,0.75)--(6.5,0.75)
			   
			   %conecciones entre 3 WATTMETROs
			(6.5,0.75)--(6.5,2.75)
			
		%BORNES DE ALTA TENSION
			
			[black,fill=black] (10,2.4) circle (.5ex)
			(10.4,2.4) node {$H_1$}
			
			[black,fill=black] (10,1.4) circle (.5ex)
			(10.4,1.4) node {$H_2$}
			
			 [black,fill=black] (10,0.4) circle (.5ex)
			 (10.4,0.4) node {$H_3$}
			
			;
			\draw
				%Generador de funciones
				
				%(0,0)to[vco,l=$V$](0,2)
				
				%GENERADOR
				[black] (0,0)rectangle (1,3)
				
				
				%TRANSFORMADOR
				[gray,shift={(8.75,0)}](0,0)rectangle (2.15,4)
				
			;
		
		\end{circuitikz}
		\caption{Circuito de prueba trifásico}
		\label{fig:CircuitoPruebaTrifasico}
	\end{figure}
	
	
	\begin{table}[h!]
		\centering
		\begin{tabular}{|c|c|c|}
			\hline
			Fase       & Wattmetro & Ampermetro \\ \hline
			1          & 14        & 2.5        \\ \hline
			2          & 14        & 2.5        \\ \hline
			3          & 8         & 2.4        \\ \hline
			Total(x10) & 360       &            \\ \hline
		\end{tabular}
		\caption{Wattmetros en Fases}
		\label{WattmetrosFases}
	\end{table}
	
	
	En el cuadro \ref{WattmetrosFases} se ven los datos obtenidos tras hacer las conexiones en la figura \ref{fig:CircuitoPruebaTrifasico} y debido a el efecto causado por los TC es necesario tomar en cuenta el factor de multiplicación que éstos aplican (x10).\\
	
	
	
	
    
	
	\section{Conclusiones}
	
	El objetivo de la práctica se cumplió porque logramos verificar las pérdidas de potencia del transformador experimentalmente, además de verificar la impedancia del transformador.

	%\section{Referencias}
	

	%\bibliographystyle{plain}
    %\bibliography{Referencias.bib}
    


	
	
\end{document}

