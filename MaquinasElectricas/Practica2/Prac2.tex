\documentclass[]{article}
\usepackage[spanish.mexico]{babel}
\usepackage[T1]{fontenc}
\usepackage[utf8]{inputenc}
%\usepackageB.\\{lmodern}
\usepackage[a4paper]{geometry}

%\usepackage{natbib}
\usepackage{cite}


%Grafico de barras
\usepackage{pgfplots}

%Graficos e imagenes
\usepackage{graphicx}


%\title{Proyecto de Optimización de Energía}
%\author{Pablo Vivar Colina}
%\date{Mayo 2018}

%agregar portada

\begin{document}
	
	%\usepackage[top=2cm,bottom=2cm,left=1cm,right=1cm]{geometry}


\begin{titlepage}
     \begin{center}
	\includegraphics[width=0.09\textwidth]{UNAM}\Large Universidad Nacional Autónoma de México
        	\includegraphics[width=0.09\textwidth]{FI}\\[1cm]
        \Large Facultad de Ingeniería\\[1cm]
       % \Large División de Ciencias Básicas\\[1cm]
         \Large Laboratorio de Dispositivos y Circuitos Electrónicos (6654)\\[1cm]
         %la clave antes era:4314
         \footnotesize Profesor: Zapata Rosales Arturo Ing.\\[1cm]
        \footnotesize Semestre 2018-1\\[1cm]
        %\Large Práctica No. 1\\[1cm]
    
        %\Large Práctica No. 2\\[1cm]
        
        %\Large Práctica No. 3\\[1cm]
       
        %\Large Práctica No. 4\\[1cm]
         
               
         %\Large Práctica No. 5\\[1cm]
         
         
         %\Large Práctica No. 6\\[1cm]
         
         %\Large Práctica No. 7\\[1cm]
         
             %\Large Práctica No. 8\\[1cm]
       

        \Large Práctica No. 9\\[1cm]
        
           %####AQUI VAMOS#### ya ahora sii
           
        %\Large Práctica No. 11\\[1cm]
        %\Large Práctica No. 12\\[1cm]
        %\Large Práctica No. 13\\[1cm]
        
        %\Large Amplificador Operacional como Integrador\\[1cm]
        %\Large{Filtros}\\[1cm]
         %\Large{Medición de  corrientes en un circuito}\\[1cm]
         %practica 4
         %Large{Amplificador operacional como seguidor de voltaje en entrada inversora}\\[1cm]
         %practica5
         %\Large{Amplificador operacional como integrador}
         
         %Practica 7
%Comportamiento de un diodo Zener

\Large Diodo Zener
        
         %Texto a la derecha
          \begin{flushright}
\footnotesize  Grupo 13\\[0.5cm]
\footnotesize Brigada: 7\\[0.5cm]

\footnotesize Vivar Colina Pablo\\[0.5cm]
 \end{flushright}
    %Texto a la izquierda
          \begin{flushleft}
        \footnotesize Ciudad Universitaria Abril de 2018.\\
          \end{flushleft}
         
          
        %\vfill
        %\today
   \end{center}
\end{titlepage}
 
	

	%\maketitle
	
	%\tableofcontents  % Write out the Table of Contents
	
	%\listoffigures  % Write out the List of Figures
		
	\section{Introducción}
	
	
	\subsection{Resistencia de Aislamiento}
	
	La resistencia de aislamiento
	de un aislante es la resistencia que opone al paso de 
	la   corriente   eléctrica,   medida   en   la   dirección   en   que   se   tiene   que   asegurar   
	el aislamiento.\\
	
	La corriente de fuga de un aislante sigue dos caminos: uno sobre la superficie y 
	otro  a  través  del  interior  del  material.  La  resistencia  de  aislamiento  que  presenta  el  
	material se debe al efecto de estos dos caminos en paralelo.\\
	
	La 
	$resistividad$ $de$ $aislamiento$ $superficial$
	se mide
	en $M \Omega$
	y es debida a la resistencia 
	que  ofrece  la  superficie  del  material  al  paso  de  la  corriente  cuando  se  aplica  tensión  
	entre dos puntos de dicha superficie. Evidentemente esta magnitud está muy 
	afectada  por  el  estado  de  limpieza  de  la  superficie.  La  suciedad  (grasa,  polvo,  etc.)  
	depositada  sobre  la  superficie  de  un  aislante 
	reduce  la  resistividad  de  aislamiento  
	superficial. Por esta razón, las piezas aislantes hay que construirlas lisas y pulidas.\\
	
	Según  la  norma  UNE
	21303  la  resistividad  superficial  es  igual  a  la  resistencia  
	superficial 
	que presenta una superficie cuadrada y es independiente del tamaño de este 
	cuadrado. Para 
	obtener
	esta resistividad [...] se mide 
	la resistencia entre los electrodos 
	(por ejemplo, 
	usando  una  fuente  de  tensión  continua, 
	midiendo la corriente que aparece y aplicando la ley de Ohm) y l
	a resistividad 
	se calcula 
	multiplicando  esta
	resistencia
	por  el  perímetro  de  un  electrodo  y  dividiéndola 
	por  la  
	distancia
	entre 
	los 
	electrodos.
	La 
	resistividad de aislamiento transversal o volumétrica
	se mide en $\frac{M \Omega cm^2}{cm}$ y 
	es debida a la resistencia que ofrece el dieléctrico a ser atravesado por una corriente 
	cuando  se  aplica  tensión  entre  dos  de  sus  caras.  Esta  magnitud  no  tiene  un  
	valor constante para un mismo material, ya que le afectan la temperatura, la humedad, 
	el espesor de la pieza, el envejecimiento del material.\\
	
	\section{Objetivos}
	
	Los conductores de los devanados del transformador deben estar perfectamente asilados para evitar que entren en contacto las espiras, la bobina de alta y baja tensión, así como con el núcleo.\\
	
	La primera prueba para detectar el estado de los aislamientos es la medición de su resistencia (valor de orden de cuentos de megaohms). El aislamiento se debe medir:\\
	
	\begin{itemize}
		\item 	Entre los devanados de alta y baja tensión.
	    \item 	Entre el devanado de alta tensión y tierra.
		\item   Entre el devanado de baja tensión y tierra.
	\end{itemize}

	\section{Resultados}
	
	\begin{table}[h!]\footnotesize
		\caption{Resultados de la prueba de resistencia de aislamiento}
		\centering
		\begin{tabular}{ |c|c| }
			\hline
			\multicolumn{2}{|c|}{Resistencia de Aislamiento} \\
	        \hline
			Referencias & Valor Medio $M \Omega$  \\
			\hline
			Alta y Baja Tensión & 240  \\
			\hline
			Alta Tensión y Tierra & 172.5  \\
			\hline
			Baja Tensión y Tierra & 137  \\
			\hline
		\end{tabular}
    \end{table}
    
    \begin{table}[h!]\footnotesize
    	\caption{Resultados de la prueba de resistencia de aislamiento}
    	\centering
    	\begin{tabular}{ |c|c| }
    		\hline
    		\multicolumn{2}{|c|}{Resistencia de Aislamiento} \\
    		\hline
    		Referencias & Valor Medio $M \Omega$  \\
    		\hline
    		Alta y Baja Tensión & 350  \\
    		\hline
    		Alta Tensión y Tierra & 200  \\
    		\hline
    		Baja Tensión y Tierra & 180  \\
    		\hline
    	\end{tabular}
    \end{table}
    
    Para Un transformador trifásico con una potencia de 50KVA se realizaron las mediciones que se aprecian en el cuadro \ref{ultimo}.\\
    
    \begin{table}[h!]\footnotesize
    	\caption{Resultados de la prueba de resistencia de aislamiento}
    	\centering
    	\begin{tabular}{ |c|c| }
    		\hline
    		\multicolumn{2}{|c|}{Resistencia de Aislamiento} \\
    		\hline
    		Referencias & Valor Medio $M \Omega$  \\
    		\hline
    		Alta y Baja Tensión & 400 \\
    		\hline
    		Alta Tensión y Tierra & 400  \\
    		\hline
    		Baja Tensión y Tierra & 1  \\
    		\hline
    	\end{tabular}
    	\label{ultimo}
    \end{table}
    
	
	\section{Conclusiones}
	
Logramos comprobar experimentalmente que las terminales del transformador están correctamente aisladas, ya que la resistencia entre ellas se encuentra en el orden de $M \Omega$, por lo tento podemos decir que no exixtirá continuidad entre ellas.
	%\section{Referencias}
	

	%\bibliographystyle{plain}
    %\bibliography{Prac2}
    


	
	
\end{document}

