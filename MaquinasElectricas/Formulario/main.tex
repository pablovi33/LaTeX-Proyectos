\documentclass[]{article}
\usepackage[spanish.mexico]{babel}
\usepackage[T1]{fontenc}
\usepackage[utf8]{inputenc}
%\usepackageB.\\{lmodern}
\usepackage[a4paper]{geometry}

%\usepackage{natbib}
\usepackage{cite}


%Grafico de barras
\usepackage{pgfplots}

%Graficos e imagenes
\usepackage{graphicx}


\title{Formulario Máquinas Eléctricas}
\author{Pablo Vivar Colina}
%\date{Mayo 2018}

%EDADES, Para secundaria y prepa

\begin{document}
	
%	%\usepackage[top=2cm,bottom=2cm,left=1cm,right=1cm]{geometry}


\begin{titlepage}
     \begin{center}
	\includegraphics[width=0.09\textwidth]{UNAM}\Large Universidad Nacional Autónoma de México
        	\includegraphics[width=0.09\textwidth]{FI}\\[1cm]
        \Large Facultad de Ingeniería\\[1cm]
       % \Large División de Ciencias Básicas\\[1cm]
         \Large Laboratorio de Dispositivos y Circuitos Electrónicos (6654)\\[1cm]
         %la clave antes era:4314
         \footnotesize Profesor: Zapata Rosales Arturo Ing.\\[1cm]
        \footnotesize Semestre 2018-1\\[1cm]
        %\Large Práctica No. 1\\[1cm]
    
        %\Large Práctica No. 2\\[1cm]
        
        %\Large Práctica No. 3\\[1cm]
       
        %\Large Práctica No. 4\\[1cm]
         
               
         %\Large Práctica No. 5\\[1cm]
         
         
         %\Large Práctica No. 6\\[1cm]
         
         %\Large Práctica No. 7\\[1cm]
         
             %\Large Práctica No. 8\\[1cm]
       

        \Large Práctica No. 9\\[1cm]
        
           %####AQUI VAMOS#### ya ahora sii
           
        %\Large Práctica No. 11\\[1cm]
        %\Large Práctica No. 12\\[1cm]
        %\Large Práctica No. 13\\[1cm]
        
        %\Large Amplificador Operacional como Integrador\\[1cm]
        %\Large{Filtros}\\[1cm]
         %\Large{Medición de  corrientes en un circuito}\\[1cm]
         %practica 4
         %Large{Amplificador operacional como seguidor de voltaje en entrada inversora}\\[1cm]
         %practica5
         %\Large{Amplificador operacional como integrador}
         
         %Practica 7
%Comportamiento de un diodo Zener

\Large Diodo Zener
        
         %Texto a la derecha
          \begin{flushright}
\footnotesize  Grupo 13\\[0.5cm]
\footnotesize Brigada: 7\\[0.5cm]

\footnotesize Vivar Colina Pablo\\[0.5cm]
 \end{flushright}
    %Texto a la izquierda
          \begin{flushleft}
        \footnotesize Ciudad Universitaria Abril de 2018.\\
          \end{flushleft}
         
          
        %\vfill
        %\today
   \end{center}
\end{titlepage}
 %agregar portada

\maketitle

%\tableofcontents  % Write out the Table of Contents

%\listoffigures  % Write out the List of Figures


\section{Entrehierro}

Sección del núcleo del transformador conformada por aire, es importante tenerla en cuenta ya que es un factor al momento de hacer cálculos correspondientes sobre los efectos magnéticos inducidos por las bobinas que conforman al transformador.\\

\section{Permanencia Entrehierro}

\begin{figure}[h!]
	\centering
	\includegraphics[width=0.3\textwidth]{EntreHierro}
	\caption{Diagrama de dimensiones entrehierro}
	\label{Entrehierro}
\end{figure}

\begin{figure}[h!]
	\centering
	\includegraphics[width=0.3\textwidth]{PermanenciaEntrehierro}
    \caption{Permanencia entrehierro}
    \label{permEntrehierro}
\end{figure}

\section{Longitud media Núcleo de transformadores}

La longitud media de un núcleo del transformador es la sección lineal que se encuentra en el centro del laminado y tiene unidades de longitud, para contabilizarla se tiene que restar el entrehierro.\\


\section{Curvas de Magnetización}

\begin{figure}[h!]
	\centering
	\includegraphics[width=1\textwidth]{curvasMagnetizacion}
	\caption{Curvas de Magentización para materiales ferromagnéticos típicos}
	\label{curvasMag}
\end{figure}




\section{Máquina de corriente directa}


\begin{figure}[h!]
	\centering
	\includegraphics[width=1\textwidth]{maquina4polos}
	\caption{}
	\label{maq4polos}
\end{figure}

\begin{figure}[h!]
	\centering
	\includegraphics[width=1\textwidth]{maquinaCD2polos}
	\caption{}
	\label{maq2polos}
\end{figure}

\begin{figure}[h!]
	\centering
	\includegraphics[width=1\textwidth]{densidadFlujoEntrehierro2polos}
	\caption{}
	\label{flujo2polos}
\end{figure}


\begin{figure}[h!]
	\centering
	\includegraphics[width=1\textwidth]{maquinaInduccionEjemplo}
	\caption{Ejemplo maquina de inducción}
	\label{fig:maqInduccion}
\end{figure}





%\bibliographystyle{plain}
%\bibliography{Prac1}



\begin{thebibliography}{9}
	%\bibitem{latexcompanion} 
	%Michel Goossens, Frank Mittelbach, and Alexander Samarin. 
	%\textit{The \LaTeX\ Companion}. 
	%Addison-Wesley, Reading, Massachusetts, 1993.
	
	%\bibitem{einstein} 
	%Albert Einstein. 
	%\textit{Zur Elektrodynamik bewegter K{\"o}rper}. (German) 
	%[\textit{On the electrodynamics of moving bodies}]. 
	%Annalen der Physik, 322(10):891–921, 1905.
	
	
	%\bibitem{ebullicion} 
	 %Punto ebullición,
	%\\\texttt{http://www.cie.unam.mx/~ojs/pub/Liquid3/node8.html}
	
	
	%\bibitem{coccion} 
   %UnComo:Tiempo de cocción de los frijoles,
	%\\\texttt{https://comida.uncomo.com/articulo/tiempo-de-coccion-de-los-frijoles-34777.html}
\end{thebibliography}






\end{document}
